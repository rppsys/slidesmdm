\begin{landscape}
\section{Caso 2 - Na Unidade DEF os consultores se inscrevem para elaborar mas o supervisor precisa consentir}

\subsection*{Configurações da Unidade DEF}

Suponha a existência de uma unidade denominada \textbf{Unidade DEF} configurada da seguinte forma: 

\CONFIGURAUNID{DEF}{\mssim}{\msnao}{\msnao}{\msnao}

\begin{itemize}
	\item Permite auto-atribuição de elaboradores mas não dispensa o consentimento do supervisor.
	\item Não permite auto-atribuição de revisores.
\end{itemize}

A Unidade DEF não é tão liberal quando a Unidade ABC mas não é tão conservadora como a Unidade GHI do terceiro caso. Ela permite a auto-atribuição de elaboradores mas não dispensa a aprovação de um supervisor quando isso acontece.

\subsection*{Descrição do Caso}

\begin{itemize}
	\item \SC, chefe da Unidade DEF, vai liberar a auto-atribuição de elaboradores para a solicitação \SOLT. 
	
	\item Haverão três voluntários para serem elaboradores: \EU, \ED e \EQ. Contudo \SC vai rejeitar a inscrição de \EQ pois julga que ele já está sobrecarregado.

	\item Durante a elaboração, \SC atribuirá \RU para ser revisor já que a auto-atribuição de revisores está desabilitada na Unidade DEF. Ele precisa aceitar. Ele aceita.

	\item No meio do trabalho, \ED desiste pois vai tirar férias e a solicitação segue sendo elaborada somente por \EU.
	
	\item O trabalho termina com \EU fazendo upload dos artefatos e \SC encaminhando a solicitação para a ASSEL.

\end{itemize}

\subsection*{Simulação}

\begin{enumerate}
	\item Solicitação \SOLT chega na unidade com estado \euni{Não Lido};

	\GERSOLUNID{\SOLT}{Gabinete Delmasso}{CONS}{}{}{}{}{Não}{Distribuição}{\euni{Não Lido}}{-}{-}{Indefinido}

	\item \SC, que é supervisor, acessa visualizar fazendo com que o estado mude para \euni{Em Análise};

	\GERSOLUNID{\SOLT}{Gabinete Delmasso}{CONS}{}{}{}{}{Não}{Distribuição}{\euni{Em Análise}}{-}{-}{Indefinido}

	\item Supervisor \SC analisa a solicitação e verifica que sua unidade é a unidade destinada a realizar o trabalho. Assim ele seleciona ``Colocar na Fila''. Assim, o estado da solicitação passa a ser \euni{Na Fila} indicando que a solicitação está liberada para que CLs se inscrevam.

	\GERSOLUNID{\SOLT}{Gabinete Delmasso}{CONS}{}{}{}{}{Não}{Distribuição}{\euni{Na Fila}}{-}{-}{Indefinido}

	\item \EU acessa o módulo e verifica que a solicitação \SOLT está no estado \euni{Na Fila}. Então decide se inscrever neste trabalho para atuar como elaborador. Ele marca a solicitação e clica em \bInscrever. Abre-se um modal para que escolha-se a qualidade da inscrição: Elaborador ou Revisor? Ele escolhe ``Elaborador'' e clica em \textbf{Ok}. 

	\item Neste momento o sistema verifica as configurações da Unidade DEF e verifica que a opção ``Dispensar consentimento para auto-atribuição de elaboradores'' \textbf{não está habilitada}. Então o sistema \textbf{vai pedir que algum supervisor aceite a inscrição} de \EU. O sistema notifica todos os supervisores da Unidade DEF com uma mensagem ``Consultor \EU se inscreveu para atuar na solicitação \SOLT. Esse pedido deve ser consentido por algum supervisor''.
	
	\item Na área de trabalho \EU aparece o seguinte registro:
	
	\AREATRABCL{\SOLT}{Gabinete Delmasso}{PDL}{}{}{}{}{Não}{Distribuição}{\ecl{Aguardando consentimento para elaboração}}{-}{-}{\EU}
	
	\item Supervisor \SC acessa o sistema e verifica no módulo ``Minhas Notificações'' a notificação que todos os supervisores receberam pedindo que algum supervisor consinta e aprove a inscrição de \EU para atuar na solicitação \SOLT como elaborador.
	
	\item Dessa forma o supervisor \SC acessa o botão \bGerInscricoes, procura a solicitação \SOLT e  verifica que \EU pediu para se inscrever como elaborador. Assim, \SC clica no botão ``Aprovar'' e aprova a auto-atribuição de \EU.
	
	\item Assim, como \EU é o primeiro elaborador, o estado da solicitação passa a ser \euni{Em Elaboração}:
	
	\GERSOLUNID{\SOLT}{Gabinete Delmasso}{CONS}{}{}{}{}{Não}{Distribuição}{\euni{Em Elaboração}}{\EU}{-}{Indefinido}
	
	\item Na área de trabalho \EU o registro da solicitação também é atualizado:

	\AREATRABCL{\SOLT}{Gabinete Delmasso}{PDL}{}{}{}{}{Não}{Distribuição}{\ecl{Em Elaboração}}{\EU}{-}{\EU}
	
	\item Assim, \EU começa a trabalhar na solicitação e pede ajuda para seus colegas \ED e \EQ.
	
	\item Então, da mesma forma, cada um no seu computador, \ED e \EQ acessam o módulo de gerenciamento de solicitações da Unidade DEF, marcam a solicitação e clicam em \bInscrever. Ambos escolhem se inscrever na qualidade de ``elaborador'' e clicam em \textbf{Ok}. 
	
	\item E do mesmo modo, o sistema envia as notificações para todos os supervisores da Unidade DEF: ``Consultor \ED se inscreveu para atuar na solicitação \SOLT. Esse pedido deve ser consentido por algum supervisor'' e ``Consultor \EQ se inscreveu para atuar na solicitação \SOLT. Esse pedido deve ser consentido por algum supervisor''.
	
	\item Na área de trabalho \ED temos o seguinte:
	
	\AREATRABCL{\SOLT}{Gabinete Delmasso}{PDL}{}{}{}{}{Não}{Distribuição}{\ecl{Aguardando consentimento para elaboração}}{\EU}{-}{\ED}
	
	\item E na área de trabalho \EQ:

	\AREATRABCL{\SOLT}{Gabinete Delmasso}{PDL}{}{}{}{}{Não}{Distribuição}{\ecl{Aguardando consentimento para elaboração}}{\EU}{-}{\EQ}	

	\item Deste modo, Supervisor \SC acessa o sistema e verifica no módulo ``Minhas Notificações'' as duas notificações que todos os supervisores da Unidade DEF receberam pedindo que algum supervisor aprove as inscrições de \ED e \EQ.
	
	\item Em seguida, supervisor \SC acessa o botão \bGerInscricoes, procura a solicitação \SOLT e  verifica que há duas pendências de pedidos de inscrições não aprovadas: \ED e \EQ pedindo para se inscrever na solicitação como elaboradores. 
	
	\item Inicialmente \SC aprova o pedido de inscrição de \ED. Tão logo essa aprovação é feita o registro da solicitação no módulo de gerenciamento de solicitações da unidade atualiza-se para incluir \ED como elaborador:
	
	\GERSOLUNID{\SOLT}{Gabinete Delmasso}{CONS}{}{}{}{}{Não}{Distribuição}{\euni{Em Elaboração}}{\EU, \ED}{-}{Indefinido}
	
	\item \ED também tem o registro da solicitação em sua área de trabalho atualizada:
	
	\AREATRABCL{\SOLT}{Gabinete Delmasso}{PDL}{}{}{}{}{Não}{Distribuição}{\ecl{Em Elaboração}}{\EU, \ED}{-}{\ED}
	
	\item Contudo, \SC avalia que \EQ já está atuando em outras solicitações mais urgentes e prefere rejeitar o pedido de inscrição de \EQ para atuar como elaborador na solicitação \SOLT. Então, dentro do modal de gerenciamento de inscrições, \SC seleciona o pedido de inscrição de \EQ e rejeita a inscrição. Abre-se uma tela com um campo para \SC motivar a ação e ele escreve ``Pedido rejeitado pois o consultor precisa finalizar solicitações mais urgentes''.
	
	\item Neste momento, o sistema envia uma notificação direta para \EQ dizendo ``\SC rejeitou seu pedido de inscrição na solicitação \SOLT como elaborador. Motivo: Pedido rejeitado pois o consultor precisa finalizar solicitações mais urgentes''. 
	
	\item E assim, o registro da solicitação que antes aparecia na área de trabalho de \EQ com o estado ``Aguardando consentimento para elaboração'' desaparece. O estado do registro da solicitação no módulo de gerenciamento de solicitações da Unidade DEF também não se altera e mantém-se como estava:
	
	\GERSOLUNID{\SOLT}{Gabinete Delmasso}{CONS}{}{}{}{}{Não}{Distribuição}{\euni{Em Elaboração}}{\EU, \ED}{-}{Indefinido}	
	
	\item Continuando, \ED percebe que a solicitação precisa de um revisor. Ele conversa com \RU que aceita ser revisor do trabalho em desenvolvimento.
	
	\item Dessa forma \RU tenta se inscrever na solicitação como revisor. Mas ao clicar no botão \bInscrever o sistema consulta as configurações da Unidade DEF e verifica que a Unidade DEF não permite auto-atribuição de revisores. Então essa opção nem sequer aparece para \RU.  
	
	\item Dessa forma, \RU conversa com \ED que por sua vez conversa com \SC e pede para adicionar \RU como revisor da solicitação que está sendo elaborada. \SC atende o pedido, acessa o módulo de gerenciamento de solicitações da Unidade DEF e clica no botão \bGerAtrib.
	
	\item Na interface do modal de gerenciamento de atribuições, \SC verifica que realmente apenas \EU e \ED atuam na solicitação na qualidade de elaboradores, mas não há ainda nenhum revisor atribuído. Então ele clica no botão ``Adicionar Revisor'' e escolhe o nome de \RU.
	
	\item O modal de gerenciamento de atribuições adiciona o nome de \RU na lista de revisores pendentes uma vez que \RU ainda precisa aceitar essa atribuição em sua área de trabalho. As informações apresentadas para a solicitação \SOLT ainda permanecem inalteradas já que \RU ainda não aceitou a atribuição de revisor:
	
	\GERSOLUNID{\SOLT}{Gabinete Delmasso}{CONS}{}{}{}{}{Não}{Distribuição}{\euni{Em Elaboração}}{\EU, \ED}{-}{Indefinido}	
	
	\item Além disso, uma notificação é enviada diretamente para \RU com a mensagem ``\SC atribuiu você para atuar na solicitação \SOLT na qualidade de revisor. Acesse sua área de trabalho para aceitar ou rejeitar a atribuição''.
	
	\item \RU acessa o sistema e percebe no canto superior direito que há notificações não lidas. Então ele acessa o módulo ``Minhas Notificações'' e lê a notificação. Em seguida, ele acessa sua área de trabalho e se depara com um novo registro:
	
	\AREATRABCL{\SOLT}{Gabinete Delmasso}{PDL}{}{}{}{}{Não}{Distribuição}{\ecl{Aceitar Revisão}}{\EU, \ED}{-}{\RU}	
	
	\item Dessa forma \RU lembra do compromisso que fez com \ED de ser revisor da solicitação \SOLT. Prontamente, ele clica na solicitação em sua área de trabalho e acessa o botão \bAvaliar. Um modal é aberto e ele decide aceitar a atribuição de revisor.
	
	\item Assim, após \RU aceitar a atribuição, as informações apresentadas para a solicitação \SOLT são atualizadas tanto no módulo de gerenciamento de solicitações da Unidade DEF quanto nas áreas de trabalho de todos os envolvidos:
	
	\GERSOLUNID{\SOLT}{Gabinete Delmasso}{CONS}{}{}{}{}{Não}{Distribuição}{\euni{Em Elaboração}}{\EU, \ED}{\RU}{Indefinido}		
	
	\AREATRABCL{\SOLT}{Gabinete Delmasso}{PDL}{}{}{}{}{Não}{Distribuição}{\ecl{Em Elaboração}}{\EU, \ED}{\RU}{\EU}	
	
	\AREATRABCL{\SOLT}{Gabinete Delmasso}{PDL}{}{}{}{}{Não}{Distribuição}{\ecl{Em Elaboração}}{\EU, \ED}{\RU}{\ED}	

	\AREATRABCL{\SOLT}{Gabinete Delmasso}{PDL}{}{}{}{}{Não}{Distribuição}{\ecl{Em Elaboração}}{\EU, \ED}{\RU}{\RU}	
	
	\item Os três trabalham na elaboração dos documentos pedidos na solicitação até \ED, que vai entrar de férias, resolve desistir de atuar como elaborador nessa solicitação.
	
	\item Assim, \ED entra em sua área de trabalho, seleciona a solicitação e clica em \bDesistir. O sistema abre uma janela de confirmação perguntando se o usuário tem certeza que deseja desistir de ser elaborador da solicitação e \ED confirma.
	
	\item Neste momento uma notificação é enviada para os supervisores da Unidade DEF informando que \ED desistiu de ser elaborador da solicitação. Assim, as informações da solicitação no módulo de gerenciamento de solicitações da unidade e nas áreas de trabalho dos consultores é atualizado removendo o nome de \ED da lista de elaboradores:
	
	\GERSOLUNID{\SOLT}{Gabinete Delmasso}{CONS}{}{}{}{}{Não}{Distribuição}{\euni{Em Elaboração}}{\EU}{\RU}{Indefinido}

	\AREATRABCL{\SOLT}{Gabinete Delmasso}{PDL}{}{}{}{}{Não}{Distribuição}{\ecl{Em Elaboração}}{\EU}{\RU}{\EU}	

	\AREATRABCL{\SOLT}{Gabinete Delmasso}{PDL}{}{}{}{}{Não}{Distribuição}{\ecl{Em Elaboração}}{\EU}{\RU}{\RU}	
	
	\item Além disso, essa solicitação desaparece da área de trabalho de \ED já que ele deixou de ser elaborador.
	
	\item Finalmente \EU e \RU finalizam o trabalho. \EU acessa sua área de trabalho no sistema, procura a solicitação \SOLT em que ele está atuando como elaborador e clica em \bConcluir. O sistema verifica que há pelo menos um \CL atuando na qualidade de revisor e permite que o upload dos artefatos finais seja feito.

	\item Após o upload, o sistema altera o estado para \euni{Conclusão da Solicitação} tanto no módulo de gerenciamento quanto nas áreas de trabalho dos consultores:
	
	\GERSOLUNID{\SOLT}{Gabinete Delmasso}{CONS}{}{}{}{}{Não}{Distribuição}{\euni{Conclusão da Solicitação}}{\EU}{\RU}{Indefinido}

	\AREATRABCL{\SOLT}{Gabinete Delmasso}{PDL}{}{}{}{}{Não}{Distribuição}{\ecl{Conclusão da Solicitação}}{\EU}{\RU}{\EU}	

	\AREATRABCL{\SOLT}{Gabinete Delmasso}{PDL}{}{}{}{}{Não}{Distribuição}{\ecl{Conclusão da Solicitação}}{\EU}{\RU}{\RU}	
		
	\item \ST marca a solicitação no módulo de gerenciamento e manda encaminhar para a ASSEL. Assim, a solicitação desaparece da caixa de entrada da unidade e também das áreas de trabalho de cada um dos consultores que participaram da elaboração e revisão do trabalho.
\end{enumerate}

\end{landscape}

\pagebreak


