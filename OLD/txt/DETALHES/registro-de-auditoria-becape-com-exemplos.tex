% \newcommand{\sepCapaProjeto}{Projeto Sistema ASSEL}
% \newcommand{\sepCapaTipo}{\textsc{Funcionalidades e Requisitos}}
% \newcommand{\sepCapaTitulo}{Registro de Auditoria}
% \newcommand{\sepCapaData}{22 de Setembro de 2022}
% \newcommand{\sepCapaVersao}{V1}
% \newcommand{\sepCapaFile}{sep-capa-vinho.pdf}
% \newcommand{\sepCapaCor}{blue}

\chapter*{Alterações}

\section*{29/11/2022}

Melhor detalhamento de como o registro do evento deve ser escrito na seção ``\ref{sec-evento} \nameref{sec-evento}''.


\chapter{Registro de Auditoria}
\label{detalhes:log-auditoria}

Esse documento sugere os requisitos da funcionalidade de registro de auditoria. 

É importante ressaltar que abandonamos o termo ``log de auditoria'' de forma a não criar confusão com outros tipos de log do sistema. 

Assim, preferimos o nome ``registro de auditoria'' para estabelecer uma funcionalidade que será usada pelos administradores do sistema para rastrear as ações dos usuários de forma a facilitar a solução de problemas futuros.

\section{Modelagem do Registro de Auditoria}

De forma que a modelagem da tabela de registro de auditoria no banco de dados seja simples, o registro de auditoria deve estar em uma tabela única sem relacionamentos com nenhuma outra tabela do sistema.

Portanto, o registro de auditoria terá os seguintes atributos elencados nas próximas seções.

\subsection{timestamp}

Data e Hora completa no formato ISO8601.

\subsection{tipo}

Será um caracter que representa o tipo desse evento.

\textbf{Tipo}: 
\begin{itemize}
	\item \textbf{U}: Evento cuja origem é uma ação do usuário;
	\item \textbf{S}: Evento sem relação com usuários (gerado pelo sistema);
\end{itemize}

\subsection{ator}

Identifica o ator do evento.

\subsubsection{Tipo U}

Se o ``tipo'' for \textbf{U} então o campo ator deverá conter o nome de rede do usuário logado no sistema.

Exemplos.: ``\textbf{ronie.porfirio}'',`` \textbf{gilberto.sousa}'', ``\textbf{robson.alencar}'';

\subsubsection{Tipo S}

Se o ``tipo'' for \textbf{S} então o campo ator deverá conter o nome do módulo do sistema que gerou o evento e poderá ser:

\begin{itemize}
	\item ``\textbf{sistema.web}'': Evento tem origem no Sistema ASSEL; 
	\item ``\textbf{sistema.sei}'': Evento tem origem no módulo de integração com o SEI; 
\end{itemize}


\subsection{host}

Nome do Host ou endereço IP da máquina que o ator estava usando para acessar o sistema ou vazio caso não se aplique.

\subsection{classe}

Classe ou categoria de classificação;

É um carácter, ou seja, uma letra maiúscula, que representa a classe ou categoria desse registro e podem ser:

\textbf{Classes}:
\begin{itemize}
	\item \textbf{A - ACESSO}: Usuário obteve \sigla{a}cesso a uma tela/módulo/aba/formulário modal do sistema sem necessariamente realizar qualquer tipo de ação;
	\item \textbf{D - DADO}: Eventos no qual há alteração no banco de \textbf{d}ados;
	\item \textbf{N - NOTIFICAÇÕES}: \textbf{N}otificações emitidos pelos sistema para o usuário, que não são considerados erros;
	\item \textbf{I - INFORMAÇÃO}: Eventos \textbf{I}nformativos que não são considerados erros;
	\item \textbf{W - WARNING}: Eventos de Aviso (\textbf{w}arning) que potencialmente podem se transformar em erros;
	\item \textbf{E - ERRO}: \textbf{E}rro;
	\item \textbf{F - FATAL}: Erro \textbf{f}atal que faz o sistema parar de funcionar. Exemplo: Não conseguir acessar o SEI;
\end{itemize}

\subsection{tela}

Sigla da tela ou da funcionalidade aonde o evento ocorreu.

Serão 4 caracteres que representa a funcionalidade/tela/aba/modal/mensagem onde ocorreu o evento. 

\begin{itemize}
	\item Primeira Letra: Indica o ator que normalmente utilizaria aquela tela do sistema;
	\begin{itemize}
		\item Genérico ou \textbf{I}nício: I
		\item \textbf{S}olicitante: S
		\item \textbf{A}dministrador: A		
		\item A unidade ASS\textbf{E}L: E
		\item \textbf{U}nidades Temáticas da ASSEL: U
		\item Consul\textbf{t}ores Legislativos: T
	\end{itemize}
	\item Segunda Letra: Funcionalidade
	\item Terceira Letra: Tela
	\item Quarta Letra: Aba / Modal / Mensagem
\end{itemize}


%As tabelas \ref{tab:auditoria:telas0}, \ref{tab:auditoria:telas1}, \ref{tab:auditoria:telas2} e \ref{tab:auditoria:telas3} apresentam as siglas possíveis (até o momento).


%\begin{table}[!h]
	\begin{center}
		\begin{tabular}{|p{4cm}|p{1.3cm}|p{5cm}|p{0.8cm}|}
			\hline
			\rowcolor{corCOULD!40} \multicolumn{4}{|c|}{\Large Siglas de Telas \textbf{I}niciais do Sistema \normalsize} \\ \hline
			
			% CABEÇALHO
			\rowcolor{lightgray} Funcionalidade & Tela & Aba / Modal / Mensagem & \textbf{Sigla} \\ \hline
			% CONTEUDO
			% -----			
			\rowcolor{corSIM!30} \textbf{S}istema ASSEL & \sigla{P}rincipal & \sigla{L}ogin & \sigla{ISPL}  \\ \hline
			\rowcolor{corSIM!30} \textbf{S}istema ASSEL & \sigla{P}rincipal & \sigla{P}rincipal & \sigla{ISPP}  \\ \hline
			\rowcolor{corSIM!30} \textbf{S}istema ASSEL & Integração com o \sigla{S}EI & \sigla{P}rincipal & \sigla{ISSP}  \\ \hline
		\end{tabular}    
%		\caption{\label{tab:auditoria:telas0} Siglas de Telas Gerais do Sistema}
	\end{center}
%\end{table}


%\begin{table}[!h]
	\begin{center}
		\begin{tabular}{|p{4cm}|p{1.3cm}|p{5cm}|p{0.8cm}|}
			\hline
			\rowcolor{corCOULD!40} \multicolumn{4}{|c|}{\Large Siglas das Telas \textbf{Administrativas} do Sistema \normalsize} \\ \hline
			
			% CABEÇALHO
			\rowcolor{lightgray} Funcionalidade & Tela & Aba / Modal / Mensagem & Sigla \\ \hline
			% CONTEUDO
			% -----			
			% Administrar Perfis
			\rowcolor{cldfB!30} \textbf{A}dministrar \sigla{P}erfis & \sigla{P}rincipal & \sigla{P}rincipal & \sigla{APPP}  \\ \hline
			\rowcolor{cldfB!30} \textbf{A}dministrar \sigla{P}erfis & \sigla{P}rincipal & Modal \sigla{N}ovo Perfil & \sigla{APPN}  \\ \hline
			\rowcolor{cldfB!30} \textbf{A}dministrar \sigla{P}erfis & \sigla{P}rincipal & Modal \sigla{V}izualizar Perfil & \sigla{APPV}  \\ \hline
			\rowcolor{cldfB!30} \textbf{A}dministrar \sigla{P}erfis & \sigla{P}rincipal & Modal Edi\sigla{t}ar Perfil & \sigla{APPT}  \\ \hline
			\rowcolor{cldfB!30} \textbf{A}dministrar \sigla{P}erfis & \sigla{P}rincipal & Mensagem Tornar Padrão \sigla{I}nterno & \sigla{APPI}  \\ \hline
			\rowcolor{cldfB!30} \textbf{A}dministrar \sigla{P}erfis & \sigla{P}rincipal & Mensagem Tornar Padrão \sigla{E}xterno & \sigla{APPE}  \\ \hline
			\rowcolor{cldfB!30} \textbf{A}dministrar \sigla{P}erfis & \sigla{P}rincipal & Mensagem E\sigla{x}cluir & \sigla{APPX}  \\ \hline


			% Administrar Usuários
			\rowcolor{cldfA!30} \textbf{A}dministrar \sigla{U}suários & \sigla{P}rincipal & \sigla{P}rincipal & \sigla{AUPP}  \\ \hline
			\rowcolor{cldfA!30} \textbf{A}dministrar \sigla{U}suários & \sigla{P}rincipal & Modal \sigla{N}ovo Usuário & \sigla{AUPN}  \\ \hline
			\rowcolor{cldfA!30} \textbf{A}dministrar \sigla{U}suários & \sigla{P}rincipal & Modal \sigla{V}isualizar Usuário & \sigla{AUPV}  \\ \hline
			\rowcolor{cldfA!30} \textbf{A}dministrar \sigla{U}suários & \sigla{P}rincipal & Modal Edi\sigla{t}ar Usuário & \sigla{AUPT}  \\ \hline
			\rowcolor{cldfA!30} \textbf{A}dministrar \sigla{U}suários & \sigla{P}rincipal & Modal Vincular Per\sigla{f}il de Usuário & \sigla{AUPF}  \\ \hline
			\rowcolor{cldfA!30} \textbf{A}dministrar \sigla{U}suários & \sigla{P}rincipal & Mensagem \sigla{A}tivar Usuário & \sigla{AUPA}  \\ \hline
			\rowcolor{cldfA!30} \textbf{A}dministrar \sigla{U}suários & \sigla{P}rincipal & Mensagem \sigla{I}nativar Usuário & \sigla{AUPI}  \\ \hline
			\rowcolor{cldfA!30} \textbf{A}dministrar \sigla{U}suários & \sigla{P}rincipal & Mensagem E\sigla{x}cluir Usuário & \sigla{AUPX}  \\ \hline

			% Administrar Unidades
			\rowcolor{cldfC!30} \textbf{A}dministrar U\sigla{n}idades & \sigla{P}rincipal & \sigla{P}rincipal & \sigla{ANPP}  \\ \hline
			\rowcolor{cldfC!30} \textbf{A}dministrar U\sigla{n}idades & \sigla{P}rincipal & Modal \sigla{N}ova Unidade & \sigla{ANPN}  \\ \hline
			\rowcolor{cldfC!30} \textbf{A}dministrar U\sigla{n}idades & \sigla{P}rincipal & Modal \sigla{V}isualizar Unidade & \sigla{ANPV}  \\ \hline
			\rowcolor{cldfC!30} \textbf{A}dministrar U\sigla{n}idades & \sigla{P}rincipal & Modal Edi\sigla{t}ar Unidade & \sigla{ANPT}  \\ \hline
			\rowcolor{cldfC!30} \textbf{A}dministrar U\sigla{n}idades & \sigla{P}rincipal & Mensagem E\sigla{x}cluir Unidade & \sigla{ANPX}  \\ \hline
		\end{tabular}    
%		\caption{\label{tab:auditoria:telas1} Siglas de Telas Administrativas}
	\end{center}
%\end{table}


% \begin{table}[!h]
	\begin{center}
		\begin{tabular}{|p{4cm}|p{1.3cm}|p{5cm}|p{0.8cm}|}
			\hline
			\rowcolor{corCOULD!40} \multicolumn{4}{|c|}{\Large Siglas das Telas de \textbf{Solicitantes} do Sistema \normalsize} \\ \hline
			
			% CABEÇALHO
			\rowcolor{lightgray} Funcionalidade & Tela & Aba / Modal / Mensagem & Sigla \\ \hline
			% CONTEUDO
			% -----			
			
			% Solicitar Ordem de Serviço
			\rowcolor{cldfE!30} \textbf{S}olicitar \textbf{O}rdem de Serviço & \sigla{P}rincipal & \sigla{P}rincipal & \sigla{SOPP}  \\ \hline
			\rowcolor{cldfE!30} \textbf{S}olicitar \textbf{O}rdem de Serviço & \sigla{P}rincipal & Modal \sigla{N}ova Solicitação & \sigla{SOPN}  \\ \hline
			\rowcolor{cldfE!30} \textbf{S}olicitar \textbf{O}rdem de Serviço & \sigla{P}rincipal & Modal \sigla{V}izualizar Solicitação & \sigla{SOPV}  \\ \hline
			\rowcolor{cldfE!30} \textbf{S}olicitar \textbf{O}rdem de Serviço & \sigla{P}rincipal & Mensagem \sigla{C}ancelar Solicitação & \sigla{SOPC}  \\ \hline							
			
		\end{tabular}    
% 		\caption{\label{tab:auditoria:telas2} Siglas de Telas do Solicitante}
	\end{center}
% \end{table}


% \begin{table}[!h]
	\begin{center}
		\begin{tabular}{|p{4cm}|p{1.3cm}|p{5cm}|p{0.8cm}|}
			\hline
			\rowcolor{corCOULD!40} \multicolumn{4}{|c|}{\Large Siglas das Telas da \textbf{ASSEL} \normalsize} \\ \hline
			
			% CABEÇALHO
			\rowcolor{lightgray} Funcionalidade & Tela & Aba / Modal / Mensagem & Sigla \\ \hline
			% CONTEUDO
			% -----			
			
			% Gerenciar e Analisar Solicitações - Principal
			\rowcolor{cldfF!30} ASS\textbf{E}L Gerenciar e Analisar S\textbf{o}licitações  & \sigla{P}rincipal & \sigla{P}rincipal & \sigla{EOPP}  \\ \hline
			\rowcolor{cldfF!30} ASS\textbf{E}L Gerenciar e Analisar S\textbf{o}licitações  & \sigla{P}rincipal & Modal Atribuir \sigla{D}estino & \sigla{EOPD}  \\ \hline
			\rowcolor{cldfF!30} ASS\textbf{E}L Gerenciar e Analisar S\textbf{o}licitações  & \sigla{P}rincipal & Modal Atribuir \sigla{A}tribuição & \sigla{EOPA}  \\ \hline

			% Gerenciar e Analisar Solicitações - Detalhes
			\rowcolor{cldfG!20} ASS\textbf{E}L Gerenciar e Analisar S\textbf{o}licitações  & \sigla{D}etalhar & \sigla{P}rincipal & \sigla{EODP}  \\ \hline

			\rowcolor{cldfH!20} ASS\textbf{E}L Gerenciar e Analisar S\textbf{o}licitações  & \sigla{A}nalisar & \sigla{P}rincipal & \sigla{EOAP}  \\ \hline
			\rowcolor{cldfH!20} ASS\textbf{E}L Gerenciar e Analisar S\textbf{o}licitações  & \sigla{A}nalisar & Aba Proprie\textbf{d}ades & \sigla{EOAD}  \\ \hline
			\rowcolor{cldfH!20} ASS\textbf{E}L Gerenciar e Analisar S\textbf{o}licitações  & \sigla{A}nalisar & Aba Proto\textbf{c}olo & \sigla{EOAC}  \\ \hline
			\rowcolor{cldfH!20} ASS\textbf{E}L Gerenciar e Analisar S\textbf{o}licitações  & \sigla{A}nalisar & Aba Propo\textbf{s}ições Relacionadas & \sigla{EOAS}  \\ \hline
			\rowcolor{cldfH!20} ASS\textbf{E}L Gerenciar e Analisar S\textbf{o}licitações  & \sigla{A}nalisar & Aba Solicitações \textbf{R}elacionadas & \sigla{EOAR}  \\ \hline
			\rowcolor{cldfH!20} ASS\textbf{E}L Gerenciar e Analisar S\textbf{o}licitações  & \sigla{A}nalisar & Aba Adição de O\textbf{b}servações & \sigla{EOAB}  \\ \hline
			\rowcolor{cldfH!20} ASS\textbf{E}L Gerenciar e Analisar S\textbf{o}licitações  & \sigla{A}nalisar & Aba Adição de Ar\textbf{q}uivos & \sigla{EOAQ}  \\ \hline
			\rowcolor{cldfH!20} ASS\textbf{E}L Gerenciar e Analisar S\textbf{o}licitações  & \sigla{A}nalisar & Aba Enca\textbf{m}inhamento & \sigla{EOAM}  \\ \hline			
		\end{tabular}    
% 		\caption{\label{tab:auditoria:telas3} Siglas de Telas da ASSEL}
	\end{center}
% \end{table}


\subsection{Evento}
\label{sec-evento}

Descrição textual do evento com entidades envolvidas em chaves e objetos envolvidos em colchetes e seus IDs em parenteses quando existir. 

\begin{center}
 \Large
	Verbo de Ação + \{nome da entidade\}[texto do objeto](ID)
\end{center}

\subsubsection{Verbo de Ação}

O texto de evento deve iniciar com um verbo de ação conjugado no passado com a primeira letra em maiúscula e o restante da palavra em minúsculas. Exemplos.:  Acessou, Adicionou, Ativou, Selecionou, Vinculou, Excluiu.

\subsubsection{Nome da Entidade}
	Após o verbo de ação, deve haver um espaço e, em seguida, a entidade entre chaves em letras minúsculas. 
	
	No paradigma de orientação a objetos a entidade equivale à classe de um objeto. Em um banco de dados, a classe seria equivalente ao nome da tabela que armazena registros.      

	Alguns exemplos de entidades que deverão aparecer nos cenários de BDDs foram listados a seguir:
	
	\begin{itemize}
		\item \textbf{anexo}: Refere-se a um registro da tabela de anexos do banco de dados;

		\item \textbf{funcionalidade}: Refere-se a um registro da tabela de funcionalidades do banco de dados. \textbf{São os módulos ou telas do sistema};

		\item \textbf{minuta\_parecer}: Refere-se a um registro da tabela de minutas de pareceres do banco de dados;

		\item \textbf{ordem\_servico}: Refere-se a um registro da tabela de ordens de serviços do banco de dados;

		\item \textbf{perfil}: Refere-se a um registro da tabela de perfis do banco de dados;

		\item \textbf{solicitante}: Refere-se a um registro da tabela de solicitantes do banco de dados;

		\item \textbf{unidade}: Refere-se a um registro da tabela de unidades do banco de dados;

		\item \textbf{usuario}: Refere-se a um registro da tabela de usuários do banco de dados;

		\item \textbf{botao}: Refere-se a botões que são clicados no sistema;
		 		
		\item \textbf{modal}: Refere-se aos modais que são acessados dentro do sistema;
		
		\item \textbf{mensagem}: Refere-se às mensagens que são exibidas;
		
	\end{itemize}



\subsubsection{Texto do Objeto}
	Após o nome da entidade, em seguida sem espaços, o texto do objeto entre colchetes. 
	
	No paradigma de orientação a objetos o texto do objeto equivale ao objeto que é instanciado a partir de uma classe. Em um banco de dados, o objeto seria equivalente ao nome do registro particular de uma tabela. 
	
	O texto do objeto deve ser o mais fiel possível ao texto do registro no banco de dados ou ao do componente visual do protótipo do sistema. Ver exemplos.
	
\subsubsection{ID}
	 Finalmente, se existir e se o cenário definir, o código do registro entre parênteses. 

\subsubsection{Exemplos}

	\begin{itemize}
		\item Acessou \{funcionalidade\}[Administrar Usuários](1);	
		\item Acessou \{modal\}[Modal de Novo Usuário];	
		\item Ativou \{botao\}[Novo Usuario];	
		\item Adicionou \{unidade\}[SEASI](20);	
		\item Adicionou \{usuario\}[robson.alencar](5);	
	\end{itemize}



\section{Limpeza do Banco de Dados por período (ou número de registros)}

Sempre que o número de registros dessa tabela ultrapassar um número considerável de registros, o sistema deverá manter apenas os registros de auditoria mais recentes  (com até alguns meses de idade). Os demais dados deverão ser exportados para CSV e deletados do banco de dados.


\chapter{Exemplos de Registros de Auditoria}

% ----------------------------------
% Comando Tabela
% ----------------------------------
\newcommand{\tbLogEx}[9][U]{
	\subsection{#2}
	
	\subsubsection*{Descrição do Exemplo:}	
	
	#9
	
	\begin{tabular}{|m{2cm}|m{0.4cm}|m{2.1cm}|m{2.2cm}|m{0.65cm}|m{0.6cm}|}
		\hline
		% CABEÇALHO 1        
		\rowcolor{clLogF} \cellcolor{clLogA!40} \textbf{Timestamp} & \textbf{Tipo} & \textbf{Ator} & \textbf{Host} & \cellcolor{clLogB} \textbf{Classe} & \cellcolor{clLogB} \textbf{Tela} \\ \hline
		% CONTEÚDO 1
		\rowcolor{clLogF} \cellcolor{clLogA!40} #3 & #1 & #4 & #5 & \cellcolor{clLogB} #6 & \cellcolor{clLogB} #7 \\ \hline
	\end{tabular}

	\subsubsection*{Texto da coluna ``Evento'' da tabela:}

	``#8''

	\subsubsection*{Exemplo de Cenário de BDD}	

	\begin{env-auditoria}{Exemplo: #2}
		\begin{enumerate}
			\item (...)
			\item ...
			\item \textbf{E} adiciona o registro de auditoria:
			\item | Timestamp | Tipo | Ator | Host | Classe | Tela | Evento |			
			\item | \$timestamp\$ | #1 | \$usuário logado\$ | \$host\$ | #6 | #7 | ``#8'' |
		\end{enumerate}
	\end{env-auditoria}
}


\section{Exemplos de Categoria A - Acesso}

% ----- Exemplo Inicio ------------------------------------
\tbLogEx[U]{Usuário faz login}
{2022-09-20 15:53:58}
{ronie.porfirio}
{MASI1234}
{A}
{SSPL}
{Login \{usuario\}[ronie.porfirio](4)}
{Usuário faz login no sistema com êxito.}
% ----- Exemplo Fim ------------------------------------

% ----- Exemplo Inicio ------------------------------------
\tbLogEx[U]{Usuário Acessa Tela ``Administrar Unidade''}
{2022-09-20 16:47:10}
{ronie.porfirio}
{MSEASI-1234}
{A}
{AUPP}
{Acessou \{funcionalidade\} [administrar unidade]}
{Usuário acessa na barra lateral o botão para acessar a tela de administração de unidades.}
% ----- Exemplo Fim ------------------------------------

% ----- Exemplo Inicio ------------------------------------
\tbLogEx[U]{Clica no botão novo da tela de administração de unidades}
{2022-09-20 17:43:25}
{ronie.porfirio}
{MSEASI-1234}
{A}
{AUPN}
{Exibiu \{modal\} [novo usuário]}
{Usuário clica no botão novo da tela de administração de unidades e acessa modal ``Dados da Unidade''}
% ----- Exemplo Fim ------------------------------------

% ----- Exemplo Inicio ------------------------------------
\tbLogEx[U]{Clica no botão excluir da tela de administração de perfis}
{2022-09-20 17:43:25}
{gilberto.sousa}
{MSASSEL-4321}
{A}
{APPX}
{Exibição de \{mensagem\} [excluir usuário]}
{Usuário clica no botão novo da tela de administração de unidades e acessa modal ``Dados da Unidade''}
% ----- Exemplo Fim ------------------------------------

\section{Exemplos de Categoria I - Informação}

% ----- Exemplo Inicio ------------------------------------
\tbLogEx[U]{Em Solicitar Ordem de Serviço, altera o combobox da Unidade Solicitante}
{2022-09-21 12:43:25}
{gilberto.sousa}
{MSASSEL-4321}
{I}
{SOPP}
{Troca para \{unidade\}[LABHINOVA]}
{Usuário está na tela de Solicitar Ordem de Serviço, e altera o combobox da Unidade Solicitante selecionando a unidade LABHINOVA}
% ----- Exemplo Fim ------------------------------------


\section{Exemplos de Categoria D - Alteração no Banco de Dados}

\section{Exemplos de Categoria N - Notificações}

\section{Exemplos de Categoria W - Warning}

\section{Exemplos de Categoria E - Erro}

\section{Exemplos de Categoria F - Erro Fatal}

% ----- Exemplo Inicio ------------------------------------
\tbLogEx[U]{Ao acessar sistema, o sistema não consegue se conectar ao SEI}
{2022-09-21 12:43:25}
{robson.alencar}
{MSASSEL-4321}
{F}
{SSPP}
{Sem comunicação com o SEI-CLDF}
{Usuário acaba de fazer login mas sistema não consegue se conectar com o SEI.}
% ----- Exemplo Fim ------------------------------------




