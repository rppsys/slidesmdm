\chapter{Mapeamento de Andamentos}
\label{detalhes:mapeamento-andamentos}

\section*{Alterações}

% Não colocar dentro de table para não recriar a numeração e mudar os números de tabelas já existentes

\begin{center}
	\begin{tabular}{|c|p{0.4\textwidth}|c|}
		\hline
		\rowcolor{lightgray!50} \multicolumn{3}{|c|}{\Large Alterações Mapeamento de Andamentos \normalsize} \\ \hline \hline
		% CABEÇALHO        
		\rowcolor{lightgray}\textbf{Data e Hora} & \textbf{Alterações} & \textbf{Ref Hash}  \\ \hline
		% CONTEÚDO
		% Código escrito manualmente
		\rowcolor{corCOULD!10} 18/07/2022 11:30:00 & Criação do Documento & 0a15c04 \\ \hline			
	\end{tabular}    
\end{center}

\section{Introdução}

O ``Andamento'' é um conceito importante no Sistema ASSEL pois ele identifica quando uma Solicitação tramita de uma unidade para outra dentro do Sistema ASSEL.

O BDD\_02\_JANELA\_DETALHAR.feature recém criado peça THS modificou as propriedades de um ``Andamento''.

A partir de agora um ``Andamento'' contém todas as informações da ``Caixa de Informações do Andamento'' que é o grupo de componentes da Tela ``Gerenciar e Analisar OS pela ASSEL''.

\textbf{Atributos de um Andamento}:

\begin{itemize}
	\item Unidade Remetente
	\item Usuário Remetente
	\item Unidade de Destino
	\item \textbf{Andamento} - Andamento propriamente dito.
	\item Mensagem
	\item Data e Hora do Andamento
\end{itemize}

O \textbf{Andamento propriamente dito} é um atributo que possui opções \emph{hard-coded} no sistema que o usuário pode escolher no momento de definir um Encaminhamento.

O \textbf{encaminhamento} de hoje será o \textbf{andamento} de amanhã.

Dessa forma é preciso mapear quais serão os ``Andamentos'' possíveis dentro do Sistema ASSEL.

O objetivo desse documento é esse.

\section{Mapeamento de Andamentos}

\begin{table}[h]
	\begin{center}
		\begin{tabular}{|c|c|p{0.5\textwidth}|}
			\hline
			\rowcolor{lightgray} \multicolumn{3}{|c|}{\Large Mapeamento de Andamentos \normalsize} \\ \hline \hline
			% CABEÇALHO        
			\rowcolor{lightgray}\textbf{UNIDADE} & \textbf{UNIDADE} &   \\ 
			\rowcolor{lightgray}\textbf{REMETENTE} & \textbf{DESTINO} & \textbf{ANDAMENTO} \\ 
			\rowcolor{lightgray}\textbf{(DE)} & \textbf{(PARA)} &  \\ \hline
			% CONTEÚDO
			% Código escrito manualmente
			\rowcolor{corCOULD!10} SOLICITANTE & ASSEL & NOVA SOLICITAÇÃO \\ \hline
			\rowcolor{corCOULD!10} SOLICITANTE & ASSEL & RESOLUÇÃO DE PENDÊNCIA \\ \hline
			\rowcolor{corSHOULD!20} ASSEL & SOLICITANTE & ENTREGAR SOLICITAÇÃO CONCLUÍDA \\ \hline
			\rowcolor{corSHOULD!20} ASSEL & SOLICITANTE & NOTIFICAÇÃO DE DESCONTINUIDADE \\ \hline
			\rowcolor{corSHOULD!20} ASSEL & SOLICITANTE & SOLICITAÇÃO DE PENDÊNCIA \\ \hline
			\rowcolor{corSHOULD!20} ASSEL & UNIDADE & DISTRIBUIÇÃO \\ \hline
			\rowcolor{corSHOULD!20} ASSEL & UNIDADE & SOLICITAÇÃO DE REVISÃO \\ \hline			
			\rowcolor{corMUST!20} UNIDADE & ASSEL & CONCLUSÃO DA SOLICITAÇÃO \\ \hline
			\rowcolor{corMUST!20} UNIDADE & ASSEL & RETORNO DA UNIDADE \\ \hline
		\end{tabular}    
		\caption{\label{tab:mapeamento-andamentos:andamentos:todos} Todos os Andamentos}
	\end{center}
\end{table}

\textbf{Observações}

\begin{itemize}
	\item Os andamentos disponíveis no Sistema ASSEL para escolha dentro de telas usadas pela Unidade ASSEL serão apenas aquelas cujo remetente será a própria Unidade ASSEL.
\end{itemize}