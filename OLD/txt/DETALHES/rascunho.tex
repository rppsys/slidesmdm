\chapter{Rascunho}

	Quais são as edições que precisam ser feitas?
	
	\subsubsection{Ações}
	\textbf{Apoio ou Supervisor - Análises}:
	\begin{itemize}
		\item Analisar se tem Solicitação Duplicada;
		\item Atribuir Urgência para a Solicitação;
		\item Indicar para qual Unidade a Solicitação deve ser Distribuída;		
	\end{itemize}
	
	
	
	\textbf{Só o Supervisor}:
	\begin{itemize}
		\item Aprovar e executar algo que foi Analisado. 
		\item Atribuir Solicitação para um Servidor da ASSEL analisar ela;
	\end{itemize}


	\subsubsection{Desfechos Possíveis:}
	\begin{itemize}
		\item Se vem do Solicitante:
		\begin{itemize}
			\item Cancelamento - Devolve para Solicitante
			\item Pendencia - Devolve para Solicitante
			\item Distribuição - Entrega para alguma Unidade
		\end{itemize}

		\item Se vem do Retorno da Unidade
		\begin{itemize}
			\item Cancelamento - Devolve para Solicitante
			\item Pendencia - Devolve para Solicitante
			\item (Re)Distribuição - Entrega para alguma Unidade
		\end{itemize}

		\item Se vem do Concluso de Unidade:
		\begin{itemize}
			\item CONCLUSAO - Notifica Solicitante 
		\end{itemize}

	\end{itemize}


	\subsubsection{Itens da Coluna Ações}
	
	
	\toGil{Tem que ver o que o Supervisor faz para separar do que o Apoio faz}.
	

	
	\begin{itemize}
		\item \textbf{ANALISAR} - Solicitação que chegou aqui sempre chega com essa ação. Ou seja, deve ser analisado pela ASSEL.
		
		\item \textbf{CANCELAR} - Resultado de uma análise em que a ASSEL entende que a Solicitação deve ser Cancelada por qualquer motivo (Duplicidade, Pedido do Solicitante, Etc...)
		
		\item \textbf{PENDENCIA} - Resultado da análise na qual entende-se que o Solicitante precisa fornecer uma informação adicional para que a Solicitação possa seguir em frente.
		
		\item \textbf{ENVIAR PARA UCJ} - Resultado da análise na qual entende-se que a Solicitação deve ser enviado à UCJ.
		
		\item \textbf{ENVIAR PARA URP} - Resultado da análise na qual entende-se que a Solicitação deve ser enviado à URP.
		
		\item \textbf{ENVIAR PARA UEF} - Resultado da análise na qual entende-se que a Solicitação deve ser enviado à UEF.
		
		\item \textbf{ENVIAR PARA USE} - Resultado da análise na qual entende-se que a Solicitação deve ser enviado à USE.
		
		\item \textbf{ENVIAR PARA UDA} - Resultado da análise na qual entende-se que a Solicitação deve ser enviado à UDA.
		
		\item \textbf{CONCLUIR} - Resultado da análise na qual entende-se que a Unidade Interna concluiu o trabalho e portanto os artefatos podem ser entregues ao Solicitante.
	\end{itemize}


	\subsubsection{Resumo: Funcionalidades}
	
	
	As sub-funcionalidades que devem ter são:
	\begin{itemize}
		\item Atribuir Servidor - Forma de Atribuir a Solicitação a um determinado Usuário da Unidade ASSEL.
		
		\item Atribuir Ação - Forma de Escolher dentre os itens da coluna ação listadas anteriormente qual Ação deve ser realizada.
		
		Se um usuário resolve atribuir uma ação ele mesmo (sem que ele seja o servidor atribuído para fazer a análise)  o sistema atribui o Servidor Atribuido a esse usuário.
		
		
		\item Executar Ação - Executar a Ação atribuída à Solicitação.		
	\end{itemize}
		
	


	Se vem do Solicitante:
		Apoio: Verifica Requisitos
			- Duplicidade => Cancela;
			- Pendencia => Pendencia;
			- Está Válido;
		Chefe: Sugere Unidade para Distribuir
		Chefe: Executa
		
	Se vem da Unidade como Retorno
		Apoio: Verifica Requisitos
			- Cancelamento
			- Pendencia => Pendencia;
			- Redistribuição
			Chefe: Sugere Unidade para (Re)Distribuir
			Chefe: Executa
			
	Se vem da Unidade como Serviço Elaborado
		Apoio:  
		Chefe: "De Acordo"
		Chefe: Notifica Solicitante
		
		
		
	Apoio:
		Cancelamento;
		Pendencia;
		Validação;
		Redistribuição;
	
	Chefe: 
		Sugere Unidade para Distribuir
		Dá o De Acordo
