\chapter{Simulações de Interações}
\label{detalhes:modulos-simula}

\newcommand{\euni}[1]{\colorbox{yellow!50}{\textbf{#1}}}
\newcommand{\ecl}[1]{\colorbox{orange!50}{\textbf{#1}}}


\begin{nota}{Capítulo Desatualizado!}
	Esse documento foi desenvolvido como ponto de partida para a concepção dos protótipos necessários para atingir os objetivos.
	
	Com a realização de reuniões com representantes das unidades temáticas da Assessoia Legislativa, alguns casos de simulações elaborados neste capítulo já podem estar desatualizados!
	
	\tcblower
	
	Contudo, esse capítulo foi mantido pois os casos de simulação ajudam o leitor a compreender as funcionalidades pretendidas.
\end{nota}

\section*{Introdução}

Nesta seção vamos descrever alguns casos de uso simulando as interações entre usuários com os papéis de \textbf{Supervisor} e \textbf{Consultor Legislativo}.

Sabemos que haverão pelo menos três módulos necessários para possibilitar essa interação que foram apresentados nos capítulos anteriores:

\begin{itemize}
	\item Gerenciamento de Solicitações da Unidade;
	\item Área de Trabalho do Consultor Legislativo;
	\item Minhas Notificações
\end{itemize}


\begin{landscape}
\subsection*{Gerenciamento de Solicitações da Unidade}

	Trata-se do Módulo de acesso \textbf{compartilhado} entre todos os usuários de uma mesma unidade quer possuam o papel de supervisor ou não.  

	As solicitações neste módulo são apresentados numa tabela com colunas diversas. Nesta simulação vamos focar nas seguintes colunas:
	
	\begin{itemize}
		\item Solicitação: Mostra o identificador da solicitação;
		\item Estado: Mostra o estado da solicitação no módulo de gerenciamento;
		\item Elaborador(es): Mostra os nome(s) dos consultor(es) atribuídos a essa solicitação na qualidade de \textbf{elaborador}.
		\item Revisor(es): Mostra os nome(s) dos consultor(es) atribuídos a essa solicitação na qualidade de \textbf{revisor}.
	\end{itemize}
	
	Além disso, o módulo de gerenciamento de solicitações da unidade apresentam os seguintes botões:
	
	\begin{itemize}
		\item \bAnalisar: Botão de uso exclusivo de supervisores vai abrir o modal de ``Análise''. Esse modal será parecido com o desenvolvido no módulo do apoio. Será usado para o supervisor decidir se coloca a solicitação na fila para elaboração ou se devolve a solicitação para a ASSEL com o estado de ``RETORNO DA UNIDADE'' caso o supervisor acredite que a solicitação não tem condições de ser elaborada. 

		\item \bGerAtrib: Botão de uso exclusivo de supervisores vai abrir o modal de ``Gerenciamento de Atribuições''. Esse modal deve listar para a solicitação selecionada quem são os consultores atribuídos como elaboradores, revisores. Permitirá fazer atribuições também.
		
		\item \bGerInscricoes: Botão de uso exclusivo de supervisores vai abrir o modal de ``Gerenciamento de Inscrições''. Esse modal deve listar todas as solicitações que possuam pedidos de inscrições não aprovados.
		
		\item \bInscrever: Botão utilizado pelo \CL para se auto-atribuir uma solicitação. Ele abre um modal para que se escolha a qualidade: elaborador ou revisor.
		
	\end{itemize}

	Em um dado momento o módulo de ``Gerenciar Solicitações Unidade'' vai apresentar diversas solicitações presentes na Unidade: 

	\begin{center}
		\scalebox{0.8}{
		\begin{tabularx}{22cm}{|c c|c|X|X|}
			\hline
			\rowcolor{yellow!70} \multicolumn{5}{|c|}{ \textbf{Gerenciar Solicitações Unidade}} \\ \hline
			
			% CABEÇALHO
			\rowcolor{yellow!20} & \textbf{Solicitação} & \textbf{Estado} & \textbf{Elaborador(es)} & \textbf{Revisor(es)} \\ \hline
			
			% CONTEUDO
			\rowcolor{cldfG!10} \msnao & EST0002-2023 & \euni{Não Lido} & - & - \\ \hline
			\rowcolor{cldfG!10} \msnao & PL1700F006-2021 & \euni{Não Lido} & - & - \\ \hline			
			\rowcolor{cldfG!10} \msnao & CONS008-2023 & \euni{Na Fila} & - & - \\ \hline			
			\rowcolor{cldfG!10} \msnao & PL1700S007-2021 & \euni{Na Fila} & - & - \\ \hline			
			\rowcolor{cldfG!10} \mssim & PLC73F001-2021 & \euni{Em Elaboração} & \EU, \ET, \EQ & - \\ \hline			
			\rowcolor{cldfG!10} \mssim & DEN1994P002-2021 & \euni{Em Elaboração} & \EC & \RT \\ \hline			
			\rowcolor{cldfG!10} \msnao & PDL279S001-2022 & \euni{Solicitação Concluída} & \EU, \ED & \RD \\ \hline			
			% -----			
		\end{tabularx}   
		\begin{tabularx}{22cm}{c}
			\rowcolor{white} \bVisualizar \\ 
			\rowcolor{white} \bDetalhes \\ 
			\rowcolor{white} \bAnalisar \\ 
			\rowcolor{white} \bGerAtrib \\ 			
			\rowcolor{white} \bGerInscricoes \\ 	
			\rowcolor{white} \bInscrever \\ 
			% -----			
		\end{tabularx}   
	}
	\end{center}

	Contudo, nesta simulação, não vamos mostrar a tabela inteira. Vamos focar em apenas um registro dessa tabela de cada vez. Também iremos ocultar os botões do lado direito. 
	
	Assim, por exemplo, em um dado momento o estado da solicitação \textbf{PLC73F001-2021} será mostrada assim:
	
	\GERSOLUNID{PLC73F001-2021}{Gabinete Delmasso}{}{}{}{}{}{Não}{Distribuição}
	{\euni{Em Elaboração}}{\EU, \ET, \EQ}{-}{Indefinido}

	Dessa forma, a tabela indica que em um determinado momento a solicitação \textbf{PLC73F001-2021} encontra-se no estado \textbf{``Em Elaboração''}, possui três consultores legislativos atribuídos a ela na qualidade de \textbf{elaboradores} e nenhum consultor atribuído na qualidade de \textbf{revisor}.
	
	\pagebreak
	
\subsection*{Área de Trabalho do Consultor Legislativo}
	
	Trata-se do Módulo de acesso \textbf{exclusivo} de um consultor legislativo onde ele poderá ver qual é sua carga de trabalho atual, isto é, quais são as solicitações atribuídas a ele e em qual qualidade (elaborador ou revisor).

	Da mesma forma que no módulo anterior, as solicitações neste módulo são apresentados numa tabela com as mesmas colunas a saber:
	
	\begin{itemize}
		\item Solicitação;
		\item Estado;
		\item Elaborador(es);
		\item Revisor(es);
	\end{itemize}	

	Os botões, já descritos no capítulo destinado à área de trabalho dos consultores, são os listados abaixo:
	
	\begin{itemize}
		\item \bVisualizar
		\item \bDetalhes
		\item \bAvaliar
		\item \bDesistir
		\item \bConcluir
	\end{itemize}		

	Contudo, é importante destacar que o conjunto de estados possíveis em cada módulo serão diferentes.
	
	Dessa forma, para facilitar o entendimento, estados que vão aparecer no módulo de gerenciamento serão apresentados com a cor de fundo \euni{amarela} enquanto estados que vão aparecer no módulo da área de trabalho serão apresentados na cor de fundo \ecl{laranja}.	
	
	\pagebreak
	
	Por exemplo, a ``Área de Trabalho'' do consultor legislativo \EU em um dado momento poderia ter os seguintes registros:

	\begin{center}
	\scalebox{0.8}{
		\begin{tabularx}{22cm}{|c c|c|X|X|}
			\hline
			\rowcolor{orange!70} \multicolumn{5}{|c|}{ \textbf{Área de Trabalho de \EU}} \\ \hline
			
			% CABEÇALHO
			\rowcolor{orange!20} & \textbf{Solicitação} & \textbf{Estado} & \textbf{Elaborador(es)} & \textbf{Revisor(es)} \\ \hline
			
			% CONTEUDO
			\rowcolor{cldfG!10} \msnao & PL1700S006-2021 & \ecl{Aceitar Elaboração} & - & - \\ \hline			
			\rowcolor{cldfG!10} \msnao & PL1700S007-2021 & \ecl{Aceitar Revisão} & - & - \\ \hline			
			\rowcolor{cldfG!10} \mssim & PLC73F001-2021 & \ecl{Em Elaboração} & \EU, \ET, \EQ & - \\ \hline			
			\rowcolor{cldfG!10} \msnao & PDL279S001-2022 & \ecl{Solicitação Concluída} & \EU, \ED & \RD \\ \hline			
			% -----			
		\end{tabularx}   
		\begin{tabularx}{22cm}{c}
			\rowcolor{white} \bVisualizar \\ 
			\rowcolor{white} \bDetalhes \\ 
			\rowcolor{white} \bAvaliar \\ 
			\rowcolor{white} \bDesistir \\ 
			\rowcolor{white} \bConcluir \\ 
			% -----			
		\end{tabularx}   
		}
	\end{center}

	No entanto, da mesma forma que no módulo de gerenciamento, não vamos mostrar a tabela inteira. Vamos focar em apenas um registro dessa tabela de cada vez. Também iremos ocultar os botões do lado direito. 

	Assim, por exemplo, em um dado momento o estado da solicitação \textbf{PLC73F001-2021} será mostrada assim na Área de Trabalho de \EU:
	
	
	\AREATRABCL{PLC73F001-2021}{Gabinete Delmasso}{CONS}{}{}{}{}{Não}{Distribuição}
	{\ecl{Em Elaboração}}{\EU, \ET, \EC}{-}{\EU}
	
	
	\pagebreak
	
\subsection*{Nomes Fictícios}
	
Como já estamos fazendo, na descrição das simulações de casos de interação vamos utilizar os nomes fictícios de usuários abaixo.

Para facilitar o entendimento, usuários do sistema que terão o papel de \textbf{Supervisores} são nomes que começam com ``S'':

\begin{multicols}{3}
\begin{itemize}
	\item \SU
	\item \SD
	\item \ST
	\item \SQ
	\item \SC
\end{itemize}
\end{multicols}

Consultores Legislativos que farão o papel de \textbf{Elaboradores} terão nomes que começam com ``E'':
\begin{multicols}{3}
\begin{itemize}
	\item \EU
	\item \ED
	\item \ET
	\item \EQ
	\item \EC
\end{itemize}
\end{multicols}

E Consultores Legislativos que farão o papel de \textbf{Revisores} terão nomes que começam com ``R'':
\begin{multicols}{3}
\begin{itemize}
	\item \RU
	\item \RD
	\item \RT
	\item \RQ
	\item \RC
\end{itemize}
\end{multicols}
\end{landscape}

\pagebreak