\chapter{Figuras e gráficos}

\thispagestyle{empty} 

Sugiro que você guarde todas as figuras na pasta ``figs'' para que seu projeto fique mais organizado. A figura \ref{fig:logolatex} mostra como é fácil inserir uma figura com legenda e referência à fonte.

\begin{figure}
	\centering
	\begin{minipage}{0.6\linewidth}
		\centering
		\caption{Logo \LaTeX.}
		\label{fig:logolatex}
		\includegraphics[width=\linewidth]{figs/1280px-LaTeX-logo.png}
		\source{Wikimedia Commons \cite{wikimedia-latex}.}
	\end{minipage}
\end{figure}

Além de figuras, \index{figuras} é possível inserir caixas de texto de diversos tipos, como axiomas, teoremas etc. Elas podem ser configuradas e novos tipos delas podem ser criados no arquivo \begin{verbatim}config/envs.tex\end{verbatim}.

Existem pacotes que permitem criar figuras e gráficos no próprio código \LaTeX. Por exemplo, temos

\begin{itemize}
    \item PGFPlots \url{http://pgfplots.sourceforge.net/}
    \item TikZ \url{http://www.texample.net/tikz/examples/all/}
    \item Metapost \url{http://tex.loria.fr/prod-graph/zoonekynd/metapost/metapost.html}
    \item PSTricks \url{https://tug.org/PSTricks/main.cgi?file=examples}
\end{itemize}

\begin{question}
    Explique como Isaac Newton usaria cada um dos pacotes seguintes, se vivesse no tempo presente:
    \begin{enumerate}[label=(\Alph*)]
        \item Metapost
        \item TikZ
        \item PGFPlots
        \item PSTricks
    \end{enumerate}
\end{question}

\begin{solution}
    \begin{enumerate}[label=(\Alph*)]
        \item Para fazer figuras 3D.
        \item Para fazer diagramas.
        \item Para traçar gráficos.
        \item Para fazer de um tudo.
    \end{enumerate}
\end{solution}

\section{Soluções deste capítulo}

\printsolutions