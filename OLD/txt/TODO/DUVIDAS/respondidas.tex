\section{Dúvidas Respondidas}

\subsection{Respostas para Wagner - Modelagem}

\begin{enumerate}
	
	\item A seta entre Avaliar OS e Histórico parece estar no sentido contrário. É feito um registro a cada avaliação ou é consultado o histórico para verificar se o trabalho já foi demandado?
	
	Sim, é consultado o histórico para verificar se o trabalho já foi demandado. Troquei o sentido da seta.
	
	
	\item O que acontece se o chefe da unidade não aprovar os artefatos?
	
	Se o chefe da unidade não aprovar os artefatos, ele devolve o artefato para o consultor legislativo corrigir.
	
	\item O demandante tem a opção de rejeitar a entrega?
	
	Não sei. Ver com eles.
	
	%\TODO[1]{Ver isso}	
	
	\item Acho que não precisa da seta entre Entregar Artefatos ao Solicitante e Solicitante já que o próximo passo é o envio do e-mail que é, para mim, o registro formal da entrega.
	
	Concordo.	
	
	\item Faltou o evento de início (círculo verde) e um evento de fim (círculo vermelho).
	
	O evento de início é o evento de início especial de recebimento de mensagem.
	
	O evento de final é o evento de final especial de envio de mensagem.	
	
	
	\item Pegar opinião entre modelo com retornos ou sem. 
	
	Vai ser com retornos.
\end{enumerate}

\subsection{Controle de Perfis e Papéis}


\begin{pergunta}[1]{Pergunta}
	Dúvida sobre algo que devo precisar no Sistema da ASSEL: 
	
	- Que informações sobre determinado usuário consigo pegar pelo Active Directory?
	
	- Tem como, pelo login, saber se o usuário é Consultor Legislativo ou Consultor Técnico Legislativo?
\end{pergunta}	

\begin{resposta}[1]{Resposta}
	não use o AD pra regra de negocio de sistema, pelo amor de deus!
\end{resposta}	

\begin{resposta}[1]{Resposta}
	Usar o AD então só para autenticação e depois fazer um controle de papéis interno dentro do sistema. Provavelmente vou precisar de ajuda de vcs quando eu chegar nessa etapa.
	
	Lá na ASSEL, dependendo do Cargo da Pessoa, unidade, lotação, etc, cada um só terá acesso a uma determinada "tela" .
\end{resposta}	

\begin{resposta}[1]{Resposta}
	Acho arriscado amarrar as atribuições em função do cargo. Com as mudanças nas chefias e no quadro de servidores, podem mudar do dia para a noite. No PLe, as funcionalidades são utilizadas para configurar os perfis de acesso e alterações podem ser feitas a qualquer tempo por um administrador do sistema.
\end{resposta}	

\begin{resposta}[1]{Resposta}
	a falar isso: esquece a ideia de "cargo", pensa em "perfil". é atribuição do usuario gestor determinar qual pessoa tem qual perfil em qual unidade. Se ela qusier colcoar um faxineiro com perfil de CL, é problema dele, nao nosso.
	
	e uma mesma pessoa pode ter varios papeis. um chefe pode ser tambem um CL. E tambem assessor de um deputado.
\end{resposta}	
