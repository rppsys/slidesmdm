\chapter{Planejamento Futuro}

\section{Tela de Gerenciamento de Configurações}

Configurações Identificadas:

\begin{itemize}
	\item url do sei;

	\item numero do processo;

	\item numero do documeto de solicitao;

	\item numero do documento de anexo;
\end{itemize}

os numeros atuais estao na pasta de figuras.



\section{Plano}

Esses 4 ficaram na 1 Sprint:

Login/Logout
Gerenciar Usuário
Gerenciar Perfil
Solicitar Ordem de Serviço

Gerenciar Unidades tem que estar na Próxima


Importante:
Gerenciar Ordens de Serviço


Fazer juntos:
Gerenciar Pacotes de Artefatos
Gerenciar Artefatos


Log e Auditoria por último



\section{Telas que serão necessárias}

Ver modelagem-telas que estou criando. 

\begin{itemize}
	\item Duas telas para a ASSEL

	\begin{itemize}
		\item Tela para Analisar (e aprovar) OSs;
		\item Tela para Aprovar OSs Analisadas;
	\end{itemize}
		
	Temos aqui dois Papéis: O Apoio e o Chefe.	

	\begin{itemize}
		\item Chefe = Perfil de Supervisor
		\item Apoio = Perfil de Apoio
	\end{itemize}

	
	
	
	
	
	


\end{itemize}




\section{Tela Principal da ASSEL}

Penso numa tela onde haverá um tabela contendo as OSs que \emph{estão} na ASSEL aguardando providências.

Então temos um ``Estado Interno'' da ASSEL ou ``Classes'' da OS que podem ser:


Classes das OSs dentro da ASSEL:
\begin{itemize}
	\item \textbf{Recebida} ou \textbf{Para Analisar} - OSs recebidas na ASSEL e que aguardam uma providência.
	
	Dentro dessa Classe temos as Subclasses baseadas em \emph{De onde a OS veio?} e para isso, as Subclasses fariam com que as Linhas da Tabela tivessem cores.
	
	\begin{itemize}
		\item Vermelho - OS chegou na ASSEL originando-se de um Solicitante;
		\item Amarelo - OS chegou na ASSEL pela via do ``Retorno'' de uma das Unidades Internas (UDA, Etc...);
		\item Verde - OS chegou na ASSEL pela via da ``Conclusão'' de uma das Unidades Internas (UDA, Etc...);
	\end{itemize} 
	
	
	\item \textbf{Para Cancelar} - Trata de OS já analisada e assumiu o estado ``Para Cancelar''. A linha da tabela ainda tem a cor da subclasse original, contudo, a fonte torna-se negrito. Essa OS precisa receber o consentimento do ``Perfil de Chefe'' da ASSEL para que, de fato, a OS seja cancelada e siga para frente.
	
	\item \textbf{Para Pendência} - Trata de OS já analisada e assumiu o estado ``Para Pendencia''. A linha da tabela ainda tem a cor da subclasse original, contudo, a fonte torna-se negrito. Essa OS precisa receber o consentimento do ``Perfil de Chefe'' da ASSEL para que, de fato, a OS siga seu caminho.
	
	\item \textbf{Para Distribuir} - Idem.
	
	\item \textbf{Para Entregar} - Idem.	
	
	
\end{itemize}


Outra forma de classificar as OSs dentro da ASSEL e, portanto separá-las, são em OSs não analisadas, e OSs analisadas e esperando aprovação do ``Perfil de Chefe''.

Podemos criar duas Telas separadas. 

Na primeira estarão as OSs não analisadas. Na outra, as OSs analisadas e aguardando aprovação para seguir seu caminho.

Na primeira, cujo acesso será do Apoio, contem as OSs não analisadas. De forma que o apoio possa analisar e dar um destino a elas até que a tabela fique sem nada. O trabalho do Apoio é justamente analisar OSs dessa tabela.

Na outra, acessa somente quem tem o ``Perfil de Chefe'' e o objetivo é verificar a análise que foi feita aprovando ou não a análise feita e dando seguimento às OSs.

















\section{Backlog Futuro}

\hypertarget{backlog}{Backlog}

Aqui vou adicionando funcionalidades pedidas pelos usuários que deverão se tornar funcionalidades para o futuro.

\begin{itemize}
	\item \mschecknao \xspace \hyperlink{r1308-1}{Capacidade do Apoio fazer OS em nome de outro Solicitante}
	
	\item \mschecknao \xspace \hyperlink{r160522-1}{Evolução: Campo Deputado e Orgao do SOS já vem pré-preenchido}
	
	
	
	
	
	
	
\end{itemize}












