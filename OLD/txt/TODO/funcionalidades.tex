\chapter{Funcionalidades Futuras}

\begin{table}[!h]
	\begin{center}
		\begin{tabular}{|p{0.4\textwidth}|c|c|}
			\hline
			\rowcolor{corCOULD!50} \multicolumn{3}{|c|}{\Large Legenda \normalsize} \\ \hline \hline
			% CABEÇALHO        
			\rowcolor{lightgray}\textbf{Estado} & \textbf{Sigla} & \textbf{Ícone} \\ \hline
			% CONTEÚDO
			% Código escrito manualmente
			\rowcolor{corCOULD!10} Requisitos a Levantar & RLN  & \msrln \\ \hline
			\rowcolor{corCOULD!20} Requisitos Levantado & RLS & \msrls \\ \hline
			\rowcolor{corCOULD!10} Requisito Novo a Aprovar & RLV & \msrlv \\ \hline
			\rowcolor{corCOULD!20} Funcionalidade Construída & FCS & \msfcs \\ \hline
			\rowcolor{corCOULD!10} Funcionalidade Não Construída & FCN & \msfcn \\ \hline
		\end{tabular}    
		\caption{\label{tab:legenda:funcionalidades} Legendas de Funcionalidades.}
	\end{center}
\end{table}

% Requisito Levantado Sim / Não / Novo
% Funcionalidade Construida Sim / Não


\section{Login/Logout \msfcn - TODOS}

\section{Gerenciar Usuário \msrls - ADM/APOIO DA ASSEL}

	\subsection{Listar}
	
	Sigla
	
	Nome Completo
	
	Divisão: Interna ou Externa
	
	Quantidade de Usuários Participantes
	
		
	\subsection{Incluir}
	\subsection{Editar}
	\subsection{Visualizar}
	\subsection{Excluir}
	\subsection{Pesquisar}
	\subsection{Vincular Perfil}


\section{Gerenciar Perfil \msrls - ADM DA ASSEL}


	\subsection{Listar}
	\subsection{Incluir}
	\subsection{Editar}
	\subsection{Visualizar}
	\subsection{Excluir}
	\subsection{Pesquisar}


\section{Solicitar Ordem de Serviço \msfcn - SOLICITANTE}



	\subsection{Listar}

	\subsection{\textcolor{red}{Incluir pelo Solicitante} \msrls}
	
	Esse é o original.
	
	
	\begin{evolutivo}[1]{\hypertarget{r160522-1}{Evolução: Campo Deputado e Orgao já vem pré-preenchido}}
		\begin{itemize}
			\item Juliana deu essa idéia e vale a pena acatar.
			\item Isso é uma coisa que deverá ser alterada depois em alguma Sprint no futuro.
			\item Coloquei isso no \hyperlink{backlog}{backlog}.
		\end{itemize}
		\tcblower		
		\begin{itemize}
			\item No Solicitar Ordem de Serviço, no formulário de criação de nova OS, existe o Campo ``Deputado/Orgao'' que está livre para o usuário escolher o que quer.
		
			\item 	O ideal é que isso não seja possível pois o ``Deputado/Orgao'' é aquele escolhido no ComboBox de Unidades.
			
			\item 	Ou seja, isso terá de ser alterado lá para que venha já pré-preenchido sem capacidade de ser Editado.
			
			\item Repassei isso para o Robson na reunião de 05 de Setembro 2022.	
		\end{itemize}
	\end{evolutivo}
	

	
	\subsection{\textcolor{red}{Incluir pelo Apoio} \msrlv}
	
	Esse foi pedido pelo Gil, mas ainda não foi feito pela Fábrica.
	
	\subsection{Visualizar}
	\subsection{Cancelar}
	
	
	
	%\toPedro{Melhorar Solicitar OS - Parte de como funciona no SEI} RESOLVIDO
	
	
	\begin{enumerate}
		\item Processo é gerado na Unidade do Sei pertencente ao Solicitante.
	
		\item 	O Arquivo da OS é gerado junto com os anexos e espera-se que ele seja assinado.
		
		\item 	Depois que é assinado, nada mais é feito.
		
		\item 	O sistema manda o processo para a ASSEL.
		
		\item 	Recebe na ASSEL, coloca um Acompanhamento Especial (ou algo assim) e arquiva (ou nao). Esse comportamento tem que ser conversado com o Gilberto.	

		%\toGil{Qual comportamento deve ocorrer?} RESOLVIDO

	\end{enumerate}
	

	
		
	%\TODO{Vai precisar editar OS para incluir anexos caso ocorra pendencia}.	RESOLVIDO
	
	
	
\section{\textcolor{red}{Gerenciar Usuários do Solicitante} \msrlv - SOLICITANTE}

\begin{funcionalidade}{Tela Gerenciar Usuários do Solicitante}
	Vai ter que ter essa tela no futuro.
\end{funcionalidade}

Essa tela deve estar disponível por cada Solicitante para que eles possam gerenciar seus próprios usuários.



	

\section{Gerenciar Unidades \msrln - ADM/APOIO ASSEL}


	\begin{nota}{Novas Eleições}
		Futuramente, essa Gerenciar Unidade pode ter de passar por algumas revisões para comportar a característica de coisas que podem ser necessárias quando ocorrem eleições.
		
		O vínculo dos Gabinetes é com os Deputados.
		
		Talvez seja necessário ATIVAR e DESATIVAR unidades inteiras. Ou seja, um Deputado que saiu terá sua unidade de seu Gabinete desativado e poderá ser reativado depois, caso reeleito no futuro, e assim terá de volta o histórico de Solicitações feitas por ele.
		
		Aqui é importante ressaltar que pode-se perder o vínculo com o SEI.
		
		O ideal é perguntar para o Jefferson como isso será feito no SEI e talvez espelhar o comportamento para o sistema.				
	\end{nota}
		
		
	\begin{nota}{Novas Eleições}		
		Ao conversar com o Gilberto, ele diz que o histórico do solicitante no caso de ser Gabinete é atrelado ao Deputado. 
		
		Assim, o Deputado tem uma pasta. 
		
		Seria bom ver com o Jefferson mesmo assim como isso se dará no SEI.
	\end{nota}

	%\TODO{Perguntar isso ao Jefferson}
	
	
	O negócio vai ser vincular os Gabinetes aos Deputados. Fazer uma tabela separada de Deputados e vincular eles 1 a 1. É uma forma.	


	\subsection{Listar}
	
	SIGLA
	
	NOME
	
	Divisão
	
	Quantidade de Usuários Ativos Vinculados à Unidade
	
	
	\subsection{Incluir}
	
	Elas terão que vir do SEI.
	
	Inclusão Individual: Lista com campo de pesquisa.
	
	Seleciona apenas 1.
	
	Combobox \emph{Origem} com 2 Opções: Externo Interno
	Externo é o Default
	
	Salvar	
	
	Tem que ter vinculação com a Unidade do SEI.
	
	Não pode criar a Unidade Duas Vezes;
	
	
	\subsection{Editar}
	
	Permitindo Ações
	
	
	
	
	\subsection{Visualizar}
	
	Não permitiria Ação
	
	
	Nome
	
	Sigla
	
	Origem: Interno ou Externo
	
	%\TODO[1]{Mudar o nome da divisão}
	% Eu desisti de mudar esse nome. Vou deixar divisão. Virou origem.	
		
	Apresentar a Lista de Usuários Vinculados Áquela Unidade
	
	Filtro
	
	Ativar / Desativar aqui nessa Visualização
	
	Desvincular dessa Unidade () Não
	
	Imprimir um PDF  (Usuários Ativos da Unidade)
	
	
	% \TODO{Fazer padrão de como seria o pdf com isso.}	
	
	\subsection{Excluir}
	
	Não pode excluir se tiverem usuários vinculados à Unidade.
	
	Inicialmente tornar impossível excluir as Unidades Internas que já devem vir pré-cadastradas.
	
	\subsection{Pesquisar}

	Filtro da Lista



	% \TODO{Ver questão da eleição - O que acontece com os Gabinetes?} Já tem TODO para isso.
	
	%\TODO{Questão de ter usuários sem unidade.} Já está nos ajustes.
	
	
	

\section{Gerenciar Ordens de Serviço da Assel \msrln - ASSEL}
\label{tela:ger-os-assel}

	Reuniões: \ref{reuniao:r200921}.
	
	Ver detalhamento em \ref{detalhes:ger-os-assel}.
	

	\subsection{Listar}
		Ver Protótipo que fiz no Google SpreadSheets.
		
		Listar vai ter os campos:
		
		\begin{env-cor}{Campos Associados à Solicitação}{blue}
			\begin{itemize}
				\item Solicitação
				\item Unidade Solicitante
				\item Tipo
				\item Subtipo
				\item Urgente
			\end{itemize}
		\end{env-cor}
		
		\begin{env-cor}{Informações Enviadas Pelo Remetente}{cyan}
			\begin{itemize}
				\item Remetente
				\item Assunto	
			\end{itemize}
		\end{env-cor}
				
		\begin{env-cor}{Somente Aqui na ASSEL}{orange}
			\begin{itemize}
				\item Atribuição
				\item Análise
				\item Destino
				\item Ação	
			\end{itemize}
		\end{env-cor}


		Vou descrever cada coluna:

		\subsubsection{Solicitação} 
			Contém o código único da Solicitação.
		\subsubsection{Unidade Solicitante} 
			Unidade Solicitante
		\subsubsection{Tipo} 
			Tipo
		\subsubsection{Subtipo} 
			Subtipo
		\subsubsection{Urgente} 
			Atributo da Solicitação que ainda não tinha aparecido nas especificações porque somente aqui na ASSEL que ele é definido.
			
			Urgente é um atributo booleano (SIM/NÃO). Default é Não.
			
			Aqui na ASSEL é que o Apoio vai definir se uma Solicitação é Urgente ou não.			
		\subsubsection{Remetente} 
			Última unidade onde tramitou a Solicitação antes de chegar aqui na ASSEL.
		\subsubsection{Assunto} 
			Nesse sistema, todo trâmite de solicitações terá associado uma mensagem (digitada pelo usuário) com um assunto. Os assuntos serão definidos pelo sistema.
			
		\subsubsection{Atribuição} 			
		
			Default: Não Atribuído.
			
			Toda vez que uma Solicitação é enviada para a ASSEL, esse campo é mantido.
			

			
		\subsubsection{Análise}
			
			Default: Não Analisado.			
			
			Toda vez que uma Solicitação é enviada para a ASSEL, esse campo é resetado para Não Analisado.
			
			As opções para Usuários são: Cancelar, Pendencia, Distribuir. Mas o campo ainda pode receber o valor Não Se Aplica.
			
		\subsubsection{Destino}			
		
			Default: Não Especificado.
				
			Toda vez que uma Solicitação é enviada para a ASSEL, esse campo é resetado para Não Especificado.
			
			As opções são: Enviar para UDA, USE...
			
			
		\subsubsection{Ação}			
		
			Ação é um resultado de uma regra aplicada sobre os campos Análise e Destino.
			
		
		
		
		% \TODO{Falar que tem que ter 2 popups: Ementa ou Especificação do Trabalho e Texto do Remetente} RESOLVIDO

		% \TODO{Falar do Memo Texto da Análise} Está previsto.
		
		% \TODO{Especificar as funcionalidades} Está especificado.
	
	\subsection{Editar}

	\begin{itemize}
		\item Atribuir Urgência
		\item Atribuir Servidor  - Forma de Atribuir a Solicitação a um determinado Usuário da Unidade ASSEL.
		\item Atribuir Análise
		\item Atribuir Destino
		\item Executar Ação
	\end{itemize}	


	%\begin{funcionalidade}[1]{Não esquecer disso - Funcionalidades}
	
	% Em algum lugar aqui deve ter que ser possível adicionar novos arquivos na Solicitação.
		
	% Deve ser possível também adicionar ``Anotações'' na Solicitação.
		
	% Deve ser possível vincular uma Solicitação a Outras.
	
	% Talvez o lugar para fazer essas coisas acima seja dentro da tela de Analisar.
	
	% Resolvido
	
	%\end{funcionalidade}

	
	





	\subsection{Visualizar}

	Vai mostrar todos os detalhes da Solicitação.
	
	Aqui deve ser possível inserir novos Anexos (Artefatos) à Solicitação. Artefatos incluídos aqui deverão ser classificado com o tipo ``ASSEL''.
	 

	\subsection{Pesquisar}
	\subsection{Filtrar}
	\subsection{Gerar Relatório}


\section{\red{Gerenciar Ordens de Serviço das Unidades} \msrlv - UNIDADES}

\TODO{A gente só vai se preocupar com isso depois de corrigir os defeitos e deixar tudo funcionando.}








\section{Gerenciar Pacotes de Artefatos \msrln}


	\subsection{Listar}
	\subsection{Incluir}
	\subsection{Excluir}
	\subsection{Pesquisar}
	
	Essa funcionalidade é uma das mais importantes pois acho que aqui devemos fazer a pesquisa textual dentro dos artefatos.



\section{Gerenciar Artefatos \msrln}


	\subsection{Listar}
	\subsection{Incluir}
	\subsection{Editar}
	\subsection{Excluir}
	\subsection{Pesquisar}


\section{Log e Auditoria \msrln}


	\subsection{Pesquisar}


\chapter{Telas Adicionais}


\section{Tela de Pesquisa de Artefatos \msrln}




