\section{Reunião 06 08 21 - Ronie, Gilberto e Pedro}


\begin{itemize}

	
	\item \mschecksim Nomenclatura: Vamos chamar de OS ou de Solicitação? O tempo todo?
	
	\item \mschecksim Login/Logout
	
	
	Duvida para Pedro: Como ocorre o cadastro de usuários, é um por um?
	
	Duvida para Gilberto: Pode ocorrer de um Usuário dentro do Gabinete não poder ter acesso ao ambiente de Solicitação/Acompanhamento de OSs? Espero que não.

	Solução: Fica do jeito que está e haverá a possibilidade dentro do sistema de manualmente vincular usuários ao Cadastro de Unidades Solicitantes.

	O login não muda, porém, haverá uma etapa a mais de verificação caso o usuário que tenta acessar aquela unidade não esteja naquela unidade. Será verificar dentro dos usuários vinculados àquela unidade 

	Sabe de uma coisa?
	
	A melhor solução é criar um procedimento via SEI para que Unidades Solicitantes e Usuários sejam cadastradas no sistema manualmente.
	
	Unidades Solicitantes podem permitir quem eles quiserem.
	
	Um Usuário pode estar cadastrado em mais de uma Unidade Solicitante.
		
	Uma Unidade Solicitante pode ter vários Usuários Cadastrados.
	
	O combobox da Unidade vai trazer apenas as Unidades Solicitantes Cadastradas no Sistema.
	
	O sistema verifica que aquele usuário está vinculado à Unidade que ele quer entrar e autoriza ou não o acesso.
	
	Teremos que fazer um formulário manual no SEI para que uma unidade que quer fazer pedidos à ASSEL cadastre-se e envie uma lista de usuários permitidos.
	
	Isso resolve todos os problemas.
		
	
	\item \mschecksim Estados das OSs
	
	
	
	
	\item \mschecksim Inclusão da funcionalidade <<Editar>> dentro de <<Solicitar Ordem de Serviço>> 


Dentro da funcionalidade <<Solicitar Ordem de Serviço>>

Além de:
Listar, Incluir, Visualizar, Cancelar

Incluir:
Editar

A ação de Editar só será habilitada caso o Estado da OS retorne para o Solicitante com o Estado "Pendência". E aí ele teria de poder reabrir a OS e incluir novas informações para resolver a pendência. Após isso, ele devolve a OS para o Apoio.

Nesse caso, como a OS já está autorizada, nada é preciso ocorrer. 


	\item \mschecksim Dúvida para Pedro: Perguntar aos desenvolvedores se é possível incluir no Processo via SEI uma mensagem do sistema.
	
	\item \mschecksim Modelo de Documento Para Cancelar OS Válida
	
	\item \mschecksim Estados das OSs => Mostrar Tabela
	
	\item \mschecksim Dentro de Gerenciar OS temos que incluir funcionalidades:
	
	Baixar Artefatos
	
	Incluir Anexos
	
	Eu quero outra coisa também. Na hora de incluir os Anexos, quero que cada anexo tenha associado a ele os seguintes atributos: Quem incluiu? Data e Hora da Inclusao? É artefato final?
\end{itemize}
