% Cenário Forrester A
\newcommand{\cenFA}{Cenário \emph{Forrester} A: Distribuição Equivalente} 
\section{\cenFA}
\label{sec-cenfa}

    O \cenFA \xspace despreza inicialmente as classes \emph{MoSCoW} e distribui pesos iguais para as 10 áreas avaliadas. A conta é fácil: $\frac{100\%}{15} = 10\%$.
    
\subsection*{Distribuição dos Pesos}    

    A tabela \ref{tab:cenFA:pesos} apresenta essa distribuição igual de pesos.

% cenFA - Tabela de Pesos
\begin{table}[!h]
    \begin{center}
    \begin{tabular}{|p{0.4\textwidth}|c|c|}
        \hline
            % NOME DA TABELA        
            \rowcolor{cldfB1} \multicolumn{3}{|c|}{\Large \cenFA} \\  
            \rowcolor{cldfB1}
            \multicolumn{3}{|c|}{\large \textbf{Tabela de Pesos}} \\ \hline \hline
            % CABEÇALHO        
            \rowcolor{lightgray}\textbf{Áreas de Capacidade} & \textbf{Classe MoSCoW} & \textbf{Pesos} \\ \hline
            % CONTEÚDO
            % Código gerado pela tabela do Google SpreadSheets Cenário cenFA
            \rowcolor{corMUST!80}Segurança & MUST & 10\% \\ \hline
            \rowcolor{corMUST!80}Big Data & MUST & 10\% \\ \hline
            \rowcolor{corMUST!80}Preparação de Dados & MUST & 10\% \\ \hline
            \rowcolor{corSHOULD!80}Arquitetura & SHOULD & 10\% \\ \hline
            \rowcolor{corSHOULD!80}Interfaces Gráficas & SHOULD & 10\% \\ \hline
            \rowcolor{corSHOULD!80}Mobile & SHOULD & 10\% \\ \hline
            \rowcolor{corSHOULD!80}Opções de Implantação & SHOULD & 10\% \\ \hline
            \rowcolor{corCOULD!50}Criação de Apps Personalizados & COULD & 10\% \\ \hline
            \rowcolor{corCOULD!50}BI Avançado & COULD & 10\% \\ \hline
            \rowcolor{corWOULD!50}Sistemas de Insight & WOULD & 10\% \\ \hline
            % TOTAL
            \rowcolor{lightgray!30} \multicolumn{2}{|r|}{\large Total: \normalsize} & 100\% \\ \hline 
    \end{tabular}    
    \caption{\label{tab:cenFA:pesos} Pesos para \cenFA}
    \end{center}
\end{table}

\subsection*{Resultados}  

    Ao multiplicar os pesos da tabela \ref{tab:cenFA:pesos} aos \emph{scores} da tabela apresentada no Anexo \ref{anexo-tabelafw} encontramos os resultados exibidos na tabela \ref{tab:cenFA:resultados}.

    % cenFA - Tabela de Resultados
    \begin{table}[!h]
        \begin{center}
        \begin{tabular}{|c|cc|}
            \hline
                % NOME DA TABELA        
                \rowcolor{cldfB1} \multicolumn{3}{|c|}{\Large \cenFA} \\  
                \rowcolor{cldfB1}
                \multicolumn{3}{|c|}{\large \textbf{Resultados}} \\ \hline \hline
                % CABEÇALHO        
                \rowcolor{lightgray}\textbf{Fornecedor} & \multicolumn{2}{c|}{\textbf{\emph{Score} [1-5]}} \\ \hline
                % CONTEÚDO
                % Código gerado pela tabela do Google SpreadSheets Cenário GB
                \rowcolor{corP1!80}MicroStrategy & \progressbar{0.88} & 4,4 \\ \hline
                \rowcolor{corP1!80}TIBCO Software & \progressbar{0.88} & 4,4 \\ \hline
                \rowcolor{corPF!20}Sisense & \progressbar{0.72} & 3,6 \\ \hline
                \rowcolor{corPF!20}Tableau Software & \progressbar{0.72} & 3,6 \\ \hline
                \rowcolor{corPF!20}Yellowfin & \progressbar{0.72} & 3,6 \\ \hline
                \rowcolor{corPF!20}Qlik & \progressbar{0.68} & 3,4 \\ \hline
                \rowcolor{corPF!20}SAS & \progressbar{0.68} & 3,4 \\ \hline
                \rowcolor{corPF!20}Birst & \progressbar{0.64} & 3,2 \\ \hline
                \rowcolor{corPF!20}Information Builders & \progressbar{0.64} & 3,2 \\ \hline
                \rowcolor{corPF!20}IBM & \progressbar{0.56} & 2,8 \\ \hline
                \rowcolor{corPF!20}Microsoft & \progressbar{0.56} & 2,8 \\ \hline
                \rowcolor{corPF!20}ThoughtSpot & \progressbar{0.56} & 2,8 \\ \hline
                \rowcolor{corPF!20}OpenText & \progressbar{0.44} & 2,2 \\ \hline
        \end{tabular}    
        \caption{\label{tab:cenFA:resultados} Resultados para \cenFA}
        \end{center}
    \end{table}


\subsection*{Análise dos Resultados} 

    Analisando os resultados obtidos nesta tabela \ref{tab:cenFA:resultados} podemos verificar o seguinte:
    
    \begin{itemize}
        \item A \emph{MicroStrategy} e a \emph{TIBCO Software} empatam em primeiro lugar com $4,4$ pontos;
        \item Todos os demais fornecedores pontuam abaixo de $4,0$ pontos e, portanto, de acordo com o critério de seleção, são eliminados;
    \end{itemize}
    
    Novamente, o \cenFA \xspace apresenta fornecedores que podem ser considerados em um caso genérico onde todas as áreas apresentam mesma importância. Em seguida, de forma parecida com o que foi realizado no capítulo anterior, vamos continuar as análises criando casos de uso com distribuições de pesos mais específicas para as necessidades avaliadas.
    
    