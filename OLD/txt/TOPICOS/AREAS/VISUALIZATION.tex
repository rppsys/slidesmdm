\subsection{Visualização de Dados (\emph{Data Visualization})}
\label{sub-visualization}
\index{Visualização}

Na camada de apresentação encontra-se a capacidade de visualização de dados. No passado, essa capacidade já foi elemento de diferenciação dentre as diversas plataformas existentes, contudo a consolidação do mercado de BI fez com que recursos de Visualização de Dados se tornassem itens obrigatórios \cite{gartner:magicquadrant}.

Conceitua-se essa capacidade como:
\begin{definition}[Capacidade de Visualização de Dados]
Suporte para painéis altamente interativos e exploração de dados por meio da manipulação gráfica de imagens. Incluem-se um conjunto de opções de visualização que vão além dos gráficos de pizza, barras e linhas, como: mapas de calor e árvore, mapas geográficos, gráficos de dispersão e outros recursos visuais especiais.
\end{definition}

% Ronie: Pra mim é Crítico

Naturalmente recursos de visualização de dados devem estar presentes de forma que essa área será qualificada como \MUST.
