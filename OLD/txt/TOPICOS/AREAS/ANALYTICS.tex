\subsection{\emph{Analytics} Avançados (\emph{Advanced Analytics})}
\label{sub-analytics}
\index{Analytics Avançados}

O \emph{analytics} é o motivo pelo qual as plataformas de BI são muitas vezes chamadas de Plataformas de ABI. Vejamos a definição da área de \emph{Analytics} Avançados:

\begin{definition}[Capacidade de Realizar \emph{Analytics} Avançados]
Recursos analíticos avançados que são facilmente acessados pelos usuários, estando contidos na própria plataforma ou utilizáveis por meio da importação e integração de modelos externos.
\end{definition}

Conforme já foi descrito na seção \ref{sub-niveis}, a análise de dados é dividida em três modalidades: análise descritiva, análise preditiva e análise prescritiva. Enquanto o \emph{Business Intelligence} (BI) ocupa-se de fazer análise descritiva, ou seja, ajudar a entender o que está acontecendo, o \emph{Analytics} preocupa-se com as análises preditiva e prescritiva, isto é, prever o que acontecerá e antecipar ações. Ainda foi visto que o sucesso do \emph{analytics} depende, em grande parte, de experiência prévia da instituição no mundo de BI.

Contudo, a realidade atual da CLDF é outra. Ela está começando sua jornada no mundo da análise de dados e, por enquanto, fornecer uma visão descritiva das coisas, isto é, fornecer uma percepção do que aconteceu e o que está acontecendo já irá certamente revolucionar a forma como ela fiscaliza. 

O foco necessário neste primeiro momento é em BI e não em \emph{analytics}. Isso porque atualmente, suas unidades organizacionais ainda não possuem a \emph{expertise} de BI necessária para justificar investir em capacidades de \emph{analytics}, e muito menos, em capacidades de \emph{analytics} avançados!

 Em contrapartida, após vencer as etapas necessárias para criar um ambiente de BI satisfatório, o enfoque em \emph{analytics} será o caminho natural.

% 3 - COULD

    Deste modo, em virtude dessas considerações, entende-se que a Área de \emph{Analytics} Avançados deve ser classificada como \COULD.