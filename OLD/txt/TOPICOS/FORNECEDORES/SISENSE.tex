\section{\emph{Sisense}}
\label{sub-sisense}
\index{Sisense}

\begin{wrapfigure}[4]{r}{0.4\textwidth}     
    \centering
    \includegraphics[width=0.35\textwidth,height=2cm]{fig/f/sisense.png}
\end{wrapfigure}

O produto avaliado é o Sisense 8.0.1.

\subsection*{Destaques}

O Sisense possui forte avaliação na Incorporação de Análises e suporta um \emph{workflow} personalizado para uma aplicação web com arquitetura microsserviços. Além disso, o Sisense Quest permite que usuários apliquem modelos de \emph{analytics} avançados a widgets. \cite{gartner:criticalcapabilities} Ademais, o Sisense foi redesenhado para funcionar em ambiente Docker/Kubernetes \cite{gartner:criticalcapabilities} (o que mostra a sua capacidade de atendimento ao modelo \hyperref[caas]{CaaS}).

\subsection*{Pontos Fortes}

O \relGMQ \xspace e o \relGCC \xspace trazem boas avaliações do Sisense nas áreas de capacidade crítica de ação de dados, gerenciamento e segurança.

\subsection*{Pontos Fracos}

O uso de linguagem de programação é frequentemente necessário para o uso de modelos avançados. O uso de Consulta em Linguagem Natural só possui suporte na língua inglesa. 

\subsection*{Avaliação}

O Sisense ficou entre os 3 primeiros fornecedores em 1 dos 6 cenários de caso de uso elaborados com base no levantamento de necessidades. Seu resultado mostra uma ferramenta que é menos recomendável do que outras avaliadas.

% \cite{gartner:magicquadrant}
% \cite{gartner:criticalcapabilities}