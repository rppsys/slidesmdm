\section{\emph{Birst}}
\label{sub-birst}
\index{Birst}

\begin{wrapfigure}[3]{r}{0.25\textwidth}     
    \centering
    \includegraphics[width=0.2\textwidth]{fig/f/birst.png}
\end{wrapfigure}

O produto avaliado é o Birst Fall 19 (7.2).

\subsection*{Destaques}

O Birst destaca-se pelas capacidades de gerenciamento que suportam modelagem de metadados centralizada, assim como modelos gerados e promovidos pelos usuários. Os usuários de negócio podem importar dados e misturá-los aos existentes, e os modelos e integrações são criados automaticamente, sem necessidade de mão-de-obra especializada \cite{gartner:criticalcapabilities}.

\subsection*{Pontos Fortes}

O \relGMQ \xspace e o \relGCC \xspace trazem boas avaliações do Birst nas áreas de capacidade de incorporação de análises, complexidade de modelos e geração de relatórios \cite{gartner:criticalcapabilities}.

\subsection*{Pontos Fracos}

A ferramenta possui uma performance problemática, de acordo com a maioria dos clientes, além de suporte deficiente e dificuldade para uso \emph{self-service} \cite{gartner:magicquadrant}.

\subsection*{Avaliação}

O Birst ficou entre os 3 primeiros fornecedores em 1 dos 6 cenários de caso de uso elaborados com base no levantamento de necessidades. Seu resultado mostra uma ferramenta que é menos recomendável do que outras avaliadas.


% \cite{gartner:magicquadrant}
% \cite{gartner:criticalcapabilities}