\section{\emph{QLik}}
\label{sub-qlik}
\index{Qlik}

\begin{wrapfigure}[3]{r}{0.3\textwidth}     
    \centering
    \includegraphics[width=0.2\textwidth,height=1cm]{fig/f/qlik.png}
\end{wrapfigure}

O produto avaliado é o Qlik Sense Enterprise.

\subsection*{Destaques}

O QLik Sense destaca-se pelo motor associativo Qlik, que permite a usuários de diversos níveis de conhecimento combinarem dados e explorarem informações sem as limitações de ferramentas baseadas em linguagens de programação \cite{gartner:magicquadrant}. Esse motor cria uma interface amigável ao usuário, que lhe permite fazer as combinações existentes no catálogo e montar os seus próprios painéis. Desse recurso, pode-se extrair um valor de economia em recursos humanos, dada a desnecessidade de mão de obra especializada nessa fase do processo. Ainda, A Qlik possui arquitetura baseada em microserviços \cite{gartner:magicquadrant}, o que proporciona uma melhor integração com um ambiente em formato cloud, que pode ser privada ou pública. Ademais, o Qlik conta com \emph{insights} associativos como uma capacidade de Analytics Aumentada, que utiliza o seu motor cognitivo para desobrir \emph{insights} ocultos.

\subsection*{Pontos Fortes}

Os pontos fortes da ferramenta podem ser vistos como flexibilidade de implantação, expansividade das capacidades da plataforma e literalidade dos dados\cite{gartner:magicquadrant}. O \relGMQ \xspace e o \relGCC \xspace trazem boas avaliações da Qlik nas áreas de capacidade crítica de criação de apps personalizados, \emph{insights} automatizados e nuvem, com a flexibilidade de uso de nuvem pública, privada e híbrida. Para atender melhor às capacidades de gerenciamento, consulta em linguagem natural e geração de relatórios, são oferecidos complementos à ferramenta. 

\subsection*{Pontos Fracos}

A ferramenta não possui geração de linguagem natural, bem como o seu baixo acoplamento faz com que haja necessidade de bom planejamento para se decidir os complementos necessários para aquisição.

\subsection*{Avaliação}

A Qlik ficou entre os 3 primeiros fornecedores em 2 dos 6 cenários de caso de uso realizados com base no levantamento de necessidades. Seu resultado sofreu influência negativa em razão da necessidade de contratação de complementos para melhor pontuação nas capacidades críticas, que de acordo com \cite{gartner:criticalcapabilities}, não foram levados em consideração em razão do tempo de avaliação dos pesquisadores. Considerando tudo, essa plataforma aparenta ser uma boa alternativa dentre as identificadas neste estudo. 