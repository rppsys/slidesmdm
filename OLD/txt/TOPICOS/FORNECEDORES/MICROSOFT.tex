\section{\emph{Microsoft}}
\label{sub-microsoft}
\index{Microsoft}

\begin{wrapfigure}[3]{r}{0.4\textwidth}     
    \centering
    \includegraphics[width=0.35\textwidth,height=2cm]{fig/f/microsoft.png}
\end{wrapfigure}

O produto avaliado é o Power BI.

\subsection*{Destaques}

O Power BI destaca-se por fazer parte de um conjunto integrado de componentes em um único serviço que funciona no provedor de nuvem Microsoft Azure. A solução do Power BI adequada para um uso corporativo é o Power BI Premium ou os serviços contratados na forma de \hyperref[saas]{SaaS} na nuvem Microsoft Azure.

\subsection*{Pontos Fortes}

O \relGMQ \xspace e o \relGCC \xspace trazem boas avaliações do Power BI em todas as áreas de capacidade crítica, sendo considerado excelente para \emph{Analytics} Avançados, Complexidade de Modelos, Incorporação de Análises, \emph{Insights} Automatizados e Conectividade de Fontes de Dados. A sua única capacidade que recebeu nota menor do que 3,5 de 5 foi a Geração de Linguagem Natural \cite{gartner:criticalcapabilities}.

\subsection*{Pontos Fracos}

A ferramenta entrega o seu valor máximo somente quando fornecida na modalidade \hyperref[saas]{SaaS} do provedor Microsoft Azure de nuvem, não sendo oferecida na modalidade \hyperref[iaas]{IaaS}, de acordo com \cite{gartner:criticalcapabilities}. A versão de execução local (\emph{on premises}) possui restrição severa de funcionalidades, tais como a habilidade de compartilhamento de painéis, \emph{insights} automatizados e consulta em linguagem natural, de acordo com \cite{gartner:criticalcapabilities}.

Essa restrição de aproveitamento somente no provedor próprio de nuvem traz um risco severo de \emph{lock in} da solução, tendo em vista a menor capacidade de desacoplamento no modelo nuvem, bem como a restrição severa de funcionalidades quando executada no modelo de instalação local.

Ademais, deve-se levar em consideração que, diferentemente do Power BI Desktop, mais voltado ao uso doméstico e pontual, o Power BI Premium, que é voltado ao uso corporativo, possui valores consideravelmente mais altos de licenciamento.

\subsection*{Avaliação}

A Microsoft ficou entre os 3 primeiros fornecedores em 3 dos 6 cenários de caso de uso elaborados com base no levantamento de necessidades. Seu resultado apresenta uma consistência de atendimento balanceado das áreas de capacidade crítica. Ocorre que o Power BI sofre severas limitações quando instalado em ambiente \emph{on premises} local. Baseado nisso, essa plataforma aparenta ser uma  alternativa adequada dentre as identificadas neste estudo somente no caso de uma contratação feita no modelo \hyperref[saas]{SaaS} em nuvem pública pelo provedor Microsoft Azure (alternativa otimizada de aproveitamento dos recursos) ou com contratação do Power BI Premium, que já se mostra limitada. Entretanto, alerta-se sobre o risco de aprisionamento de fornecedor (\emph{lock in}) que pode ser trazido pelas peculiaridades da ferramenta.