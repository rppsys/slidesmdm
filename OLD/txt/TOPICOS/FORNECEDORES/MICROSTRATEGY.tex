\section{\emph{MicroStrategy}}
\label{sub-microstrategy}
\index{MicroStrategy}

\begin{wrapfigure}[3]{r}{0.4\textwidth}     
    \centering
    \includegraphics[width=0.35\textwidth,height=1cm]{fig/f/microstrategy.png}
\end{wrapfigure}

O produto avaliado é o MicroStrategy 2020.

\subsection*{Destaques}

O MicroStrategy 2020 destaca-se pelo recurso denominado \emph{HyperIntelligence}. Segundo \cite{microstrategy:hyper}, \emph{HyperIntelligence} são \emph{cards} que ``colocam conhecimento especializado nas mãos dos usuários, onde eles já trabalham. Seja no Outlook, Salesforce.com ou até mesmo em seus sites favoritos, os \emph{cards} permitem que você simplesmente passe o cursor do mouse sobre palavras em destaque para exibir informações contextualizadas. Cada \emph{card} apresenta, de modo elegante, indicadores-chave previamente definidos – obtidos de maneira segura a partir do MicroStrategy.''

\subsection*{Pontos Fortes}

O \relGMQ \xspace e o \relGCC \xspace trazem boas avaliações da MicroStategy nas áreas de capacidade crítica de: segurança, geração de relatórios, gerenciamento de usuários, incorporação de análises, conectividade de fontes de dados e visualização. Esses relatórios destacam, também, a capacidade de prover suporte a modelos complexos de dados onde as organizações podem mapear várias instâncias de fontes de dados diferentes (incluindo bancos de dados relacionais, \emph{big data}, serviços da web, sistemas em nuvem, etc.) para um único projeto de BI da MicroStrategy. 

\subsection*{Pontos Fracos}

Os relatórios apontam que a capacidade de prover \emph{insights} automatizados apresentou a menor pontuação dentre as áreas de capacidade avaliadas pois os usuários não conseguem utilizar esse recurso com apenas alguns cliques. É necessário utilizar interfaces gráficas complicadas ou usar conectores específicos para integrar o MicroStrategy com ferramentas como o \emph{RStudio} ou \emph{Jupyter Notebooks}.

Apesar de boas pontuações em áreas críticas de capacidade, o \relGCC \xspace adverte que o custo pode ser uma barreira. De acordo com o relatório, metade dos clientes de referência da MicroStrategy identificaram o custo de seu software como uma barreira para uma implantação mais ampla, em comparação com a média do mercado de cerca de um quinto dos fornecedores.

\subsection*{Avaliação}

O MicroStrategy lidera os casos de uso utilizando o \emph{Gartner} ficando em primeiro lugar em todos eles e aparece entres os três primeiros fornecedores em 2 dos 3 cenários de caso de uso usando \emph{Forrester}. Portanto, do ponto de vista qualitativo, isto é, sem levar em conta a questão financeira do custo do software, essa plataforma aparenta ser uma boa alternativa dentre as identificadas neste estudo. 
