\chapter{Levantamento Bibliográfico}
\label{cap-literatura}

A primeira atividade de um estudo que, sem dúvidas, influenciará nos seus resultados é a realização de amplo levantamento bibliográfico sobre os assuntos que serão investigados. Esse capítulo pretende introduzir as principais fontes de informação utilizadas para embasar e fundamentar este estudo.

\section{Produções CLDF}

    Vamos iniciar apresentando as produções da Câmara Legislativa do Distrito Federal.

\subsection{A \emph{proposta}}

    Conforme já foi alertado na nota de rodapé da seção \ref{sec-metafora}, para simplificar, estamos nos referindo à ``Proposta para Modernização da Função Institucional de Fiscalização no âmbito da Comissão de Defesa dos Direitos Humanos, Cidadania, Ética e Decoro Parlamentar, com Aplicação Computacional de Ciência de Dados \& BI'' \cite{propostaCDDHCEDP} com o termo em itálico: ``\emph{proposta}''. Ela foi fruto do segundo e mais recente projeto concluído (antes deste estudo) na CMI sobre o tema de Ciências de Dados e \emph{Business Intelligence} e 
    é, portanto, a principal referência para o desenvolvimento deste trabalho.

\subsection{Oficina ASI-LABHINOVA}

    A Oficina ASI-Labhinova sobre Ciência de Dados Aplicada ao Poder Legislativo: ``Análise Exploratória de Dados Abertos utilizando Ferramentas Livres'' foi o primeiro projeto concluído na CMI sobre o tema \cite{asi:oficina}. Esse primeiro projeto proporcionou uma compreensão introdutória, mas abrangente, sobre ciência de dados, de forma a estimular a aplicação de saberes, habilidades e atitudes para análise de dados temáticos sobre o Distrito Federal.


\subsection{Levantamento de Necessidades com a CAS}

    Outro projeto relevante que está sendo desenvolvido em paralelo a este estudo é um projeto semelhante à \emph{proposta}, mas que está sendo realizado no âmbito da Comissão de Assuntos Sociais (CAS). Seu nome oficial é ``Proposta para Modernização da Função Institucional de Fiscalização no âmbito da Comissão de Assuntos Sociais com Aplicação Computacional de Ciência de Dados e BI''. Apesar de que esse projeto ainda não produziu resultados oficiais e conta, atualmente, somente com um termo de abertura \cite{propostaCAS} o mesmo contribuiu com o desenvolvimento deste estudo técnico.

\subsection{Plano Diretor de Tecnologia de Informação}
    Não podemos deixar de mencionar o Plano Diretor de Tecnologia de Informação (PDTI). Recentemente, enquanto este trabalho estava sendo escrito, o ``Plano Diretor de Tecnologia da Informação da Câmara Legislativa do Distrito Federal Atualização 2020'' foi aprovado pelo Ato da Mesa Diretora Nº 102, de 2020 e publicado no Diário da Câmara Legislativa Nº 231 \cite{pdti2020}.
 

\subsection{Estratégia de Sistema de Informação}
    Finalmente a Estratégia de Sistema de Informação (ESI) \cite{asiESI} estabelece as prioridades de aplicações de computação a serem providenciadas para a CLDF. Estas prioridades advêm da seguinte visão, projetada para o Sistema de Informação da CLDF: 
    
    \begin{enumerate}[label=\Alph*)]
        \item População plenamente informada, por meio de aplicação de computação, sobre assuntos institucionais de seu interesse;
        
        \item População plenamente informada, por meio de aplicação de computação, sobre momentos oportunos para participação em assuntos institucionais de seu interesse;
        
        \item Participação popular facilitada e estimulada a partir de aplicações de computação;
        
        \item Parlamentares plenamente munidos de informações sobre os temas do Distrito Federal, por meio de aplicação de computação;
        
        \item Parlamentares amparados por conhecimento sistematizado sobre as políticas públicas, por meio de aplicação de computação;
        
        \item Parlamentares plenamente informados, por meio de aplicação de computação, sobre assuntos institucionais em que atuam;
        
        \item Funções finalísticas - representação, legiferação e fiscalização - desempenhadas em meio digital, considerando os princípios da sustentabilidade na Administração Pública;
        
        \item Relacionamento entre Câmara e população acontecendo por meio de aplicações de computação.
    \end{enumerate}


\section{Livro de Referência}

A 4ª edição do livro intitulado ``BUSINESS INTELLIGENCE e ANÁLISE DE DADOS para gestão do negócio'' de autoria de \emph{Ramesh Sharda}, \emph{Dursun Delen} e \emph{Efraim Turban} \cite{turban2019} auxiliou a esclarecer boa parte dos conceitos que serão discutidos no próximo capítulo ``\autoref{cap-referencial} -- \nameref{cap-referencial}''. Certamente, esse livro é leitura obrigatória para qualquer um que deseje se aprofundar nos assuntos de análise de dados e \emph{business intelligence}.

\section{Relatórios Técnicos}
\label{sec-relatorios}

Durante a atividade de levantamento bibliográfico também identificamos organizações que fornecem serviços e prestam consultoria para os fornecedores e usuários da indústria de análise de dados. Dentre os serviços oferecidos estão: \textbf{análises comparativas entre fornecedores}, cobertura de novos desenvolvimentos, avaliação de tecnologias específicas, desenvolvimento de artigos técnicos e assim por diante \cite{turban2019}. Dessa feita, este estudo se baseia nos relatórios técnicos listados abaixo:

\begin{env-destaque}{Relatórios Técnicos}
\begin{itemize}
    \item \RelatorioGMQ \xspace \cite{gartner:magicquadrant};
    
    \item \RelatorioGCC \xspace \cite{gartner:criticalcapabilities};
    
    \item \RelatorioFCM \xspace \cite{forrester:clientmanaged};
    
    \item \RelatorioFVM \xspace \cite{forrester:vendormanaged};
\end{itemize}
\end{env-destaque}


Os dois primeiros são produzidos pelo Grupo \emph{Gartner} e os dois últimos pelo \emph{Forrester}. A seguir faremos uma breve descrição dessas organizações e respectivos relatórios utilizados neste estudo.


\subsection{Grupo \emph{Gartner}}

Fundada em 1979, O \emph{Gartner} é uma empresa americana de pesquisa e consultoria em TI. A empresa afirma preparar os executivos para tomar as decisões certas e ficar à frente das mudanças \cite{gartner:about}. 

O Grupo \emph{Gartner} é responsável pela publicação de dois tipos de relatórios de pesquisa de mercado conhecidos como ``\emph{Gartner Magic Quadrant}'' e ``\emph{Critical Capabilities}''. Podemos traduzir esses termos como: ``Quadrantes Mágicos'' e ``Capacidades Críticas'' do \emph{Gartner}. 
O primeiro consiste em um relatório cuja metodologia de pesquisa utiliza um conjunto uniforme de critérios de avaliação para fornecer um gráfico onde os fornecedores de uma determinada tecnologia são classificados em quatro tipos: Líderes, Visionários, Desafiantes e Competidores de Nicho. Trata-se de uma forma visual de  mostrar o posicionamento competitivo de cada 
provedor facilitando verificar se eles estão cumprindo com suas visões declaradas e, também, investigar como é o desempenho de um provedor em relação aos demais dentro da visão de mercado do \emph{Gartner}.
O segundo relatório apresenta notas de pesquisa complementar que fornecem uma visão aprofundada sobre a capacidade e adequação dos produtos e serviços de TI dos provedores com base em casos de uso específicos ou personalizados \cite{gartner:magicquadrantsecriticalcapabilities}. 

Esses relatórios são produzidos através de métodos proprietários de análise de dados qualitativos para demonstrar tendências de mercado. Eles ajudam a descobrir e comparar os pontos fortes, pontos fracos, visão e desempenho de diferentes provedores de uma tecnologia específica de mercado. Assim, estaremos interessados no relatório do tipo ``Magic Quadrant'' para o mercado tecnológico específico de ``\emph{Business Intelligence and Analytics Platforms}''. O relatório do tipo ``Critical Capabilities'' é também fundamental para analisar os diferentes \emph{players} pois permite comparar o desempenho de cada um em diferentes cenários. Ainda é possível produzir casos de uso personalizados e é justamente esse o nosso interesse.

\subsection{\emph{Forrester}}

Fundada em 1983, o \emph{Forrester} é uma empresa americana de pesquisa de mercado e TI. Ela oferece pesquisa, \emph{analytics}, programas executivos, certificações, consultoria e eventos.

O \emph{Forrester} possui uma metodologia chamada \emph{The Forrester Wave}, que consiste de um guia para compradores que considerem suas opções de compra no mercado de tecnologia. Esse guia utiliza de metodologia disponível publicamente, que é aplicada em todos os fornecedores participantes \cite{forrester:waveMethodology}. O processo \emph{Forrester Wave} consiste dos seguintes passos \cite{forrester:waveMethodology}:

\begin{enumerate}
    \item Pesquisa sobre a categoria e definição de escopo;
    \item Estabelecimento dos critérios de inclusão e avaliação;
    \item Realização de conferência com os respectivos fornecedores.
\end{enumerate}

Dessa forma, percebe-se que se trata de um processo impessoal, que busca utilizar ao máximo possível de critérios objetivos para padronizar uma análise entre diferentes fornecedores dentro de uma determinada categoria.
