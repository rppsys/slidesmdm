\part{Conclusões}
\label{parte-conclusoes}

\chapter{Conclusões}
\label{cap-conclusoes}

O presente estudo técnico propiciou uma compreensão abrangente sobre os temas de análise de dados e \emph{business intelligence}. Essa compreensão é construída nos Capítulos \ref{cap-literatura} e \ref{cap-referencial} da ``\autoref{parte-revisao} -- \nameref{parte-revisao}'' e é fortalecida pelos demais capítulos que compõe o estudo. Podemos afirmar, portanto, que a leitura deste estudo concede um conjunto valioso de conhecimentos sobre os assuntos tratados. 

A análise comparativa de plataformas de BI é realizada por meio da elaboração de casos de uso personalizados a partir de relatórios técnicos de referência nos Capítulos \ref{cap-casos-gartner} e \ref{cap-casos-forrester} da ``\autoref{parte-estudosdecaso} -- \nameref{parte-estudosdecaso}''. No entanto, para que essa análise comparativa pudesse ocorrer foi preciso, primeiro, estudar e estabelecer o panorama de necessidades da instituição identificando e avaliando requisitos a serem atendidos por uma futura plataforma de BI. Isto foi realizado no Capítulo \ref{cap-necessidades} da ``\autoref{parte-necessidades} -- \nameref{parte-necessidades}'' em conjunto com os elementos apresentados no ``\autoref{cap-descricao} -- \nameref{cap-descricao}''. A análise comparativa é concluída no Capítulo \ref{cap-fornecedores} da ``\autoref{parte-plataformas} -- \nameref{parte-plataformas}'' quando as características e peculiaridades dos produtos de cada fornecedor de destaque são examinados em maior detalhe.

A ``\autoref{parte-proposta} -- \nameref{parte-proposta}'' atinge os demais objetivos específicos do termo de abertura. Os Capítulos \ref{cap-acoes} e \ref{cap-esforcos} estabeleceram um \emph{roadmap} de implantação de BI com diretrizes de uma arquitetura corporativa a ser utilizada na CLDF, de modo a padronizar as implantações e otimizar as atividades de implantação, operação, expansão e evolução desta tecnologia. Nesse sentido, ações foram recomendadas e uma ordem de priorização de esforços foi definida para que o ambiente de análise de dados seja construído de forma segura, eficaz e evitando alguns riscos potenciais. Em seguida, o Capítulo \ref{cap-proposta} finaliza essa parte especificando as próximas atividades a serem realizadas para produzir os primeiros resultados concretos e começar a gerar o valor esperado para as Comissões.

Conclui-se que o estudo cumpre os objetivos do Projeto ``Estudo Técnico sobre Plataformas de \emph{Business Intelligence}'' instituído pelo Ato do Vice-Presidente Nº 56, de 2020.
