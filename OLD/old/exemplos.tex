%--------------------------------------------------------------
%  Capítulos e seções
%--------------------------------------------------------------
%
% Um capítulo se inicia com o seguinte comando "\capitulo{Título}."
%
% Uma seção se inicia com o seguinte comando "\secao{Título da Seção}". Caso queira adicionar a seção ao índice acrescente o comando "\index{Título da Seção}. 
%
% O mesmo princípio é utilizado para os subníveis da seção. São eles: subseção e subsubseção, dados pelos comandos "\subsecao{Título}" e "\subsubsecao{Título}". Estes podem ser seguidos ou não do comando "\index{Título}"
%
%--------------------------------------------------------------



% Título do capítulo
\part{Exemplos}
\capitulo{Exemplos}


  Aqui fica o texto introdutório do capítulo.
  Este é o texto de um capítulo.

    % título da seção
    \secao{titulo da seção}\index{Palavra-chave}
 
        Escreva o texto da seção aqui. 
    
    
        % título da subseção
        \subsecao{titulo da subseção}
                
                Texto da seção.
        
                % título da subsubseção
                \subsubsecao{titulo da subseção}\index{palavra-chave(seção)!palavra-chave} 

                    Texto da subsubseção.

%--------------------------------------------------------------
%   Teoremas
%--------------------------------------------------------------


    \secao{Teoremas}\index{Teoremas}

        Esta seção traz exemplos de teoremas.

        \subsecao{Várias Equações}\index{Teoremas!Várias Equações}

            Exemplo de Teorema com várias equações.

            \begin{theorem}[Nome do teorema]
                In $E=\mathbb{R}^n$ all norms are equivalent. It has the properties:
                \begin{align}
                    & \big| ||\mathbf{x}|| - ||\mathbf{y}|| \big|\leq || \mathbf{x}- \mathbf{y}||\\
                    &  ||\sum_{i=1}^n\mathbf{x}_i||\leq \sum_{i=1}^n||\mathbf{x}_i||\quad\text{where $n$ is a finite integer}
                \end{align}
            \end{theorem}

        \subsecao{Linha Única}\index{Teoremas!Linha Única}
            Teorema em Linha Única.
            \begin{theorem}
                A set $\mathcal{D}(G)$ in dense in $L^2(G)$, $|\cdot|_0$. 
            \end{theorem}

%--------------------------------------------------------------


%--------------------------------------------------------------
%   Definições
%--------------------------------------------------------------
\secao{Definições}\index{Definições}

    Exemplo de Definição.
    \begin{definition}[Nova Definição]
        Given a vector space $E$, a norm on $E$ is an application, denoted $||\cdot||$, $E$ in $\mathbb{R}^+=[0,+\infty[$ such that:
        \begin{align}
            & ||\mathbf{x}||=0\ \Rightarrow\ \mathbf{x}=\mathbf{0}\\
            & ||\lambda \mathbf{x}||=|\lambda|\cdot ||\mathbf{x}||\\
            & ||\mathbf{x}+\mathbf{y}||\leq ||\mathbf{x}||+||\mathbf{y}||
        \end{align}
    \end{definition}

%--------------------------------------------------------------

%--------------------------------------------------------------
%   Notações
%--------------------------------------------------------------
\secao{Notações}\index{Notações}

    \begin{notation}
        Given an open subset $G$ of $\mathbb{R}^n$, the set of functions $\varphi$ are:
        \begin{enumerate}
            \item Bounded support $G$;
            \item Infinitely differentiable;
        \end{enumerate}
        a vector space is denoted by $\mathcal{D}(G)$. 
    \end{notation}

%--------------------------------------------------------------



%--------------------------------------------------------------
%  Indicação de ícones
%--------------------------------------------------------------
    \secao{Indicação de Ícones}\index{Indicação de Ícones}

        \mais{ Oferece novas informações que enriquecem o assunto ou “curiosidades” e notícias recentes relacionadas ao tema estudado. Não indicar simplesmente livros, filmes, links, etc. Oriente os estudantes sobre o que vão encontrar no material indicado.
        }

        \midia{ Sempre que se desejar que os estudantes desenvolvam atividades empregando diferentes mídias: vídeos, filmes, jornais, ambiente AVA, sites e outras.
        }

        \glossario{ Indica a definição de um termo, palavra ou expressão utilizada no texto.
        }

        \atividade{ Apresenta atividades em diferentes níveis de aprendizagem para que o estudante possa realizá-las e conferir o seu domínio do tema estudado.
        }

        \atencao{  Indica pontos de maior relevância no texto.
       
        o 
        
        quadro 
       
        aumenta automaticamente
      
        kkk
        }

%--------------------------------------------------------------


%--------------------------------------------------------------
%   Proposições
%--------------------------------------------------------------
    \secao{Proposições}\index{Proposições}

    Estes são exemplos de proposição.
    
        \subsecao{Várias Equações}\index{Proposições!Várias Equações}
            Exemplo com várias equações.
            \begin{proposition}[Nome da Proposição]
                It has the properties:
                \begin{align}
                    & \big| ||\mathbf{x}|| - ||\mathbf{y}|| \big|\leq || \mathbf{x}- \mathbf{y}||\\
                    &  ||\sum_{i=1}^n\mathbf{x}_i||\leq \sum_{i=1}^n||\mathbf{x}_i||\quad\text{where $n$ is a finite integer}
                \end{align}
            \end{proposition}

        \subsecao{Linha Única}\index{Proposições!Linha Única}
            Exemplo em Linha única.
            \begin{proposition} 
                Let $f,g\in L^2(G)$; if $\forall \varphi\in\mathcal{D}(G)$, $(f,\varphi)_0=(g,\varphi)_0$ then $f = g$. 
            \end{proposition}

%--------------------------------------------------------------

%--------------------------------------------------------------
%   Equações 
%--------------------------------------------------------------
    \secao{Equações}\index{Equações}
        Este é um exemplo de uma equação.
        \begin{equation}
            \cos^{2}{x} + \sin^{2}{x} = 1, \quad \forall x \in \mathbb{R}
            \label{equ:1}
        \end{equation}
        
        A equação \ref{equ:1} está referenciada.
        
        Caso não queira uma numeração para a equação:
        \begin{equation*}
            \cos^{2}{x} + \sin^{2}{x} = 1, \quad \forall x \in \mathbb{R}
        \end{equation*}       
        %Note a existência de funções como "\cos{}" e "sin{}" e não, simplismente escrito "cos()" e "sin()".
        
        Um conjunto de equações:
        
        \begin{equation}
            f(x) = 
            \begin{cases}
                2x+1, \quad  x \leq 0 \\
               x^2 +\frac{3}{2}x +\log{x},  \quad  x > 0
            \end{cases}
            \label{equ:2}
        \end{equation}   
        %Note como é criada uma fração com o comando "\frac{}{}"
        
  
       \subsecao{Matrizes}\index{Equações!Matriz}
        
         Exemplo padrão % o símbolo "&" é o separador de colunas e o "\\" indica final de linha
        
        $ %Entrando no Modo matémático entre "$$"
        \begin{bmatrix}
            1 & 2 & 3\\
            a & b & c
        \end{bmatrix}
        $
            Exemplo de equação matricial.
            \begin{equation}
                \begin{bmatrix}
                    Y_{1,1}\\ 
                    \vdots \\   
                    \vdots \\ 
                    Y_{m,1}
                \end{bmatrix}
                =\begin{bmatrix}
                    A_{1,1} & \ldots & \ldots & A_{1,n}\\ 
                    \vdots & \ddots &  & \\   
                    \vdots &  & \ddots  & \\ 
                    A_{m,1}&  &  & A_{m,n}
                \end{bmatrix}
                .\begin{bmatrix}
                    \mathbf{X_{1,1}} \\  %Negrito 
                    \mathbf{\vdots} \\   
                    \mathbf{\vdots} \\ 
                    \mathbf{X_{m,1}}
                \end{bmatrix}
                +\begin{bmatrix}
                    B_{1,1} \\ 
                    \vdots \\   
                    \vdots \\ 
                    B_{m,1}
                \end{bmatrix}
            \end{equation}


        Exemplos de outros modelos de matrizes 
        \url{https://pt.overleaf.com/learn/latex/Matrices}
      

      

%--------------------------------------------------------------
%   Exemplos 
%--------------------------------------------------------------
    \secao{Exemplos}\index{Exemplos}

        Este é um exemplo de exemplos.
        
        \subsecao{Equações e Texto}\index{Exemplos!Equações e Texto}
            \begin{example}
                Let $G=\{x\in\mathbb{R}^2:|x|<3\}$ and denoted by: $x^0=(1,1)$; consider the function:
                \begin{equation}
                    f(x)=\left\{\begin{aligned} & \mathrm{e}^{|x|} & & \text{si $|x-x^0|\leq 1/2$}\\
                    & 0 & & \text{si $|x-x^0|> 1/2$}\end{aligned}\right.
                \end{equation}
                The function $f$ has bounded support, we can take $A=\{x\in\mathbb{R}^2:|x-x^0|\leq 1/2+\epsilon\}$ for all $\epsilon\in\intoo{0}{5/2-\sqrt{2}}$.
            \end{example}
        \subsecao{Parágrafo de Texto}\index{Exemplos!Parágrafo de Texto}
            \begin{example}[Nome do Exemplo]
                \lipsum[2]
            \end{example}

%--------------------------------------------------------------

%--------------------------------------------------------------
%   Exercícios
%--------------------------------------------------------------
    \secao{Exercícios}\index{Exercícios}

        Exemplo de exercício.
        \begin{exercise}
            This is a good place to ask a question to test learning progress or further cement ideas into students' minds.
        \end{exercise}

%--------------------------------------------------------------


%--------------------------------------------------------------
%   Problemas
%--------------------------------------------------------------
    \secao{Problemas}\index{Problemas}
        \begin{problem}
            Qual é o sentido do mundo?
        \end{problem}

%--------------------------------------------------------------



%--------------------------------------------------------------
%--------------------------------------------------------------
%  Nota de rodapé
%--------------------------------------------------------------
 \secao{Nota de rodapé}\index{Rodapé}

        Notas de rodapé são legais para informações extras\footnote{Exemplo de nota de rodapé}.


%--------------------------------------------------------------
%   Listas 
%--------------------------------------------------------------
    \secao{Listas}\index{Listas}

        Listas são úteis para apresentar informção de maneira concisa e ordenada.

        \subsecao{Lista Numerada}\index{Listas!Lista Numerada}

            \begin{enumerate}
                \item The first item
                \item The second item
                \item The third item
            \end{enumerate}

        \subsecao{Bolinhas}\index{Listas!Bullet Points}

            \begin{itemize}
                \item The first item
                \item The second item
                \item The third item
            \end{itemize}

Observação: tem como mudar de bolinha para outros símbolos, pergunte ao Rodolfo.

\secao{Listas aninhadas1}\index{Listas!Lista Aninhada}

\begin{enumerate}
   \item topico 1
     \begin{itemize}
         \item subitem 1
         \item subitem 2
      \end{itemize}
   \item topico 2
       \begin{itemize}
         \item subitem 1
         \item subitem 2
       \end{itemize}
\end{enumerate}


Outros exemplos de listas aqui \url{https://pt.overleaf.com/learn/latex/Lists}

%--------------------------------------------------------------


%--------------------------------------------------------------
%   Tabelas
%--------------------------------------------------------------
    \secao{Tabelas}\index{Tabelas}

        Para criar tabelas facilmente \href{https://www.tablesgenerator.com/}{\textbf{Clique Aqui}}

        %% INÍCIO DA TABELA %%
        \begin{table}[h]  %-- Início da Tabela
            %-- Legenda e Referências --%%
            \caption{Legenda Tabela}
            \label{tab:exemplo}             % comando para permitir citar a tabela
            \addcontentsline{toc}{table}{Tabela \ref{tab:exemplo}}

            %-- Corpo da Tabela --%
            \centering
            \begin{tabular}{l l l}
                \toprule
                \textbf{tópico 1} & \textbf{Valor 2} & \textbf{Valor 3}\\
                \midrule
                A 1 & 0.0003262 & 0.562 \\
                B 2 & 0.0015681 & 0.910 \\
                C 3 & 0.0009271 & 0.296 \\
                \bottomrule
            \end{tabular}
        \end{table}  %-- Fim da Tabela
        %% FIM TABELA %%


%---------------------------------------------------------------
%   Citação de tabela 
%---------------------------------------------------------------

    Como referenciar à Tabela  \ref{tab:exemplo} no texto automaticamente.

%--------------------------------------------------------------


\newcommand{\corLN}{blue!30}
\newcommand{\corLT}{green!30}
\newcommand{\corLA}{orange!50} %Líderes
\newcommand{\corFS}{orange!20}
\newcommand{\corFQ}{yellow!20}
\newcommand{\corFM}{blue!20}
\newcommand{\corAL}{gray!20}






%--------------------------------------------------------------
%  Figuras
%--------------------------------------------------------------
    \secao{Figuras}\index{Figuras}

        Aqui será acrescentada uma figura.
        % Para adicionar uma figura, copie o código abaixo e substitua o caminho "Pictures/placeholder.jpg" pelo nome da figura desejada, escolhendo-a nas opções que aparecerão. Outros formatos além do .jpg são suportados, na dúvida só testar.
        %
        % Caso queira modificar o tamanho da imagem, altere o valor "0.3" para o desejado. Este valor representa a porcentagem do espaço de uma linha.
        %
        % Para referenciá-la no texto, dê um nome único para o argumento "fig:ref" do comando \label{}, por ex.: \label{fig:motor1.jpg}

        \begin{figure}[htbp!]
            \centering
            \includegraphics[width=0.3\textwidth]{Pictures/placeholder.jpg}
            \caption{Legenda da Figura}
            \label{fig:ref}
        \end{figure}

%--------------------------------------------------------------
%   Citação de Figuras
%--------------------------------------------------------------
% O comando \ref{nome que esta no label} irá gerar o numero da figura automaticamente

Como mostrado na Figura \ref{fig:ref}, ....


%--------------------------------------------------------------



%--------------------------------------------------------------
% Figuras lado a lado
%--------------------------------------------------------------
        \subsecao{Figuras}\index{Figuras!Figuras lado a lado}
            
            Segue um exemplo de imagens lado a lado

            %Caso deseje adicionar uma figura embaixo de outra, modifique o valor "0.4" na propriedade "width" do comando "\includegraphics", em ambas as figuras, para um valor maior.

            \begin{figure}[H]
                \centering
                \subfigure[Legenda A]{
                    \includegraphics[width=0.4\textwidth]{Pictures/placeholder.jpg}
                    \label{fig:ref1}
                }
                \hfill
                \subfigure[Legenda B]{
                    \includegraphics[width=0.4\textwidth]{Pictures/placeholder.jpg}
                    \label{fig:ref2}
                }
                \caption{Legenda das duas Figuras}
                \label{fig:ref3}
            \end{figure}
            
            Apontando para as subfiguras \ref{fig:ref1} e \ref{fig:ref2} da imagem \ref{fig:ref3}.


            Alterando o valor da propriedade "width" para "0.5", teremos o seguinte efeito:
            \begin{figure}[H]
                \centering
                \subfigure[Legenda A]{
                    \includegraphics[width=0.5\textwidth]{Pictures/placeholder.jpg}
                    \label{fig:ref4}
                }
                \hfill
                \subfigure[Legenda B]{
                    \includegraphics[width=0.5\textwidth]{Pictures/placeholder.jpg}
                    \label{fig:ref5}
                }
                \caption{Legenda das duas Figuras}
                \label{fig:ref6}
            \end{figure}      

            Para 3 figuras:
        
            \begin{figure}[H]
                \centering
                \subfigure[Legenda A]{
                    \includegraphics[width=0.3\textwidth]{Pictures/placeholder.jpg}
                    \label{fig:ref7}
                }
                \hfill
                \subfigure[Legenda B]{
                    \includegraphics[width=0.3\textwidth]{Pictures/placeholder.jpg}
                    \label{fig:ref8}
                }
                \hfill
                \subfigure[Legenda C]{
                    \includegraphics[width=0.3\textwidth]{Pictures/placeholder.jpg}
                    \label{fig:ref9}
                }
                \caption{Legenda das duas Figuras}
                \label{fig:ref10}
            \end{figure}
            
            %   Lembre-se que o conteúdo nos comandos "\label{}" devem ser únicos para o referenciamento correto de cada imagem/subimagem.
        
%--------------------------------------------------------------
% Citação de referência bibliográfica
%--------------------------------------------------------------
\secao{Citação de referência bibliográfica}\index{Referência}

Segundo \cite{sobrenome_id}, é assim que cita uma referência no \LaTeX.

\label{cap1}
\label{cap2}
\label{cap3}

%--------------------------------------------------------------
% Citação de capítulo
%--------------------------------------------------------------
\secao{Citação de capítulo}\index{Citação de Capítulo}


Citação do capítulo \ref{cap1} , capítulo \ref{cap3}, ...., é o mesmo esquema para imagens e tabelas.


%--------------------------------------------------------------
%   Corolários
%--------------------------------------------------------------
    \secao{Corolários}\index{Corolários}

        Exemplo de Corolário
        \begin{corollary}[Corollary name]
            The concepts presented here are now in conventional employment in mathematics. Vector spaces are taken over the field $\mathbb{K}=\mathbb{R}$, however, established properties are easily extended to $\mathbb{K}=\mathbb{C}$.
        \end{corollary}

%--------------------------------------------------------------



%--------------------------------------------------------------
%   Descrição
%--------------------------------------------------------------

        \subsecao{Descrição e Definições}\index{Listas!Descrição e Definições}

            \begin{description}
                \item[Name] Description
                \item[Word] Definition
                \item[Comment] Elaboration
            \end{description}
%--------------------------------------------------------------

%--------------------------------------------------------------
%   Vocabulário
%--------------------------------------------------------------
    \secao{Vocabulário}\index{Vocabulário}

        Define a word to improve a students' vocabulary.

        \begin{vocabulary}[Palavra]
            Definição da palavra.
        \end{vocabulary}