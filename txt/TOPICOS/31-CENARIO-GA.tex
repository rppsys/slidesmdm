% Cenário Gartner A
\newcommand{\cenGA}{Cenário \emph{Gartner} A: Distribuição Equivalente} 
\section{\cenGA}
\label{sec-cenga}
    O primeiro cenário criado é equivalente ao cenário de caso de uso ``General Analytics'' na qual o Grupo \emph{Gartner} distribui pesos de forma ``equivalente'' entre as 15 áreas de capacidade. Isto porque ao se distribuir 100\% por 15 áreas temos $\frac{100}{15} = 6,66$ e assim na realidade o \emph{Gartner} atribui peso de 6\% para alguns e peso de 7\% para outros itens.
    
\subsection*{Distribuição dos Pesos}    
    
    Aqui faremos algo semelhante. A diferença é que vamos atribuir 7\% para áreas críticas de capacidade \MUST e \SHOULD e 6\% para as demais. Fazendo isso a soma dos pesos daria 99\% daí este 1\% restante é adicionado à área crítica de capacidade de Geração de Relatórios. A tabela \ref{tab:cenGA:pesos} apresenta como esses pesos foram distribuídos para cada uma das áreas críticas de capacidade.

    % cenGA - Tabela de Pesos
    \begin{table}[!h]
        \begin{center}
        \begin{tabular}{|p{0.4\textwidth}|c|c|}
            \hline
                % NOME DA TABELA        
                \rowcolor{cldfB1} \multicolumn{3}{|c|}{\Large \cenGA} \\  
                \rowcolor{cldfB1}
                \multicolumn{3}{|c|}{\large \textbf{Tabela de Pesos}} \\ \hline \hline
                % CABEÇALHO        
                \rowcolor{lightgray}\textbf{Áreas de Capacidade} & \textbf{Classe MoSCoW} & \textbf{Pesos} \\ \hline
                % CONTEÚDO
                % Código gerado pela tabela do Google SpreadSheets Cenário GA
                \rowcolor{corMUST!80}Segurança & MUST & 7\% \\ \hline
                \rowcolor{corMUST!80}Capacidade de Gerenciamento & MUST & 7\% \\ \hline
                \rowcolor{corMUST!80}Conectividade de Fontes de Dados & MUST & 7\% \\ \hline
                \rowcolor{corMUST!80}Preparação de Dados & MUST & 7\% \\ \hline
                \rowcolor{corMUST!80}Visualização de Dados & MUST & 7\% \\ \hline
                \rowcolor{corSHOULD!80}Nuvem & SHOULD & 7\% \\ \hline
                \rowcolor{corSHOULD!80}Complexidade de Modelos & SHOULD & 7\% \\ \hline
                \rowcolor{corSHOULD!80}Catálogos & SHOULD & 7\% \\ \hline
                \rowcolor{corSHOULD!80}Incorporação de Análises & SHOULD & 7\% \\ \hline
                \rowcolor{corCOULD!50}Geração de Relatórios & COULD & 7\% \\ \hline
                \rowcolor{corCOULD!50}Analytics Avançados & COULD & 6\% \\ \hline
                \rowcolor{corCOULD!50}Data Storytelling & COULD & 6\% \\ \hline
                \rowcolor{corWOULD!50}Insight Automatizados & WOULD & 6\% \\ \hline
                \rowcolor{corWOULD!50}Consulta em Linguagem Natural & WOULD & 6\% \\ \hline
                \rowcolor{corWOULD!50}Geração de Linguagem Natural & WOULD & 6\% \\ \hline
                % TOTAL
                \rowcolor{lightgray!30} \multicolumn{2}{|r|}{\large Total: \normalsize} & 100\% \\ \hline 
        \end{tabular}    
        \caption{\label{tab:cenGA:pesos} Pesos para \cenGA}
        \end{center}
    \end{table}
       
\subsection*{Resultados}    

    Ao multiplicar os pesos da tabela \ref{tab:cenGA:pesos} aos \emph{scores} da tabela apresentada no Anexo \ref{anexo-tabelacc} encontramos os resultados exibidos na tabela \ref{tab:cenGA:resultados}.

    % cenGA - Tabela de Resultados
    \begin{table}[!h]
        \begin{center}
        \begin{tabular}{|c|cc|}
            \hline
                % NOME DA TABELA        
                \rowcolor{cldfB1} \multicolumn{3}{|c|}{\Large \cenGA} \\  
                \rowcolor{cldfB1}
                \multicolumn{3}{|c|}{\large \textbf{Resultados}} \\ \hline \hline
                % CABEÇALHO        
                \rowcolor{lightgray}\textbf{Fornecedor} & \multicolumn{2}{c|}{\textbf{\emph{Score} [1-5]}} \\ \hline
                % CONTEÚDO
                % Código gerado pela tabela do Google SpreadSheets Cenário GA
                \rowcolor{corP1!80}Microsoft & \progressbar{0.856} & 4,3 \\ \hline
                \rowcolor{corP1!80}MicroStrategy & \progressbar{0.857} & 4,3 \\ \hline
                \rowcolor{corP2!50}Salesforce & \progressbar{0.843} & 4,2 \\ \hline
                \rowcolor{corP2!50}TIBCO Software & \progressbar{0.838} & 4,2 \\ \hline
                \rowcolor{corP2!50}Yellowfin & \progressbar{0.84} & 4,2 \\ \hline
                \rowcolor{corP3!30}Qlik & \progressbar{0.818} & 4,1 \\ \hline
                \rowcolor{corP3!30}Tableau & \progressbar{0.823} & 4,1 \\ \hline
                \rowcolor{corPF!20}Domo & \progressbar{0.804} & 4,0 \\ \hline
                \rowcolor{corPF!20}Oracle & \progressbar{0.806} & 4,0 \\ \hline
                \rowcolor{corPF!20}SAS & \progressbar{0.802} & 4,0 \\ \hline
                \rowcolor{corPF!20}Sisence & \progressbar{0.805} & 4,0 \\ \hline
                \rowcolor{corPF!20}ThoughtSpot & \progressbar{0.791} & 4,0 \\ \hline
                \rowcolor{corPF!20}IBM & \progressbar{0.752} & 3,8 \\ \hline
                \rowcolor{corPF!20}Infor Builders & \progressbar{0.756} & 3,8 \\ \hline
                \rowcolor{corPF!20}Birst & \progressbar{0.74} & 3,7 \\ \hline
                \rowcolor{corPF!20}Board & \progressbar{0.714} & 3,6 \\ \hline
                \rowcolor{corPF!20}SAP & \progressbar{0.72} & 3,6 \\ \hline
                \rowcolor{corPF!20}Logi & \progressbar{0.705} & 3,5 \\ \hline
                \rowcolor{corPF!20}Looker & \progressbar{0.657} & 3,3 \\ \hline
                \rowcolor{corPF!20}Alibaba & \progressbar{0.602} & 3,0 \\ \hline
        \end{tabular}    
        \caption{\label{tab:cenGA:resultados} Resultados para \cenGA}
        \end{center}
    \end{table}

\subsection*{Análise dos Resultados}    

    Dessa forma, analisando os resultados obtidos nesta tabela \ref{tab:cenGA:resultados} podemos verificar o seguinte:
    
    \begin{itemize}
        \item A \emph{Microsoft} e a \emph{MicroStrategy} empatam em primeiro lugar com $4,3$ pontos;
        \item A \emph{Salesforce}, \emph{TIBCO Software} e \emph{Yellowfin} empatam em segundo lugar com $4,2$ pontos cada;
        \item A \emph{Qlik} e \emph{Tableau} aparecem em terceiro lugar com $4,1$ pontos cada;
        \item Todos os demais fornecedores são eliminados;
    \end{itemize}
    
    O \cenGA \xspace apresenta fornecedores que podem ser considerados em um caso genérico onde todas as áreas apresentam mesma importância. Contudo, o caso da CLDF exige uma distribuição de pesos mais específica.