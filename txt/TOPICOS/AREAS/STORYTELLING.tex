\subsection{Capacidade de Contar Histórias a Partir de Dados (\emph{Data Storytelling})}
\label{sub-storytelling}
\index{Data Storytelling}
\index{Histórias}

De acordo com o \relGMQ \xspace \cite{gartner:magicquadrant}, a capacidade de gerar histórias a partir de dados será um dos recursos mais procurados em plataformas de ABI. E portanto, os diferentes fornecedores estão buscando desenvolver esse tipo de capacidade para diferenciar seus produtos \cite[6]{analisegartner2020}.

Assim, podemos definir a capacidade de fazer \emph{data storytelling} como:

\begin{definition}[Capacidade de Realizar \emph{Data Storytelling}]
    A capacidade de combinar visualização de dados interativos com técnicas narrativas, a fim de demonstrar e fornecer \emph{insights} de uma forma atraente e de fácil compreensão para apresentação aos tomadores de decisão.
\end{definition}


% 3 - COULD

Em um primeiro momento, comparando esta área com as demais, a opção inicial de classificação \emph{MoSCoW} para a área de capacidade de \emph{Data Storytelling} foi WOULD. Entretanto, em virtude da citada previsão, optou-se por elevar a importância desta área classificando-a como \COULD.
