\subsection{Conectividade de Fontes de Dados (\emph{Data Source Connectivity})}
\label{sub-connectivity}
\index{Conectividade}
\index{Fontes de Dados}

% 1 - MUST 

Essa é outra capacidade muito relevante. Como o objetivo principal da solução de ABI é a modernização da função institucional de fiscalização, entende-se que os dados que serão fiscalizados estão distribuídos em diversos órgãos usando as mais variadas tecnologias e nos mais diversos formatos. 

Podemos definir essa capacidade como:

\begin{definition}[Capacidade de Conectividade de Fontes de Dados]
Recursos que permitem que os usuários se conectem e incluam dados estruturados e não estruturados contidos em vários tipos de plataformas de armazenamento, tanto no local quanto na nuvem.
\end{definition}

De fato, a competência de Fiscalização da Câmara Legislativa do Distrito Federal é em grande parte exercida pelas Comissões, Permanentes e Temporárias. Atualmente a CLDF conta com 11 (onze) Comissões Permanentes para exercer as atribuições que lhes caibam. Ora, se tão somente o levantamento de necessidades da Comissão de Defesa dos Direitos Humanos, Cidadania,
Ética e Decoro Parlamentar \cite{propostaCDDHCEDP} apontou um amplo conjunto de indicadores cuja construção decorre do cruzamento de informações contidas em diferentes fontes de dados, então as necessidades de informação ainda não mapeadas das outras dez comissões devem depender de um conjunto ainda maior de fontes de dados. 

A habilidade de se conectar a esses recursos é primordial. Assim, torna-se evidente que a capacidade de Conectividade de Fontes de Dados deve ser qualificada como \MUST.

