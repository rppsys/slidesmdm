\subsection{Preparação de Dados (\emph{Data Preparation})}
\label{sub-preparation}
\index{Preparação de Dados}

A habilidade de se conectar com distintas fontes de dados apresentada na seção anterior (\ref{sub-connectivity}) vai inevitavelmente gerar a necessidade de transformar e combinar esses dados a fim de construir os indicadores de interesse. Em outras palavras, a capacidade de conectividade cria a necessidade de preparar os dados para uso. 

Podemos conceituar a Capacidade de Preparação de Dados como segue:

\begin{definition}[Capacidade de Preparação de Dados]
Suporte para recursos do tipo ``arrastar e soltar'', realizar a combinação de diferentes fontes de dados e a criação de modelos analíticos (como medidas, conjuntos, grupos e hierarquias definidas pelo usuário).
\end{definition}

Por conseguinte, faz sentido avaliar a capacidade de preparação de dados como pertencente à classe \MUST.
