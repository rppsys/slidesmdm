\subsection{Geração de Relatórios (\emph{Reporting})}
\label{sub-reporting}
\index{Relatórios}
\index{Geração de Relatórios}

Umas das funções desejadas em soluções de ABI é a capacidade de gerar relatórios:

\begin{definition}[Capacidade de Geração de Relatórios]
A capacidade de criar e distribuir relatórios de \emph{layout} de grade com várias páginas de forma programada.
\end{definition}

% 3 - COULD
Neste contexto, durante o estudo isolado das diferentes áreas de capacidade cria-se o desejo de classifica-las todas como MUST\footnote{Chegamos, inclusive, a produzir o estudo de caso \ref{sec-cenga} atribuindo mesmo peso para todas as áreas.}. Porém, fazer isso é dizer que todas as áreas são críticas e apresentam a mesma importância (ou mesmo dizer que todas não são críticas e são todas igualmente sem importância). Sabemos que isto não é verdade e que a solução mais adequada depende do alinhamento entre as capacidades do produto e necessidades específicas identificadas. 

Assim, dentro do contexto particular de necessidades para modernização da fiscalização, verifica-se que a geração de relatórios não é uma necessidade tão importante quanto é, por exemplo, a funcionalidade de criação de alertas que será descrita na seção \ref{sec-alertas}. Dessa feita, optou-se por avaliar a Área de Capacidade de Geração de Relatórios como \COULD.