\chapter{Introdução}
\label{cap-intro}

    A Coordenadoria de Modernização e Informática (CMI) da Câmara Legislativa do Distrito Federal (CLDF) tem por finalidade o assessoramento especializado em computação à Mesa Diretora e o contínuo aperfeiçoamento do Sistema de Informação da CLDF, abrangendo as funções institucionais de representação, legiferação, fiscalização e administração, de acordo com a Estratégia de Sistema de Informação \cite{normativo:res312,asiESI}.
    
    Dessa forma, para atender a diretriz de tecnologia da informação de equilibrar a entrega de soluções entre as funções institucionais, a CMI planejou uma série de projetos com objetivo de fortalecer a Função Institucional de Fiscalização.
    
    Assim, durante o primeiro semestre de 2020 a Área de Sistema de Informação (ASI) desenvolveu projeto em parceria com a Comissão de Defesa dos Direitos Humanos, Cidadania, Ética e Decoro Parlamentar (CDDHCEDP) com objetivo de produzir \emph{proposta} técnico-científica de computação para modernizar a função institucional finalística de fiscalização, com aplicação computacional de ciência de dados e \emph{business intelligence}. 

    Em seguida, no segundo semestre de 2020, a CMI designou servidores, por meio do Ato do Vice-Presidente Nº 56, de 2020, para elaborar estudo técnico sobre o tema. Assim, iniciou-se  o desenvolvimento de projeto intitulado ``Estudo Técnico sobre Plataformas de \emph{Business Intelligence}'' elaborando-se um termo de abertura contendo planejamento das principais atividades do projeto para atender uma série de objetivos específicos estabelecidos.
    
    Este estudo pretende atingir os objetivos específicos de: realizar análise comparativa de plataformas de BI a partir de levantamento bibliográfico sobre o assunto; identificar e avaliar requisitos funcionais e não funcionais a serem atendidos por uma futura plataforma de BI; criar diretrizes para arquitetura corporativa de BI a ser utilizada na CLDF, de modo a padronizar as implantações e otimizar as atividades de implantação, operação, expansão e evolução desta tecnologia; e assim, oferecer condições para auxiliar a(s) autoridade(s) competente(s) nas contratações e planejamentos de trabalhos relacionados ao tema. 