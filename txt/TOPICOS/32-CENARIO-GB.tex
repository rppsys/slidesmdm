% Cenário Gartner B
\newcommand{\cenGB}{Cenário \emph{Gartner} B: Distribuição \emph{MoSCoW} \emph{MUST-Only}} 
\section{\cenGB}

    O segundo cenário criado já passa a levar as classes \emph{MoSCoW} sugeridas na seção \ref{sec-avaliacao} em conta para distribuição dos pesos.
    
\subsection*{Distribuição dos Pesos}    


    Dessa vez decidimos criar uma distribuição \emph{MoSCoW} ``\emph{MUST-Only}'', ou seja, o total de 100\% dos pesos será distribuído de forma igual apenas entre as áreas de capacidade classificadas como \MUST. As demais áreas receberão peso nulo.
    
    Assim, como temos 5 áreas críticas de capacidade classificadas como \MUST cada uma receberá peso de 20\% conforme pode ser verificado na tabela \ref{tab:cenGB:pesos}.
    
    % cenGB - Tabela de Pesos
    \begin{table}[!h]
        \begin{center}
        \begin{tabular}{|p{0.4\textwidth}|c|c|}
            \hline
                % NOME DA TABELA        
                \rowcolor{cldfB1} \multicolumn{3}{|c|}{\Large \cenGB} \\  
                \rowcolor{cldfB1}
                \multicolumn{3}{|c|}{\large \textbf{Tabela de Pesos}} \\ \hline \hline
                % CABEÇALHO        
                \rowcolor{lightgray}\textbf{Áreas de Capacidade} & \textbf{Classe MoSCoW} & \textbf{Pesos} \\ \hline
                % CONTEÚDO
                % Código gerado pela tabela do Google SpreadSheet Cenário GB
                \rowcolor{corMUST!80}Segurança & MUST & 20\% \\ \hline
                \rowcolor{corMUST!80}Capacidade de Gerenciamento & MUST & 20\% \\ \hline
                \rowcolor{corMUST!80}Conectividade de Fontes de Dados & MUST & 20\% \\ \hline
                \rowcolor{corMUST!80}Preparação de Dados & MUST & 20\% \\ \hline
                \rowcolor{corMUST!80}Visualização de Dados & MUST & 20\% \\ \hline
                \rowcolor{corSHOULD!80}Nuvem & SHOULD & 0\% \\ \hline
                \rowcolor{corSHOULD!80}Complexidade de Modelos & SHOULD & 0\% \\ \hline
                \rowcolor{corSHOULD!80}Catálogos & SHOULD & 0\% \\ \hline
                \rowcolor{corSHOULD!80}Incorporação de Análises & SHOULD & 0\% \\ \hline
                \rowcolor{corCOULD!50}Geração de Relatórios & COULD & 0\% \\ \hline
                \rowcolor{corCOULD!50}Analytics Avançados & COULD & 0\% \\ \hline
                \rowcolor{corCOULD!50}Data Storytelling & COULD & 0\% \\ \hline
                \rowcolor{corWOULD!50}Insight Automatizados & WOULD & 0\% \\ \hline
                \rowcolor{corWOULD!50}Consulta em Linguagem Natural & WOULD & 0\% \\ \hline
                \rowcolor{corWOULD!50}Geração de Linguagem Natural & WOULD & 0\% \\ \hline
                % TOTAL
                \rowcolor{lightgray!30} \multicolumn{2}{|r|}{\large Total: \normalsize} & 100\% \\ \hline 
        \end{tabular}    
        \caption{\label{tab:cenGB:pesos} Pesos para \cenGB}
        \end{center}
    \end{table}    

\subsection*{Resultados}   

    De maneira semelhante, ao multiplicar os pesos da tabela \ref{tab:cenGB:pesos} aos \emph{scores} da tabela apresentada no Anexo \ref{anexo-tabelacc} encontramos os resultados exibidos na tabela \ref{tab:cenGB:resultados}.
    
    % cenGB - Tabela de Resultados
    \begin{table}[!h]
        \begin{center}
        \begin{tabular}{|c|cc|}
            \hline
                % NOME DA TABELA        
                \rowcolor{cldfB1} \multicolumn{3}{|c|}{\Large \cenGB} \\  
                \rowcolor{cldfB1}
                \multicolumn{3}{|c|}{\large \textbf{Resultados}} \\ \hline \hline
                % CABEÇALHO        
                \rowcolor{lightgray}\textbf{Fornecedor} & \multicolumn{2}{c|}{\textbf{\emph{Score} [1-5]}} \\ \hline
                % CONTEÚDO
                % Código gerado pela tabela do Google SpreadSheets Cenário GB
                \rowcolor{corP1!80}MicroStrategy & \progressbar{0.928} & 4,6 \\ \hline
                \rowcolor{corP2!50}Yellowfin & \progressbar{0.892} & 4,5 \\ \hline
                \rowcolor{corP3!30}Tableau & \progressbar{0.884} & 4,4 \\ \hline
                \rowcolor{corP3!30}TIBCO Software & \progressbar{0.88} & 4,4 \\ \hline
                \rowcolor{corPF!20}Domo & \progressbar{0.868} & 4,3 \\ \hline
                \rowcolor{corPF!20}Microsoft & \progressbar{0.856} & 4,3 \\ \hline
                \rowcolor{corPF!20}Qlik & \progressbar{0.868} & 4,3 \\ \hline
                \rowcolor{corPF!20}Salesforce & \progressbar{0.852} & 4,3 \\ \hline
                \rowcolor{corPF!20}Sisence & \progressbar{0.864} & 4,3 \\ \hline
                \rowcolor{corPF!20}Birst & \progressbar{0.832} & 4,2 \\ \hline
                \rowcolor{corPF!20}Infor Builders & \progressbar{0.844} & 4,2 \\ \hline
                \rowcolor{corPF!20}SAS & \progressbar{0.84} & 4,2 \\ \hline
                \rowcolor{corPF!20}Logi & \progressbar{0.812} & 4,1 \\ \hline
                \rowcolor{corPF!20}Oracle & \progressbar{0.796} & 4,0 \\ \hline
                \rowcolor{corPF!20}Board & \progressbar{0.744} & 3,7 \\ \hline
                \rowcolor{corPF!20}ThoughtSpot & \progressbar{0.74} & 3,7 \\ \hline
                \rowcolor{corPF!20}IBM & \progressbar{0.724} & 3,6 \\ \hline
                \rowcolor{corPF!20}Alibaba & \progressbar{0.692} & 3,5 \\ \hline
                \rowcolor{corPF!20}Looker & \progressbar{0.7} & 3,5 \\ \hline
                \rowcolor{corPF!20}SAP & \progressbar{0.692} & 3,5 \\ \hline
        \end{tabular}    
        \caption{\label{tab:cenGB:resultados} Resultados para \cenGB}
        \end{center}
    \end{table}    
    
\subsection*{Análise dos Resultados}    

    Desta vez, ao atribuir 100\% do total dos pesos de forma igual entre as áreas de capacidade classificadas como \MUST, nosso critério de escolha definido na seção \ref{sec-criteriog} exibe os seguintes resultados: 
    
    \begin{itemize}
        \item A \emph{MicroStrategy} lidera em primeiro lugar com $4,6$ pontos;
        \item A \emph{Yellowfin} aparece sozinha em segundo lugar com $4,5$ ponto;
        \item A \emph{Tableau} e a \emph{TIBCO Software} empatam em terceiro lugar com $4,4$ pontos cada;
        \item Todas os demais fornecedores são eliminados;
    \end{itemize}
    
    O \cenGB \xspace apresenta potenciais fornecedores para atender as necessidades mais imediatas da \CLDF de modernização da fiscalização.
    
    Contudo, ainda não ficamos satisfeitos com este cenário pois, mesmo que as áreas críticas de capacidade \SHOULD não sejam críticas, seu grau de importância não pode ser desprezado.