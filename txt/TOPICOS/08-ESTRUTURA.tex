\chapter{Estrutura do Estudo}
\label{cap-estrutura}

    O ``\autoref{cap-intro} -- \nameref{cap-intro}'' introduz este estudo técnico apresentando o contexto de elaboração, a motivação e os objetivos específicos que pretende-se atingir. Em seguida, o ``\autoref{cap-descricao} -- \nameref{cap-descricao}'' relata o problema a partir da criação de uma metáfora utilizada para apresentar diversos cenários possíveis e propor um cenário ideal de ambiente de análise de dados. Este ``\autoref{cap-estrutura} -- \nameref{cap-estrutura}'' descreve como esse estudo técnico está organizado.
    
    Após esses três primeiros capítulos introdutórios, apresentamos a ``\autoref{parte-revisao} -- \nameref{parte-revisao}'' composta por dois capítulos.
    O ``\autoref{cap-literatura} -- \nameref{cap-literatura}'' aponta o material utilizado para fundamentar o trabalho desenvolvido. A partir de então, os conhecimentos necessários para entender o assunto são apresentados no ``\autoref{cap-referencial} -- \nameref{cap-referencial}''.  
    
    A seguir, munido dos esclarecimentos referentes aos assuntos abordados no estudo, o leitor é levado para a ``\autoref{parte-necessidades} -- \nameref{parte-necessidades}'' cujo ``\autoref{cap-necessidades} -- \nameref{cap-necessidades}'' apresenta uma metodologia para avaliar necessidades a partir da identificação e avaliação de um conjunto de áreas de capacidade de interesse.
    
    A partir do quadro de necessidades delineado, podemos então produzir casos de uso personalizados utilizando os dados de relatórios técnicos a fim de comparar as plataformas de BI de diferentes fornecedores. Isto é realizado na ``\autoref{parte-estudosdecaso} -- \nameref{parte-estudosdecaso}'' onde o ``\autoref{cap-casos-gartner} -- \nameref{cap-casos-gartner}'' produz casos de uso utilizando dados do \relGCC \xspace e o ``\autoref{cap-casos-forrester} -- \nameref{cap-casos-forrester}'' produz casos de uso utilizando dados do \relFCM.
    
    Neste momento, como os casos de uso desenvolvidos compararam diversos fornecedores, cabe descrever os fornecedores de destaque. Isto é feito na ``\autoref{parte-plataformas} -- \nameref{parte-plataformas}'' durante o  ``\autoref{cap-fornecedores} -- \nameref{cap-fornecedores}''.  
    
    Até aqui já atingimos diversos objetivos específicos do projeto, mas para completar o estudo elaboramos a ``\autoref{parte-proposta} -- \nameref{parte-proposta}'' contendo diretrizes de implantação de BI na CLDF. Dessa forma, o ``\autoref{cap-acoes} -- \nameref{cap-acoes}'' retoma a metáfora introduzida na descrição do problema, traça o cenário atual e recomenda ações para que se atinja o cenário ideal de implantação de um ambiente de análise de dados. Ainda, o ``\autoref{cap-esforcos} -- \nameref{cap-esforcos}'' sugere uma sequência de etapas para que a implantação do ambiente de análise de dados evite situações indesejadas caracterizadas pelos cenários apresentados. Por fim, o ``\autoref{cap-proposta} -- \nameref{cap-proposta}'' utiliza as lições aprendidas durante esse estudo para propor um projeto de implantação de ambiente de análise de dados realizável com os recursos disponíveis.
    
    Em seguida, a ``\autoref{parte-conclusoes} -- \nameref{parte-conclusoes}'' finaliza o estudo com as conclusões apresentadas no ``\autoref{cap-conclusoes} -- \nameref{cap-conclusoes}''. Um índice é exibido na ``\autoref{parte-indice} -- \nameref{parte-indice}'' e a ``\autoref{parte-anexos} -- \nameref{parte-anexos}'' apresenta anexos contendo figuras e tabelas de referência utilizadas ao longo do estudo. 

%\label{cap-intro}
%\label{cap-descricao}
%\label{cap-estrutura}

%\label{parte-revisao}
%\label{cap-literatura} => Levantamento 
%\label{cap-referencial} => Referencial Teórico

%\label{parte-necessidades}
%\label{cap-necessidades} => Avaliando Necessidades / Requisitos

%\label{parte-estudosdecaso} => Análise Comparativa
%\label{cap-casos-gartner}
%\label{cap-casos-forrester}

%\label{parte-plataformas} Fornecedores de Destaque 
%\label{cap-fornecedores} => Fornecedores destacados nos casos de uso;

%\label{parte-proposta}
%\label{cap-acoes} => Ações para sair do cenário atual e buscar o cenário ideal
%\label{cap-esforcos} => Roadmap e Diretrizes de Implantação de BI
%\label{cap-proposta}

%\label{parte-conclusoes}
%\label{cap-conclusoes}


%\label{parte-indice}


%\label{parte-anexos}
%\label{anexo-landscape}
%\label{anexo-gartner}
%\label{anexo-tabelacc}
%label{anexo-forrester}
%\label{anexo-tabelafw}












