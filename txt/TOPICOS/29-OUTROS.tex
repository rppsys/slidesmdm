\section{Necessidades da Comissão} 
\label{sec-necessidadescomissao}

Essa seção pretende completar o capítulo de necessidades destacando algumas necessidades específicas identificadas pela \CDDHCEDP.

\subsection{Sistema de Alertas}
\label{sec-alertas}

Uma funcionalidade desejada é a capacidade de criar alertas. Sistemas de alerta monitoram os dados e informam todas as pessoas-chave que precisam saber sobre eventos críticos assim que acontecem \cite{turban2019}.

Em entrevista à \CDDHCEDP, foi levantado o requisito de que a ferramenta tenha meios de gerar alertas quando certas situações ocorrerem. O exemplo aludido foi o de uma potencial violação de Direitos Humanos, que possa ser aferida quando determinados padrões nos dados forem detectados, de modo a criar alertas para a atuação da Comissão.

De fato, as Comissões Permanentes possuem um quadro reduzido, que dificulta o monitoramento constante de painéis, com contínua interpretação desses para a tomada de atitudes. Nesse quesito, a geração de alertas permite a proatividade da Comissão em seus trabalhos com menor consumo de recursos humanos.

%Um ``Sistema de Alertas'' geralmente é oferecido como um software próprio e não uma capacidade. Mas com certeza, se tivéssemos que avaliar esse item de acordo com a classificação \emph{MoSCoW}, ele seria classificado como \MUST.

\subsection{Dados Georreferenciados}
\label{sec-geo}

A exploração de dados utilizando georreferenciamento é outro recurso bastante procurado pelas instituições. Trata-se da capacidade de utilizar dados georreferenciados provenientes de Sistemas de Informações Geográficas (SIG). Trata-se, na verdade de um recurso da área de capacidade de ``Visualização de Dados'' já descrita na seção \ref{sub-visualization}. 

O dados georreferenciados são de grande relevância para a atuação governamental, tendo-se em vista a abrangência territorial das ações governamentais e políticas públicas, de modo a se perceber desvios que possuam um espoco territorial bem definido. Assim, percebe-se que esses dados são capazes de entregar o valor do melhor dimensionamento e otimização das políticas públicas em razão das peculiaridades geográficas da sua abrangência.

%Portanto, a capacidade de produzir visualizações com dados georreferenciados também será qualificado como \MUST.


