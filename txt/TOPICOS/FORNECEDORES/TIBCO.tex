\section{\emph{TIBCO Software}}
\label{sub-tibco}
\index{TIBCO Software}

\begin{wrapfigure}[3]{r}{0.45\textwidth}     
    \centering
    \includegraphics[width=0.35\textwidth,height=1.5cm]{fig/f/tibco.png}
\end{wrapfigure}
O produto avaliado é o TIBCO Spotfire 10.6.

\subsection*{Destaques}

O TIBCO Spotfire possui forte avaliação na área de capacidade de \emph{analytic} avançado e suporta agendamento de relatórios personalizados, além de agendamento baseado em eventos, além de prover alertas em tempo real. O TIBCO Spotfire oferece algumas das mais robustas capacidades de \emph{analytics} avançado, sendo fácil para o usuário construir regressões, árvores de decisão e predições com uso de interfaces baseadas em menu \cite{gartner:criticalcapabilities}.

\subsection*{Pontos Fortes}

O \relGMQ \xspace e o \relGCC \xspace trazem boas avaliações do TIBCO Spotfire nas áreas de capacidade crítica de \emph{analytics} avançados, visualização de dados, geração de relatórios, conectividade de dados e preparação de dados.

\subsection*{Pontos Fracos}

A ferramenta possui uma comunidade menor engajada com o uso da ferramenta, além de possuir um custo que é considerado como barreira por diversos clientes \cite{gartner:magicquadrant}.

\subsection*{Avaliação}

O TIBCO Spotfire ficou entre os 3 primeiros fornecedores em todos os 6 cenários de casos de uso elaborados com base no levantamento de necessidades. Assim sendo, mostra-se uma boa alternativa dentre as identificadas no estudo.

% \cite{gartner:magicquadrant}
% \cite{gartner:criticalcapabilities}