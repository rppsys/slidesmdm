\section{\emph{Domo}}
\label{sub-domo}
\index{Domo}

\begin{wrapfigure}[6]{r}{0.25\textwidth}     
    \centering
    \includegraphics[width=0.2\textwidth]{fig/f/domo.png}
\end{wrapfigure}

O produto avaliado é o Domo (Novembro/2019).

\subsection*{Destaques}

O Domo possui como ponto forte a facilidade de uso, com interface intuitiva para criar \emph{cards} e organizá-los em coleções ou páginas. Dessa forma, torna-se simples o desenvolvimento colaborativo e interativo entre autores e consumidores de conteúdo. Para catálogos de dados, os autores podem adicionar descrições nos objetos e implementar texto explicativo, que ficam disponíveis nas navegações da ferramenta. Ademais, existe indexação do impacto gerado por alteração nos \emph{cards}, fluxos de dados, outros repositórios e alertas \cite{gartner:criticalcapabilities}.

\subsection*{Pontos Fortes}

O \relGMQ \xspace e o \relGCC \xspace trazem boas avaliações do Domo nas áreas de capacidade crítica de incorporação de análises e segurança.

\subsection*{Pontos Fracos}

A ferramenta costuma ser implantada por linhas de negócio pelos clientes, ao invés de normalizada pela TI, deixando de se mostrar forte como solução única corporativa; ademais, a presença geográfica não contempla o Brasil \cite{gartner:magicquadrant}.

\subsection*{Avaliação}

O Domo ficou entre os 3 primeiros fornecedores em 1 dos 6 cenários de caso de uso elaborados com base no levantamento de necessidades. Seu resultado mostra uma ferramenta que é menos recomendável do que outras avaliadas.

% \cite{gartner:magicquadrant}
% \cite{gartner:criticalcapabilities}