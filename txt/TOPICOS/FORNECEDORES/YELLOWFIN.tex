\section{\emph{Yellowfin}}
\label{sub-yellowfin}
\index{Yellowfin}

\begin{wrapfigure}[3]{r}{0.4\textwidth}     
    \centering
    \includegraphics[width=0.35\textwidth,height=1.5cm]{fig/f/yellowfin.png}
\end{wrapfigure}

O produto avaliado é o Yellowfin BI versão 8.

\subsection*{Destaques}

O Yellowfin destaca-se pela variedade de opções para estórias de dados analíticos, jornalísticos e infográficos. A funcionalidade Yellowfin Present permite aos usuários a criação de \emph{slides} com relatórios totalmente interativos. O seu módulo \emph{Signals} fornece capacidades destacadas de \emph{insights} automatizados, usando aprendizado de máquina para realizar análises e detecção de pontos fora da curva \cite{gartner:criticalcapabilities}.

\subsection*{Pontos Fortes}

O \relGMQ \xspace e o \relGCC \xspace trazem boas avaliações do Yellowfin nas áreas de capacidade de preparação de dados, \emph{data storytelling}, geração de relatórios, segurança, geração de linguagem natural, catálogo e \emph{insights} automatizados.

\subsection*{Pontos Fracos}

A ferramenta possui menos opções de conectividade de dados, além de pouca presença de mercado e presença geográfica restrita \cite{gartner:magicquadrant}.

\subsection*{Avaliação}

O Yellowfin ficou entre os 3 primeiros fornecedores em 5 dos 6 cenários de caso de uso elaborados com base no levantamento de necessidades. Seu resultado é considerado equilibrado, se mostrando como uma alternativa de recomendação média dentre as identificadas neste estudo.

% \cite{gartner:magicquadrant}
% \cite{gartner:criticalcapabilities}