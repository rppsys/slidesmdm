\section{\emph{Salesforce} \& \emph{Tableau}}
\label{sub-salesforce}
\index{Salesforce}

\begin{wrapfigure}[5]{r}{0.65\textwidth}     
    \centering
    \includegraphics[width=0.55\textwidth,height=2cm]{fig/f/salesforce-tableau.png}
\end{wrapfigure}

A Salesforce anunciou um plano para adquirir o Tableau em junho de 2019 e a aquisição foi concluído em 1º de agosto do mesmo ano. No entanto, a Autoridade de Concorrência e Mercados do Reino Unido (\emph{Competition and Markets Authority} - CMA) exigiu que o Salesforce e o Tableau operassem separadamente por meio de uma ordem de \emph{``hold separate''}. O CMA suspendeu este pedido no final de novembro de 2019. Como resultado, o produto e
os planos de integração não foram desenvolvidos e disponibilizados para serem compartilhados com o \emph{Gartner} a tempo de serem considerados pelo \relGMQ \xspace de 2020. Assim, eles aparecem como fornecedores separados \cite{gartner:magicquadrant}.

Antes de adquirir o Tableau, a Salesforce oferecia uma ferramenta chamada de ``Salesforce Einstein Analytics'', contudo quando consultamos o sítio eletrônico da Salesforce eles já apresentam o Tableau como seu principal produto de \emph{business intelligence}. Dessa forma, nessa análise, vamos avaliar o Tableau.

% https://www.tableau.com/pt-br/about/press-releases/2019/salesforce-completes-acquisition-tableau

% https://www.tableau.com/pt-br/about/press-releases/2019/salesforce-completes-acquisition-tableau

\subsection*{Destaques}

O Tableau destaca-se na área de \emph{analytics} avançado e suas capacidades de governança de dados. Seus recursos de análise de dados contam com consulta em linguagem natural e geração automática de \emph{insights}. Os recursos de governança de dados contam com ferramentas para agendar e monitorar tarefas de gerenciamento de dados.


\subsection*{Pontos Fortes}

O Tableau facilita a exploração e manipulação visual dos dados. Ele apresenta alta conectividade a diferentes fontes de dados e recursos que permitem combinar e  visualizar os resultados usando as melhores práticas em
percepção visual. 

\subsection*{Pontos Fracos}

Apesar dos novos lançamentos de produtos de gerenciamento de dados que adicionaram capacidades administrativas,  percepções de recursos de governança fraca ainda persiste entre alguns dos clientes de referência do Tableau. 

\subsection*{Avaliação}

Nos três casos de uso desenvolvidos utilizando dados do \relGCC, a Salesforce aparece entre os 3 primeiros fornecedores em 2 dos 3 cenários de casos de uso. A Tableau, por sua vez, figura entre os 3 primeiros em todos os 3 casos de uso elaborados. Já em relação aos casos de uso utilizando dados do \relFCM, a Salesforce nem sequer foi incluída entre os 13 fornecedores avaliados. Porém o Tableau foi e aparece como um fornecedor potencial em 2 dos 3 casos de uso elaborados.

Apesar da incerteza criada por causa da aquisição da Tableau pela Salesforce, os resultados individuais de cada uma gera expectativa de que o Tableau torne-se uma potencial competidor no mercado de plataformas de BI.