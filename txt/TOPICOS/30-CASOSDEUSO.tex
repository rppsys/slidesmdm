\part{Casos de Uso}
\label{parte-estudosdecaso}

%\chapterimage{casosdeuso.pdf}
%\chapter{Casos de Uso - Considerações}


%\chapterimage{casosdeuso3.pdf}
\chapter{Casos de Uso usando \emph{Gartner}}
\label{cap-casos-gartner}

As diversas organizações de pesquisa descritas na seção \ref{sec-relatorios} fornecem uma forma de avaliar os fornecedores através da criação de casos de uso. Muitas vezes a metodologia de criação dos casos de uso são apresentadas juntamente com as tabelas de pontuação contendo as notas (\emph{scores}) de cada fornecedor para cada item avaliado. Dessa forma, torna-se possível criar casos de uso personalizados. Neste capítulo e nos próximos, vamos explorar justamente alguns casos de uso personalizados criados com base nos relatórios disponíveis.


% Aqui falo do critério de seleção. Isso tem que ficar em algum lugar.
\section{Critério de Seleção}
\label{sec-criteriog}

Para selecionar os fornecedores precisamos definir um critério justo de seleção e, portanto, de eliminação. Dessa forma, foi definido os seguintes critérios:

\begin{enumerate}
    \item Os \emph{scores} são arredondados para apenas 1 casa decimal;  
    \item O \emph{score} deve ser igual ou superior a $4,0$;
    \item Os fornecedores selecionados são aqueles que obtiverem as 3 (três) maiores notas dentro do conjunto de notas daquele cenário específico; 
\end{enumerate}

Conforme foi descrito, o \relGCC \xspace fornece uma tabela contendo a pontuação de cada item avaliado para cada um dos 20 fornecedores considerados. Dessa forma, cada área crítica de capacidade apresentadas na tabela \ref{tab:moscow} da seção \ref{sec-avaliacao} são pontuadas numa escala de 1 a 5. Uma pontuação igual a 1 equivale a ``fraco'', ou seja, a maioria ou todos os requisitos definidos não são alcançados, enquanto 5 equivale a 
``excelente'', ou seja, o item excede significativamente os requisitos \cite{gartner:criticalcapabilities}. A tabela contendo esses \emph{scores} é apresentada no Anexo \ref{anexo-tabelacc}.

Já a tabela \ref{tab:moscowordered} apresenta as 15 áreas de capacidade agrupadas pelas classes sugeridas. Esse agrupamento será importante para determinar os pesos para gerar o estudo de caso personalizado a partir da pontuação aferida pelo \emph{Grupo Gartner}.

\begin{table}[!h]
    \begin{center}
    \begin{tabular}{|p{0.4\textwidth}|c|}
        \hline
            \rowcolor{cldfB1} \multicolumn{2}{|c|}{\Large Classes MoSCoW \emph{Gartner} \normalsize} \\ \hline \hline
            % CABEÇALHO        
            \rowcolor{lightgray}\textbf{Áreas de Capacidade} & \textbf{Classe MoSCoW} \\ \hline
            % CONTEÚDO
            % Código gerado pela tabela do Google SpreadSheets
            \rowcolor{corMUST!80}Segurança & MUST \\ \hline
            \rowcolor{corMUST!80}Capacidade de Gerenciamento & MUST \\ \hline
            \rowcolor{corMUST!80}Conectividade de Fontes de Dados & MUST \\ \hline
            \rowcolor{corMUST!80}Preparação de Dados & MUST \\ \hline
            \rowcolor{corMUST!80}Visualização de Dados & MUST \\ \hline
            \rowcolor{corSHOULD!80}Nuvem & SHOULD \\ \hline
            \rowcolor{corSHOULD!80}Complexidade de Modelos & SHOULD \\ \hline
            \rowcolor{corSHOULD!80}Catálogos & SHOULD \\ \hline
            \rowcolor{corSHOULD!80}Incorporação de Análises & SHOULD \\ \hline
            \rowcolor{corCOULD!50}Geração de Relatórios & COULD \\ \hline
            \rowcolor{corCOULD!50}Analytics Avançados & COULD \\ \hline
            \rowcolor{corCOULD!50}Data Storytelling & COULD \\ \hline
            \rowcolor{corWOULD!50}Insight Automatizados & WOULD \\ \hline
            \rowcolor{corWOULD!50}Consulta em Linguagem Natural & WOULD \\ \hline
            \rowcolor{corWOULD!50}Geração de Linguagem Natural & WOULD \\ \hline
    \end{tabular}    
    \caption{\label{tab:moscowordered} Classificações MoSCoW para as Áreas de Capacidade}
    \end{center}
\end{table}

% Cenário Gartner A 
% Cenário Gartner A
\newcommand{\cenGA}{Cenário \emph{Gartner} A: Distribuição Equivalente} 
\section{\cenGA}
\label{sec-cenga}
    O primeiro cenário criado é equivalente ao cenário de caso de uso ``General Analytics'' na qual o Grupo \emph{Gartner} distribui pesos de forma ``equivalente'' entre as 15 áreas de capacidade. Isto porque ao se distribuir 100\% por 15 áreas temos $\frac{100}{15} = 6,66$ e assim na realidade o \emph{Gartner} atribui peso de 6\% para alguns e peso de 7\% para outros itens.
    
\subsection*{Distribuição dos Pesos}    
    
    Aqui faremos algo semelhante. A diferença é que vamos atribuir 7\% para áreas críticas de capacidade \MUST e \SHOULD e 6\% para as demais. Fazendo isso a soma dos pesos daria 99\% daí este 1\% restante é adicionado à área crítica de capacidade de Geração de Relatórios. A tabela \ref{tab:cenGA:pesos} apresenta como esses pesos foram distribuídos para cada uma das áreas críticas de capacidade.

    % cenGA - Tabela de Pesos
    \begin{table}[!h]
        \begin{center}
        \begin{tabular}{|p{0.4\textwidth}|c|c|}
            \hline
                % NOME DA TABELA        
                \rowcolor{cldfB1} \multicolumn{3}{|c|}{\Large \cenGA} \\  
                \rowcolor{cldfB1}
                \multicolumn{3}{|c|}{\large \textbf{Tabela de Pesos}} \\ \hline \hline
                % CABEÇALHO        
                \rowcolor{lightgray}\textbf{Áreas de Capacidade} & \textbf{Classe MoSCoW} & \textbf{Pesos} \\ \hline
                % CONTEÚDO
                % Código gerado pela tabela do Google SpreadSheets Cenário GA
                \rowcolor{corMUST!80}Segurança & MUST & 7\% \\ \hline
                \rowcolor{corMUST!80}Capacidade de Gerenciamento & MUST & 7\% \\ \hline
                \rowcolor{corMUST!80}Conectividade de Fontes de Dados & MUST & 7\% \\ \hline
                \rowcolor{corMUST!80}Preparação de Dados & MUST & 7\% \\ \hline
                \rowcolor{corMUST!80}Visualização de Dados & MUST & 7\% \\ \hline
                \rowcolor{corSHOULD!80}Nuvem & SHOULD & 7\% \\ \hline
                \rowcolor{corSHOULD!80}Complexidade de Modelos & SHOULD & 7\% \\ \hline
                \rowcolor{corSHOULD!80}Catálogos & SHOULD & 7\% \\ \hline
                \rowcolor{corSHOULD!80}Incorporação de Análises & SHOULD & 7\% \\ \hline
                \rowcolor{corCOULD!50}Geração de Relatórios & COULD & 7\% \\ \hline
                \rowcolor{corCOULD!50}Analytics Avançados & COULD & 6\% \\ \hline
                \rowcolor{corCOULD!50}Data Storytelling & COULD & 6\% \\ \hline
                \rowcolor{corWOULD!50}Insight Automatizados & WOULD & 6\% \\ \hline
                \rowcolor{corWOULD!50}Consulta em Linguagem Natural & WOULD & 6\% \\ \hline
                \rowcolor{corWOULD!50}Geração de Linguagem Natural & WOULD & 6\% \\ \hline
                % TOTAL
                \rowcolor{lightgray!30} \multicolumn{2}{|r|}{\large Total: \normalsize} & 100\% \\ \hline 
        \end{tabular}    
        \caption{\label{tab:cenGA:pesos} Pesos para \cenGA}
        \end{center}
    \end{table}
       
\subsection*{Resultados}    

    Ao multiplicar os pesos da tabela \ref{tab:cenGA:pesos} aos \emph{scores} da tabela apresentada no Anexo \ref{anexo-tabelacc} encontramos os resultados exibidos na tabela \ref{tab:cenGA:resultados}.

    % cenGA - Tabela de Resultados
    \begin{table}[!h]
        \begin{center}
        \begin{tabular}{|c|cc|}
            \hline
                % NOME DA TABELA        
                \rowcolor{cldfB1} \multicolumn{3}{|c|}{\Large \cenGA} \\  
                \rowcolor{cldfB1}
                \multicolumn{3}{|c|}{\large \textbf{Resultados}} \\ \hline \hline
                % CABEÇALHO        
                \rowcolor{lightgray}\textbf{Fornecedor} & \multicolumn{2}{c|}{\textbf{\emph{Score} [1-5]}} \\ \hline
                % CONTEÚDO
                % Código gerado pela tabela do Google SpreadSheets Cenário GA
                \rowcolor{corP1!80}Microsoft & \progressbar{0.856} & 4,3 \\ \hline
                \rowcolor{corP1!80}MicroStrategy & \progressbar{0.857} & 4,3 \\ \hline
                \rowcolor{corP2!50}Salesforce & \progressbar{0.843} & 4,2 \\ \hline
                \rowcolor{corP2!50}TIBCO Software & \progressbar{0.838} & 4,2 \\ \hline
                \rowcolor{corP2!50}Yellowfin & \progressbar{0.84} & 4,2 \\ \hline
                \rowcolor{corP3!30}Qlik & \progressbar{0.818} & 4,1 \\ \hline
                \rowcolor{corP3!30}Tableau & \progressbar{0.823} & 4,1 \\ \hline
                \rowcolor{corPF!20}Domo & \progressbar{0.804} & 4,0 \\ \hline
                \rowcolor{corPF!20}Oracle & \progressbar{0.806} & 4,0 \\ \hline
                \rowcolor{corPF!20}SAS & \progressbar{0.802} & 4,0 \\ \hline
                \rowcolor{corPF!20}Sisence & \progressbar{0.805} & 4,0 \\ \hline
                \rowcolor{corPF!20}ThoughtSpot & \progressbar{0.791} & 4,0 \\ \hline
                \rowcolor{corPF!20}IBM & \progressbar{0.752} & 3,8 \\ \hline
                \rowcolor{corPF!20}Infor Builders & \progressbar{0.756} & 3,8 \\ \hline
                \rowcolor{corPF!20}Birst & \progressbar{0.74} & 3,7 \\ \hline
                \rowcolor{corPF!20}Board & \progressbar{0.714} & 3,6 \\ \hline
                \rowcolor{corPF!20}SAP & \progressbar{0.72} & 3,6 \\ \hline
                \rowcolor{corPF!20}Logi & \progressbar{0.705} & 3,5 \\ \hline
                \rowcolor{corPF!20}Looker & \progressbar{0.657} & 3,3 \\ \hline
                \rowcolor{corPF!20}Alibaba & \progressbar{0.602} & 3,0 \\ \hline
        \end{tabular}    
        \caption{\label{tab:cenGA:resultados} Resultados para \cenGA}
        \end{center}
    \end{table}

\subsection*{Análise dos Resultados}    

    Dessa forma, analisando os resultados obtidos nesta tabela \ref{tab:cenGA:resultados} podemos verificar o seguinte:
    
    \begin{itemize}
        \item A \emph{Microsoft} e a \emph{MicroStrategy} empatam em primeiro lugar com $4,3$ pontos;
        \item A \emph{Salesforce}, \emph{TIBCO Software} e \emph{Yellowfin} empatam em segundo lugar com $4,2$ pontos cada;
        \item A \emph{Qlik} e \emph{Tableau} aparecem em terceiro lugar com $4,1$ pontos cada;
        \item Todos os demais fornecedores são eliminados;
    \end{itemize}
    
    O \cenGA \xspace apresenta fornecedores que podem ser considerados em um caso genérico onde todas as áreas apresentam mesma importância. Contudo, o caso da CLDF exige uma distribuição de pesos mais específica.

% Cenário Gartner A 
% Cenário Gartner B
\newcommand{\cenGB}{Cenário \emph{Gartner} B: Distribuição \emph{MoSCoW} \emph{MUST-Only}} 
\section{\cenGB}

    O segundo cenário criado já passa a levar as classes \emph{MoSCoW} sugeridas na seção \ref{sec-avaliacao} em conta para distribuição dos pesos.
    
\subsection*{Distribuição dos Pesos}    


    Dessa vez decidimos criar uma distribuição \emph{MoSCoW} ``\emph{MUST-Only}'', ou seja, o total de 100\% dos pesos será distribuído de forma igual apenas entre as áreas de capacidade classificadas como \MUST. As demais áreas receberão peso nulo.
    
    Assim, como temos 5 áreas críticas de capacidade classificadas como \MUST cada uma receberá peso de 20\% conforme pode ser verificado na tabela \ref{tab:cenGB:pesos}.
    
    % cenGB - Tabela de Pesos
    \begin{table}[!h]
        \begin{center}
        \begin{tabular}{|p{0.4\textwidth}|c|c|}
            \hline
                % NOME DA TABELA        
                \rowcolor{cldfB1} \multicolumn{3}{|c|}{\Large \cenGB} \\  
                \rowcolor{cldfB1}
                \multicolumn{3}{|c|}{\large \textbf{Tabela de Pesos}} \\ \hline \hline
                % CABEÇALHO        
                \rowcolor{lightgray}\textbf{Áreas de Capacidade} & \textbf{Classe MoSCoW} & \textbf{Pesos} \\ \hline
                % CONTEÚDO
                % Código gerado pela tabela do Google SpreadSheet Cenário GB
                \rowcolor{corMUST!80}Segurança & MUST & 20\% \\ \hline
                \rowcolor{corMUST!80}Capacidade de Gerenciamento & MUST & 20\% \\ \hline
                \rowcolor{corMUST!80}Conectividade de Fontes de Dados & MUST & 20\% \\ \hline
                \rowcolor{corMUST!80}Preparação de Dados & MUST & 20\% \\ \hline
                \rowcolor{corMUST!80}Visualização de Dados & MUST & 20\% \\ \hline
                \rowcolor{corSHOULD!80}Nuvem & SHOULD & 0\% \\ \hline
                \rowcolor{corSHOULD!80}Complexidade de Modelos & SHOULD & 0\% \\ \hline
                \rowcolor{corSHOULD!80}Catálogos & SHOULD & 0\% \\ \hline
                \rowcolor{corSHOULD!80}Incorporação de Análises & SHOULD & 0\% \\ \hline
                \rowcolor{corCOULD!50}Geração de Relatórios & COULD & 0\% \\ \hline
                \rowcolor{corCOULD!50}Analytics Avançados & COULD & 0\% \\ \hline
                \rowcolor{corCOULD!50}Data Storytelling & COULD & 0\% \\ \hline
                \rowcolor{corWOULD!50}Insight Automatizados & WOULD & 0\% \\ \hline
                \rowcolor{corWOULD!50}Consulta em Linguagem Natural & WOULD & 0\% \\ \hline
                \rowcolor{corWOULD!50}Geração de Linguagem Natural & WOULD & 0\% \\ \hline
                % TOTAL
                \rowcolor{lightgray!30} \multicolumn{2}{|r|}{\large Total: \normalsize} & 100\% \\ \hline 
        \end{tabular}    
        \caption{\label{tab:cenGB:pesos} Pesos para \cenGB}
        \end{center}
    \end{table}    

\subsection*{Resultados}   

    De maneira semelhante, ao multiplicar os pesos da tabela \ref{tab:cenGB:pesos} aos \emph{scores} da tabela apresentada no Anexo \ref{anexo-tabelacc} encontramos os resultados exibidos na tabela \ref{tab:cenGB:resultados}.
    
    % cenGB - Tabela de Resultados
    \begin{table}[!h]
        \begin{center}
        \begin{tabular}{|c|cc|}
            \hline
                % NOME DA TABELA        
                \rowcolor{cldfB1} \multicolumn{3}{|c|}{\Large \cenGB} \\  
                \rowcolor{cldfB1}
                \multicolumn{3}{|c|}{\large \textbf{Resultados}} \\ \hline \hline
                % CABEÇALHO        
                \rowcolor{lightgray}\textbf{Fornecedor} & \multicolumn{2}{c|}{\textbf{\emph{Score} [1-5]}} \\ \hline
                % CONTEÚDO
                % Código gerado pela tabela do Google SpreadSheets Cenário GB
                \rowcolor{corP1!80}MicroStrategy & \progressbar{0.928} & 4,6 \\ \hline
                \rowcolor{corP2!50}Yellowfin & \progressbar{0.892} & 4,5 \\ \hline
                \rowcolor{corP3!30}Tableau & \progressbar{0.884} & 4,4 \\ \hline
                \rowcolor{corP3!30}TIBCO Software & \progressbar{0.88} & 4,4 \\ \hline
                \rowcolor{corPF!20}Domo & \progressbar{0.868} & 4,3 \\ \hline
                \rowcolor{corPF!20}Microsoft & \progressbar{0.856} & 4,3 \\ \hline
                \rowcolor{corPF!20}Qlik & \progressbar{0.868} & 4,3 \\ \hline
                \rowcolor{corPF!20}Salesforce & \progressbar{0.852} & 4,3 \\ \hline
                \rowcolor{corPF!20}Sisence & \progressbar{0.864} & 4,3 \\ \hline
                \rowcolor{corPF!20}Birst & \progressbar{0.832} & 4,2 \\ \hline
                \rowcolor{corPF!20}Infor Builders & \progressbar{0.844} & 4,2 \\ \hline
                \rowcolor{corPF!20}SAS & \progressbar{0.84} & 4,2 \\ \hline
                \rowcolor{corPF!20}Logi & \progressbar{0.812} & 4,1 \\ \hline
                \rowcolor{corPF!20}Oracle & \progressbar{0.796} & 4,0 \\ \hline
                \rowcolor{corPF!20}Board & \progressbar{0.744} & 3,7 \\ \hline
                \rowcolor{corPF!20}ThoughtSpot & \progressbar{0.74} & 3,7 \\ \hline
                \rowcolor{corPF!20}IBM & \progressbar{0.724} & 3,6 \\ \hline
                \rowcolor{corPF!20}Alibaba & \progressbar{0.692} & 3,5 \\ \hline
                \rowcolor{corPF!20}Looker & \progressbar{0.7} & 3,5 \\ \hline
                \rowcolor{corPF!20}SAP & \progressbar{0.692} & 3,5 \\ \hline
        \end{tabular}    
        \caption{\label{tab:cenGB:resultados} Resultados para \cenGB}
        \end{center}
    \end{table}    
    
\subsection*{Análise dos Resultados}    

    Desta vez, ao atribuir 100\% do total dos pesos de forma igual entre as áreas de capacidade classificadas como \MUST, nosso critério de escolha definido na seção \ref{sec-criteriog} exibe os seguintes resultados: 
    
    \begin{itemize}
        \item A \emph{MicroStrategy} lidera em primeiro lugar com $4,6$ pontos;
        \item A \emph{Yellowfin} aparece sozinha em segundo lugar com $4,5$ ponto;
        \item A \emph{Tableau} e a \emph{TIBCO Software} empatam em terceiro lugar com $4,4$ pontos cada;
        \item Todas os demais fornecedores são eliminados;
    \end{itemize}
    
    O \cenGB \xspace apresenta potenciais fornecedores para atender as necessidades mais imediatas da \CLDF de modernização da fiscalização.
    
    Contudo, ainda não ficamos satisfeitos com este cenário pois, mesmo que as áreas críticas de capacidade \SHOULD não sejam críticas, seu grau de importância não pode ser desprezado.

% Cenário Gartner A 
% Cenário Gartner C
\newcommand{\cenGC}{Cenário \emph{Gartner} C: Distribuição \emph{MoSCoW} 70 25 5} 
\section{\cenGC}
\label{sec-cengc}

    O terceiro e último cenário criado, com base nas avaliações da \emph{Gartner}, para representar as necessidades de modernização da fiscalização da \CLDF leva em consideração a classificação \emph{MoSCoW} distribuindo o peso total de 100\% entre as classes \MUST, \SHOULD, \COULD e \WOULD. 
    

\subsection*{Distribuição dos Pesos}    
\label{sub-cengc-pesos}

    Neste cenário, essa divisão de pesos entre essas categorias de classes não poderia ser igual, ou seja, não podemos atribuir, por exemplo, 34\%  para MUST, 33\% para SHOULD e 33\% para COULD porque se fizéssemos isto, estaríamos dando o mesmo nível de importância a cada uma dessas categorias de classes \emph{MoSCoW}.
    
    Dessa forma, foi escolhido a seguinte divisão de pesos para cada uma dessas categorias de classe:
    
    \begin{itemize}
        \item 70\% do total do peso distribuído entre áreas críticas de capacidade classificadas como \MUST;
        \item 25\% do total do peso distribuído entre áreas críticas de capacidade classificadas como \SHOULD;
        \item 5\% do total do peso distribuído entre áreas críticas de capacidade classificadas como \COULD;
        \item Áreas críticas de capacidade classificadas como \WOULD recebem peso nulo (0\%);
    \end{itemize}
    
    Pretende-se, desta forma, dividir os pesos entre as quatro categorias de classes mantendo a coerência com o nível de importância que cada uma representa. Assim, a tabela \ref{tab:cenGC:pesos} apresenta os pesos percentuais atribuídos para cada área crítica de capacidade, de acordo com sua respectiva classe \emph{MoSCoW}.

    % cenGC - Tabela de Pesos
    \begin{table}[!h]
        \begin{center}
        \begin{tabular}{|p{0.4\textwidth}|c|c|}
            \hline
                % NOME DA TABELA        
                \rowcolor{cldfB1} \multicolumn{3}{|c|}{\Large \cenGC} \\  
                \rowcolor{cldfB1}
                \multicolumn{3}{|c|}{\large \textbf{Tabela de Pesos}} \\ \hline \hline
                % CABEÇALHO        
                \rowcolor{lightgray}\textbf{Áreas de Capacidade} & \textbf{Classe MoSCoW} & \textbf{Pesos} \\ \hline
                % CONTEÚDO
                % Código gerado pela tabela do Google SpreadSheet Cenário GB
                % MUST
                \rowcolor{corMUST!80}Segurança & MUST & 14\% \\ \hline
                \rowcolor{corMUST!80}Capacidade de Gerenciamento & MUST & 14\% \\ \hline
                \rowcolor{corMUST!80}Conectividade de Fontes de Dados & MUST & 14\% \\ \hline
                \rowcolor{corMUST!80}Preparação de Dados & MUST & 14\% \\ \hline
                \rowcolor{corMUST!80}Visualização de Dados & MUST & 14\% \\ \hline
                \rowcolor{corMUST!50!lightgray} \multicolumn{2}{|r|}{\large Total MUST: \normalsize} & 70\% \\ \hline 
                % SHOULD
                \rowcolor{corSHOULD!80}Nuvem & SHOULD & 7\% \\ \hline
                \rowcolor{corSHOULD!80}Complexidade de Modelos & SHOULD & 6\% \\ \hline
                \rowcolor{corSHOULD!80}Catálogos & SHOULD & 6\% \\ \hline
                \rowcolor{corSHOULD!80}Incorporação de Análises & SHOULD & 6\% \\ \hline
                \rowcolor{corSHOULD!30!lightgray} \multicolumn{2}{|r|}{\large Total SHOULD: \normalsize} & 25\% \\ \hline 
                % COULD
                \rowcolor{corCOULD!50}Geração de Relatórios & COULD & 2\% \\ \hline
                \rowcolor{corCOULD!50}Analytics Avançados & COULD & 1\% \\ \hline
                \rowcolor{corCOULD!50}Data Storytelling & COULD & 2\% \\ \hline
                \rowcolor{corCOULD!30!lightgray} \multicolumn{2}{|r|}{\large Total COULD: \normalsize} & 5\% \\ \hline 
                % WOULD
                \rowcolor{corWOULD!50}Insight Automatizados & WOULD & 0\% \\ \hline
                \rowcolor{corWOULD!50}Consulta em Linguagem Natural & WOULD & 0\% \\ \hline
                \rowcolor{corWOULD!50}Geração de Linguagem Natural & WOULD & 0\% \\ \hline
                \rowcolor{corWOULD!30!lightgray} \multicolumn{2}{|r|}{\large Total WOULD: \normalsize} & 0\% \\ \hline 
                % TOTAL
                \rowcolor{lightgray!30} \multicolumn{2}{|r|}{\large \textbf{Total Geral}: \normalsize} & 100\% \\ \hline 
        \end{tabular}    
        \caption{\label{tab:cenGC:pesos} Pesos para \cenGC}
        \end{center}
    \end{table}   
  
    Note que o total de 70\% de peso percentual definido para a categoria de classe \MUST foi dividida entre as 5 áreas desta classe de forma igualitária: $\frac{70\%}{5} = 14\%$. 
    
    Já a divisão inteira do total de 25\% entre as 4 áreas \SHOULD não é possível então elegeu-se a Capacidade de Suporte à Nuvem para receber o 1\% adicional: $\frac{25\%}{4} = 6\%$ e resta 1\%. 
    
    Em seguida, os 5\% restantes foram divididos entre as 3 áreas \COULD atribuindo-se 2\% para a área crítica de Geração de Relatórios e 2\% para a área de \emph{Data Storytelling} com o 1\% restante atribuído à área de \emph{Analytics} Avançadas. Verificou-se também que alternativas diferentes de distribuição final desses 5\% não altera os resultados.
    

\subsection*{Resultados}   

    Finalmente, ao multiplicar os pesos da tabela \ref{tab:cenGC:pesos} aos \emph{scores} da tabela apresentada no Anexo \ref{anexo-tabelacc} encontramos os resultados exibidos na tabela \ref{tab:cenGC:resultados}.
    
    % cenGC - Tabela de Resultados
    \begin{table}[!h]
        \begin{center}
        \begin{tabular}{|c|cc|}
            \hline
                % NOME DA TABELA        
                \rowcolor{cldfB1} \multicolumn{3}{|c|}{\Large \cenGC} \\  
                \rowcolor{cldfB1}
                \multicolumn{3}{|c|}{\large \textbf{Resultados}} \\ \hline \hline
                % CABEÇALHO        
                \rowcolor{lightgray}\textbf{Fornecedor} & \multicolumn{2}{c|}{\textbf{\emph{Score} [1-5]}} \\ \hline
                % CONTEÚDO
                % Código gerado pela tabela do Google SpreadSheets Cenário GC
                \rowcolor{corP1!80}MicroStrategy & \progressbar{0.92} & 4,6 \\ \hline
                \rowcolor{corP2!50}Domo & \progressbar{0.86} & 4,3 \\ \hline
                \rowcolor{corP2!50}Microsoft & \progressbar{0.86} & 4,3 \\ \hline
                \rowcolor{corP2!50}Qlik & \progressbar{0.86} & 4,3 \\ \hline
                \rowcolor{corP2!50}Tableau & \progressbar{0.86} & 4,3 \\ \hline
                \rowcolor{corP2!50}TIBCO Software & \progressbar{0.86} & 4,3 \\ \hline
                \rowcolor{corP2!50}Yellowfin & \progressbar{0.86} & 4,3 \\ \hline
                \rowcolor{corP3!30}Salesforce & \progressbar{0.84} & 4,2 \\ \hline
                \rowcolor{corP3!30}Sisence & \progressbar{0.84} & 4,2 \\ \hline
                \rowcolor{corPF!20}Birst & \progressbar{0.82} & 4,1 \\ \hline
                \rowcolor{corPF!20}Infor Builders & \progressbar{0.82} & 4,1 \\ \hline
                \rowcolor{corPF!20}SAS & \progressbar{0.82} & 4,1 \\ \hline
                \rowcolor{corPF!20}Oracle & \progressbar{0.8} & 4,0 \\ \hline
                \rowcolor{corPF!20}Logi & \progressbar{0.78} & 3,9 \\ \hline
                \rowcolor{corPF!20}Board & \progressbar{0.76} & 3,8 \\ \hline
                \rowcolor{corPF!20}ThoughtSpot & \progressbar{0.76} & 3,8 \\ \hline
                \rowcolor{corPF!20}Looker & \progressbar{0.74} & 3,7 \\ \hline
                \rowcolor{corPF!20}IBM & \progressbar{0.72} & 3,6 \\ \hline
                \rowcolor{corPF!20}SAP & \progressbar{0.68} & 3,4 \\ \hline
                \rowcolor{corPF!20}Alibaba & \progressbar{0.66} & 3,3 \\ \hline
            \end{tabular}    
        \caption{\label{tab:cenGC:resultados} Resultados para \cenGC}
        \end{center}
    \end{table}     


\subsection*{Análise dos Resultados} 

    Os resultados obtidos pelo \cenGC \xspace foram:
    
    \begin{itemize}
        \item A \emph{MicroStrategy} aparece em primeiro lugar com $4,6$ pontos;
        \item Em segundo lugar empataram a \emph{Domo}, \emph{Microsoft}, \emph{Qlik}, \emph{Tableau}, \emph{TIBCO Software} e \emph{Yellowfin} com $4,3$ pontos cada; 
        \item A \emph{Salesforce} e a \emph{Sisense} empataram em terceiro lugar com $4,2$ pontos cada;
        \item Todas os demais fornecedores são eliminados;
    \end{itemize}
    
    O \cenGC \xspace apresenta potenciais fornecedores de Plataformas de BI para satisfazer as necessidades identificadas de modernização da fiscalização da \CLDF.
    

% Conclusão
\section{Conclusões}
\label{sec-conclusoes-gartner}

Nesse capítulo construímos três cenários de casos de uso personalizados para as necessidades identificadas e avaliadas na ``\autoref{parte-necessidades} -- \nameref{parte-necessidades}''.

A análise de cada caso de uso permite concluir que, de acordo com o \relGCC, a MicroStrategy é certamente uma opção de fornecedora que atenderia muito bem as necessidades identificadas uma vez que ela aparece em primeiro lugar nos três casos de uso desenvolvidos. Por outro lado, as diferenças de pontuação entre a MicroStrategy e os demais fornecedores é pequeno e, portanto, os demais fornecedores não deixam de ser boas opções.

Dessa forma, apresentamos, \textbf{em ordem alfabética}, ou seja, sem predileção, os fornecedores que apareceram pelo menos uma vez entre quaisquer das três primeiras colocações nos casos de uso construídos.

\begin{env-destaque}{Fornecedores de destaque \emph{Gartner}:}
 \begin{multicols}{2}
    \begin{itemize}
        \item Domo
        \item Microsoft
        \item MicroStrategy
        \item Qlik
        \item Sisence
        \item Salesforce
        \item Tableau
        \item TIBCO Software
        \item Yellowfin
    \end{itemize}
 \end{multicols}
\end{env-destaque}

Os principais produtos destes fornecedores serão examinados com maior detalhe na ``\autoref{parte-plataformas} -- \nameref{parte-plataformas}''.

% #####################################################

\chapter{Casos de Uso usando \emph{Forrester}}
\label{cap-casos-forrester}

%Agora é a vez de criar alguns casos de uso usando os relatórios do \emh{Forrester Wave}.

Uma particularidade dos relatórios \emph{Forrester} é que existem, na verdade, dois relatórios desenvolvidos a partir de duas perspectivas diferentes: \emph{Vendor-Managed} e \emph{Client-Managed}. Tenta-se, portanto, dividir o mercado em dois grupos. O \relFVM  \xspace \cite{forrester:vendormanaged} avalia fornecedores que oferecem soluções na nuvem com custo zero de manutenção. Já o \relFCM \xspace \cite{forrester:clientmanaged} avalia fornecedores que oferecem soluções \emph{on-prem} que podem ser totalmente gerenciadas pelos próprios clientes \cite{forrester:vendorclient}. Dessa forma, as necessidades descritas e avaliadas no Capítulo \ref{cap-necessidades}, em particular a seção \ref{sub-cloud}, onde analisa-se a capacidade de oferecer suporte à nuvem, aponta no sentido de focar em soluções \emph{client-managed}. E assim, neste trabalho, os casos de uso usando as avaliações \emph{Forrester} foram realizadas com base na tabela de pontuação (\emph{scorecard}) do \relatorioFCM. Essas pontuações podem ser encontradas na tabela apresentada no Anexo \ref{anexo-tabelafw}. 

É importante evidenciar, também, algumas diferenças entre os relatórios do \emph{Forrester} e \emph{Gartner} e fazer algumas considerações sobre isso.

Em primeiro lugar, os elementos avaliados no \emph{Forrester} recebem o nome de \emph{key criteria}, isto é,``critérios chave''. Esses critérios são divididos em três grupos:``\emph{Current offering}'', ``\emph{Strategy}'' e ``\emph{Market presence}''. O primeiro grupo (\emph{Current offering}) posiciona o fornecedor no eixo vertical do \emph{Forrester Wave Graphic} indicando a força de sua ``oferta atual'', em tradução livre. Aqui são avaliadas áreas de capacidades semelhantes àquelas do Grupo \emph{Gartner}.
O segundo grupo (\emph{Strategy}) posiciona o fornecedor no eixo horizontal em relação à estratégia de vendas do fornecedor. E o terceiro grupo (\emph{Market presence}) representado pelo tamanho dos marcadores no gráfico, mede a receita de produtos e serviços de BI de cada um indicando, portanto, sua presença no mercado.  

Nos casos de uso desenvolvidos neste estudo, estamos interessados em avaliar os critérios qualitativos relativos às capacidades de recursos oferecidos por cada fornecedor. Portanto optamos por utilizar as avaliações para capacidades do grupo \emph{Current offering} desprezando as avaliações de critérios de estratégia comercial e presença de mercado.  

Em segundo lugar, cada instituição de pesquisa define seus próprios critérios de seleção de fornecedores que serão avaliados e comparados. Assim, enquanto o \emph{Gartner} avalia 20 fornecedores, o \emph{Forrester} compara apenas 13 deles.

Finalmente, as áreas de capacidades avaliadas pelo \emph{Forrester}\footnote{O \emph{Forrester Wave} avalia os fornecedores de modo análogo ao \emph{Gartner}. Assim, as áreas de capacidade \emph{Forrester} também são pontuadas numa escala que varia de 1 a 5.} não são exatamente as mesmas escolhidas pelo Grupo \emph{Gartner}. Portanto, para utilizar a avaliação \emph{MoSCoW} elaborada no Capítulo \ref{cap-necessidades} foi preciso fazer um ajuste. Esse ajuste pode ser visto na tabela \ref{tab:moscowordered-fw}. 



\begin{table}[!h]
    \begin{center}
    \begin{tabular}{|p{0.4\textwidth}|c|}
        \hline
            \rowcolor{cldfB1} \multicolumn{2}{|c|}{\Large Classes MoSCoW \emph{Forrester} \normalsize} \\ \hline \hline
            % CABEÇALHO        
            \rowcolor{lightgray}\textbf{Áreas de Capacidade} & \textbf{Classe MoSCoW} \\ \hline
            % CONTEÚDO
            % Código gerado pela tabela do Google SpreadSheets
            \rowcolor{corMUST!80}Segurança & MUST \\ \hline
            \rowcolor{corMUST!80}Big Data & MUST \\ \hline
            \rowcolor{corMUST!80}Preparação de Dados & MUST \\ \hline
            \rowcolor{corSHOULD!80}Arquitetura & SHOULD \\ \hline
            \rowcolor{corSHOULD!80}Interfaces Gráficas & SHOULD \\ \hline
            \rowcolor{corSHOULD!80}Mobile & SHOULD \\ \hline
            \rowcolor{corSHOULD!80}Opções de Implantação & SHOULD \\ \hline
            \rowcolor{corCOULD!50}Criação de Apps Personalizados & COULD \\ \hline
            \rowcolor{corCOULD!50}BI Avançado & COULD \\ \hline
            \rowcolor{corWOULD!50}Sistemas de Insight & WOULD \\ \hline
    \end{tabular}    
    \caption{\label{tab:moscowordered-fw} Classificações MoSCoW para as Áreas de Capacidade do \emph{Forrester}.}
    \end{center}
\end{table}


\section{Critério de Seleção}
\label{sec-criteriof}

Os critérios de seleção escolhidos são os mesmos da seção \ref{sec-criteriog} utilizados no capítulo \nameref{cap-casos-gartner}. Vamos repeti-los abaixo:

\begin{enumerate}
    \item Os \emph{scores} são arredondados para apenas 1 casa decimal;  
    \item O \emph{score} deve ser igual ou superior a $4,0$;
    \item Os fornecedores selecionados são aqueles que obtiverem as 3 (três) maiores notas dentro do conjunto de notas daquele cenário específico; 
\end{enumerate}







% Cenário Forrester A 
% Cenário Forrester A
\newcommand{\cenFA}{Cenário \emph{Forrester} A: Distribuição Equivalente} 
\section{\cenFA}
\label{sec-cenfa}

    O \cenFA \xspace despreza inicialmente as classes \emph{MoSCoW} e distribui pesos iguais para as 10 áreas avaliadas. A conta é fácil: $\frac{100\%}{15} = 10\%$.
    
\subsection*{Distribuição dos Pesos}    

    A tabela \ref{tab:cenFA:pesos} apresenta essa distribuição igual de pesos.

% cenFA - Tabela de Pesos
\begin{table}[!h]
    \begin{center}
    \begin{tabular}{|p{0.4\textwidth}|c|c|}
        \hline
            % NOME DA TABELA        
            \rowcolor{cldfB1} \multicolumn{3}{|c|}{\Large \cenFA} \\  
            \rowcolor{cldfB1}
            \multicolumn{3}{|c|}{\large \textbf{Tabela de Pesos}} \\ \hline \hline
            % CABEÇALHO        
            \rowcolor{lightgray}\textbf{Áreas de Capacidade} & \textbf{Classe MoSCoW} & \textbf{Pesos} \\ \hline
            % CONTEÚDO
            % Código gerado pela tabela do Google SpreadSheets Cenário cenFA
            \rowcolor{corMUST!80}Segurança & MUST & 10\% \\ \hline
            \rowcolor{corMUST!80}Big Data & MUST & 10\% \\ \hline
            \rowcolor{corMUST!80}Preparação de Dados & MUST & 10\% \\ \hline
            \rowcolor{corSHOULD!80}Arquitetura & SHOULD & 10\% \\ \hline
            \rowcolor{corSHOULD!80}Interfaces Gráficas & SHOULD & 10\% \\ \hline
            \rowcolor{corSHOULD!80}Mobile & SHOULD & 10\% \\ \hline
            \rowcolor{corSHOULD!80}Opções de Implantação & SHOULD & 10\% \\ \hline
            \rowcolor{corCOULD!50}Criação de Apps Personalizados & COULD & 10\% \\ \hline
            \rowcolor{corCOULD!50}BI Avançado & COULD & 10\% \\ \hline
            \rowcolor{corWOULD!50}Sistemas de Insight & WOULD & 10\% \\ \hline
            % TOTAL
            \rowcolor{lightgray!30} \multicolumn{2}{|r|}{\large Total: \normalsize} & 100\% \\ \hline 
    \end{tabular}    
    \caption{\label{tab:cenFA:pesos} Pesos para \cenFA}
    \end{center}
\end{table}

\subsection*{Resultados}  

    Ao multiplicar os pesos da tabela \ref{tab:cenFA:pesos} aos \emph{scores} da tabela apresentada no Anexo \ref{anexo-tabelafw} encontramos os resultados exibidos na tabela \ref{tab:cenFA:resultados}.

    % cenFA - Tabela de Resultados
    \begin{table}[!h]
        \begin{center}
        \begin{tabular}{|c|cc|}
            \hline
                % NOME DA TABELA        
                \rowcolor{cldfB1} \multicolumn{3}{|c|}{\Large \cenFA} \\  
                \rowcolor{cldfB1}
                \multicolumn{3}{|c|}{\large \textbf{Resultados}} \\ \hline \hline
                % CABEÇALHO        
                \rowcolor{lightgray}\textbf{Fornecedor} & \multicolumn{2}{c|}{\textbf{\emph{Score} [1-5]}} \\ \hline
                % CONTEÚDO
                % Código gerado pela tabela do Google SpreadSheets Cenário GB
                \rowcolor{corP1!80}MicroStrategy & \progressbar{0.88} & 4,4 \\ \hline
                \rowcolor{corP1!80}TIBCO Software & \progressbar{0.88} & 4,4 \\ \hline
                \rowcolor{corPF!20}Sisense & \progressbar{0.72} & 3,6 \\ \hline
                \rowcolor{corPF!20}Tableau Software & \progressbar{0.72} & 3,6 \\ \hline
                \rowcolor{corPF!20}Yellowfin & \progressbar{0.72} & 3,6 \\ \hline
                \rowcolor{corPF!20}Qlik & \progressbar{0.68} & 3,4 \\ \hline
                \rowcolor{corPF!20}SAS & \progressbar{0.68} & 3,4 \\ \hline
                \rowcolor{corPF!20}Birst & \progressbar{0.64} & 3,2 \\ \hline
                \rowcolor{corPF!20}Information Builders & \progressbar{0.64} & 3,2 \\ \hline
                \rowcolor{corPF!20}IBM & \progressbar{0.56} & 2,8 \\ \hline
                \rowcolor{corPF!20}Microsoft & \progressbar{0.56} & 2,8 \\ \hline
                \rowcolor{corPF!20}ThoughtSpot & \progressbar{0.56} & 2,8 \\ \hline
                \rowcolor{corPF!20}OpenText & \progressbar{0.44} & 2,2 \\ \hline
        \end{tabular}    
        \caption{\label{tab:cenFA:resultados} Resultados para \cenFA}
        \end{center}
    \end{table}


\subsection*{Análise dos Resultados} 

    Analisando os resultados obtidos nesta tabela \ref{tab:cenFA:resultados} podemos verificar o seguinte:
    
    \begin{itemize}
        \item A \emph{MicroStrategy} e a \emph{TIBCO Software} empatam em primeiro lugar com $4,4$ pontos;
        \item Todos os demais fornecedores pontuam abaixo de $4,0$ pontos e, portanto, de acordo com o critério de seleção, são eliminados;
    \end{itemize}
    
    Novamente, o \cenFA \xspace apresenta fornecedores que podem ser considerados em um caso genérico onde todas as áreas apresentam mesma importância. Em seguida, de forma parecida com o que foi realizado no capítulo anterior, vamos continuar as análises criando casos de uso com distribuições de pesos mais específicas para as necessidades avaliadas.
    
    

% Cenário Forrester B
% Cenário Forrester B
\newcommand{\cenFB}{Cenário \emph{Forrester} B: Distribuição \emph{MoSCoW} \emph{MUST-Only}} 
\section{\cenFB}
\label{sec-cenfb}

    O segundo cenário \emph{Forrester} volta a levar as classes \emph{MoSCoW} em conta.
    
\subsection*{Distribuição dos Pesos}    

    Dessa forma, o total de 100\% dos pesos é distribuído entre as áreas de capacidade classificadas como \MUST e as demais áreas recebem peso nulo. Assim, a Área de Capacidade de Segurança foi eleita para receber peso de 34\% e as outras duas receberam 33\% cada. Isso pode ser verificado na tabela \ref{tab:cenFB:pesos}.
    
% cenFB - Tabela de Pesos
\begin{table}[!h]
    \begin{center}
    \begin{tabular}{|p{0.4\textwidth}|c|c|}
        \hline
            % NOME DA TABELA        
            \rowcolor{cldfB1} \multicolumn{3}{|c|}{\Large \cenFB} \\  
            \rowcolor{cldfB1}
            \multicolumn{3}{|c|}{\large \textbf{Tabela de Pesos}} \\ \hline \hline
            % CABEÇALHO        
            \rowcolor{lightgray}\textbf{Áreas de Capacidade} & \textbf{Classe MoSCoW} & \textbf{Pesos} \\ \hline
            % CONTEÚDO
            % Código gerado pela tabela do Google SpreadSheets Cenário cenFB
            \rowcolor{corMUST!80}Segurança & MUST & 34\% \\ \hline
            \rowcolor{corMUST!80}Big Data & MUST & 33\% \\ \hline
            \rowcolor{corMUST!80}Preparação de Dados & MUST & 33\% \\ \hline
            \rowcolor{corSHOULD!80}Arquitetura & SHOULD & 0\% \\ \hline
            \rowcolor{corSHOULD!80}Interfaces Gráficas & SHOULD & 0\% \\ \hline
            \rowcolor{corSHOULD!80}Mobile & SHOULD & 0\% \\ \hline
            \rowcolor{corSHOULD!80}Opções de Implantação & SHOULD & 0\% \\ \hline
            \rowcolor{corCOULD!50}Criação de Apps Personalizados & COULD & 0\% \\ \hline
            \rowcolor{corCOULD!50}BI Avançado & COULD & 0\% \\ \hline
            \rowcolor{corWOULD!50}Sistemas de Insight & WOULD & 0\% \\ \hline
            % TOTAL
            \rowcolor{lightgray!30} \multicolumn{2}{|r|}{\large Total: \normalsize} & 100\% \\ \hline 
    \end{tabular}    
    \caption{\label{tab:cenFB:pesos} Pesos para \cenFB}
    \end{center}
\end{table}

\subsection*{Resultados}  

    De maneira semelhante, ao multiplicar os pesos da tabela \ref{tab:cenFB:pesos} aos \emph{scores} da tabela apresentada no Anexo \ref{anexo-tabelafw} encontramos os resultados exibidos na tabela \ref{tab:cenFB:resultados}.


    % cenFB - Tabela de Resultados
    \begin{table}[!h]
        \begin{center}
        \begin{tabular}{|c|cc|}
            \hline
                % NOME DA TABELA        
                \rowcolor{cldfB1} \multicolumn{3}{|c|}{\Large \cenFB} \\  
                \rowcolor{cldfB1}
                \multicolumn{3}{|c|}{\large \textbf{Resultados}} \\ \hline \hline
                % CABEÇALHO        
                \rowcolor{lightgray}\textbf{Fornecedor} & \multicolumn{2}{c|}{\textbf{\emph{Score} [1-5]}} \\ \hline
                % CONTEÚDO
                % Código gerado pela tabela do Google SpreadSheets Cenário GB
                \rowcolor{corP1!80}Birst & \progressbar{0.86} & 4,3 \\ \hline
                \rowcolor{corP1!80}Microsoft & \progressbar{0.86} & 4,3 \\ \hline
                \rowcolor{corP1!80}Tableau Software & \progressbar{0.86} & 4,3 \\ \hline
                \rowcolor{corP1!80}TIBCO Software & \progressbar{0.86} & 4,3 \\ \hline
                \rowcolor{corP1!80}Yellowfin & \progressbar{0.86} & 4,3 \\ \hline
                \rowcolor{corPF!20}Information Builders & \progressbar{0.74} & 3,7 \\ \hline
                \rowcolor{corPF!20}SAS & \progressbar{0.74} & 3,7 \\ \hline
                \rowcolor{corPF!20}Sisense & \progressbar{0.74} & 3,7 \\ \hline
                \rowcolor{corPF!20}MicroStrategy & \progressbar{0.72} & 3,6 \\ \hline
                \rowcolor{corPF!20}OpenText & \progressbar{0.6} & 3,0 \\ \hline
                \rowcolor{corPF!20}Qlik & \progressbar{0.6} & 3,0 \\ \hline
                \rowcolor{corPF!20}IBM & \progressbar{0.48} & 2,4 \\ \hline
                \rowcolor{corPF!20}ThoughtSpot & \progressbar{0.34} & 1,7 \\ \hline
        \end{tabular}    
        \caption{\label{tab:cenFB:resultados} Resultados para \cenFB}
        \end{center}
    \end{table}


\subsection*{Análise dos Resultados} 

    Outra vez, ao atribuir 100\% do total dos pesos de forma igual entre as áreas de capacidade classificadas como \MUST, o critério de escolha definido na seção \ref{sec-criteriof} exibe os seguintes resultados apresentados na tabela \ref{tab:cenFB:resultados}: 
    \begin{itemize}
        \item Em primeiro lugar empatam com $4,4$ pontos os seguintes fornecedores: \emph{BIRST}, \emph{Microsoft}, \emph{Tableau Software}, \emph{TIBCO Software} e \emph{Yellowfin}. 
        \item Todos os demais fornecedores pontuam abaixo de $4,0$ pontos e são eliminados;
    \end{itemize}
    
    Como ocorreu no \cenGB, o \cenFB \xspace não satisfaz pois é preciso considerar as demais áreas.

% Cenário Forrester C 
% Cenário Forrester C
\newcommand{\cenFC}{Cenário \emph{Forrester} C: Distribuição \emph{MoSCoW} 70 25 5} 
\section{\cenFC}
\label{sec-cenfc}

    Finalmente, o último cenário \emph{Forrester} criado para representar as necessidades de modernização da fiscalização da \CLDF leva em consideração a classificação \emph{MoSCoW} distribuindo o peso total de 100\% entre as classes \MUST, \SHOULD, \COULD e \WOULD. 

\subsection*{Distribuição dos Pesos}    

    Assim, a distribuição dos pesos é a mesma daquela utilizada no \cenGC \xspace (seção \ref{sub-cengc-pesos}), ou seja, 70\% do total distribuído entre áreas \MUST, 25\% para áreas \SHOULD e 5\% para áreas \COULD. Áreas \WOULD ficam com peso nulo (0\%). A tabela \ref{tab:cenFC:pesos} apresenta os pesos percentuais atribuídos para cada área de acordo com sua respectiva classe \emph{MoSCoW}.

% cenFC - Tabela de Pesos
\begin{table}[!h]
    \begin{center}
    \begin{tabular}{|p{0.4\textwidth}|c|c|}
        \hline
            % NOME DA TABELA        
            \rowcolor{cldfB1} \multicolumn{3}{|c|}{\Large \cenFC} \\  
            \rowcolor{cldfB1}
            \multicolumn{3}{|c|}{\large \textbf{Tabela de Pesos}} \\ \hline \hline
            % CABEÇALHO        
            \rowcolor{lightgray}\textbf{Áreas de Capacidade} & \textbf{Classe MoSCoW} & \textbf{Pesos} \\ \hline
            % CONTEÚDO
            % Código gerado pela tabela do Google SpreadSheets Cenário cenFC
            % MUST
            \rowcolor{corMUST!80}Segurança & MUST & 24\% \\ \hline
            \rowcolor{corMUST!80}Big Data & MUST & 23\% \\ \hline
            \rowcolor{corMUST!80}Preparação de Dados & MUST & 23\% \\ \hline
            \rowcolor{corMUST!50!lightgray} \multicolumn{2}{|r|}{\large Total MUST: \normalsize} & 70\% \\ \hline 
            % SHOULD
            \rowcolor{corSHOULD!80}Arquitetura & SHOULD & 7\% \\ \hline
            \rowcolor{corSHOULD!80}Interfaces Gráficas & SHOULD & 6\% \\ \hline
            \rowcolor{corSHOULD!80}Mobile & SHOULD & 6\% \\ \hline
            \rowcolor{corSHOULD!80}Opções de Implantação & SHOULD & 6\% \\ \hline
            \rowcolor{corSHOULD!30!lightgray} \multicolumn{2}{|r|}{\large Total SHOULD: \normalsize} & 25\% \\ \hline 
            % COULD
            \rowcolor{corCOULD!50}Criação de Apps Personalizados & COULD & 3\% \\ \hline
            \rowcolor{corCOULD!50}BI Avançado & COULD & 2\% \\ \hline
            \rowcolor{corCOULD!30!lightgray} \multicolumn{2}{|r|}{\large Total COULD: \normalsize} & 5\% \\ \hline 
            % WOULD
            \rowcolor{corWOULD!50}Sistemas de Insight & WOULD & 0\% \\ \hline
            \rowcolor{corWOULD!30!lightgray} \multicolumn{2}{|r|}{\large Total WOULD: \normalsize} & 0\% \\ \hline 
            % TOTAL
            \rowcolor{lightgray!30} \multicolumn{2}{|r|}{\large Total: \normalsize} & 100\% \\ \hline 
    \end{tabular}    
    \caption{\label{tab:cenFC:pesos} Pesos para \cenFC}
    \end{center}
\end{table}

\subsection*{Resultados}  

    Finalmente, ao multiplicar os pesos da tabela \ref{tab:cenFC:pesos} aos \emph{scores} da tabela apresentada no Anexo \ref{anexo-tabelafw} encontramos os resultados exibidos na tabela \ref{tab:cenFC:resultados}.

    % cenFC - Tabela de Resultados
    \begin{table}[!h]
        \begin{center}
        \begin{tabular}{|c|cc|}
            \hline
                % NOME DA TABELA        
                \rowcolor{cldfB1} \multicolumn{3}{|c|}{\Large \cenFC} \\  
                \rowcolor{cldfB1}
                \multicolumn{3}{|c|}{\large \textbf{Resultados}} \\ \hline \hline
                % CABEÇALHO        
                \rowcolor{lightgray}\textbf{Fornecedor} & \multicolumn{2}{c|}{\textbf{\emph{Score} [1-5]}} \\ \hline
                % CONTEÚDO
                % Código gerado pela tabela do Google SpreadSheets Cenário GB
                \rowcolor{corP1!80}TIBCO Software & \progressbar{0.86} & 4,3 \\ \hline
                \rowcolor{corP2!50}Tableau Software & \progressbar{0.84} & 4,2 \\ \hline
                \rowcolor{corP3!30}MicroStrategy & \progressbar{0.8} & 4,0 \\ \hline
                \rowcolor{corP3!30}Yellowfin & \progressbar{0.8} & 4,0 \\ \hline
                \rowcolor{corPF!20}Birst & \progressbar{0.78} & 3,9 \\ \hline
                \rowcolor{corPF!20}Microsoft & \progressbar{0.74} & 3,7 \\ \hline
                \rowcolor{corPF!20}Sisense & \progressbar{0.74} & 3,7 \\ \hline
                \rowcolor{corPF!20}SAS & \progressbar{0.72} & 3,6 \\ \hline
                \rowcolor{corPF!20}Information Builders & \progressbar{0.7} & 3,5 \\ \hline
                \rowcolor{corPF!20}Qlik & \progressbar{0.64} & 3,2 \\ \hline
                \rowcolor{corPF!20}OpenText & \progressbar{0.54} & 2,7 \\ \hline
                \rowcolor{corPF!20}IBM & \progressbar{0.52} & 2,6 \\ \hline
                \rowcolor{corPF!20}ThoughtSpot & \progressbar{0.44} & 2,2 \\ \hline
        \end{tabular}    
        \caption{\label{tab:cenFC:resultados} Resultados para \cenFC}
        \end{center}
    \end{table}

\subsection*{Análise dos Resultados} 

    E assim chegamos à última análise de resultados. Os resultados obtidos pelo \cenFC \xspace foram:
    
    \begin{itemize}
        \item A \emph{TIBCO Software} lidera em primeiro lugar com $4,3$ pontos;
        \item A \emph{Tableau} aparece em segundo lugar com $4,2$ pontos;
        \item E a \emph{Microstrategy} e \emph{Yellowfin} empatam em terceiro lugar com $4,0$ pontos cada;
        \item Os demais fornecedores não obtém a pontuação mínima e assim são eliminados;
    \end{itemize}
    
    Tal qual ocorreu no \cenGC, o \cenFC \xspace apresenta potenciais fornecedores de Plataformas de BI para satisfazer as necessidades de modernização da fiscalização da CLDF.

% Conclusão
\section{Conclusões}
\label{sec-conclusoes-forrester}

De maneira semelhante ao que foi feito no capítulo anterior, construímos mais três cenários de casos de uso personalizados, dessa vez utilizando os \emph{scores} do \relFCM \xspace para posicionar os principais fornecedores de Plataformas de BI entre si a partir das necessidades identificadas e avaliadas na ``\autoref{parte-necessidades} -- \nameref{parte-necessidades}''.

Assim, de maneira semelhante, apresentamos \textbf{em ordem alfabética} os fornecedores que apareceram pelo menos uma vez entre quaisquer das três primeiras colocações.

\begin{env-destaque}{Fornecedores de destaque \emph{Forrester}:}
 \begin{multicols}{2}
    \begin{itemize}
        \item Birst
        \item Microsoft
        \item MicroStrategy
        \item Tableau Software
        \item TIBCO Software
        \item Yellowfin
    \end{itemize}
 \end{multicols}
\end{env-destaque}

O ``\autoref{cap-fornecedores} -- \nameref{cap-fornecedores}'' vai descrever as principais soluções dos fornecedores destacados nos cenários de casos de uso elaborados.





