\chapter{Testes}

\begin{pergunta}{Teste 1}
	Pergunta 1
\end{pergunta}

\begin{pergunta}{Teste 2}
	Pergunta 2
\end{pergunta}

\begin{pergunta}{Teste 3}
	Pergunta 3
\end{pergunta}

\begin{pergunta}{Teste 4}
	Pergunta 4
\end{pergunta}


\begin{resposta}{Teste 1}
	Resposta 5
\end{resposta}


\printlista{\perglista}

Separa

\printlista{\resplista}





% define few variables that hold some value
\def\One{this is one}
\def\Two{this is two}
\def\Three{this is three}

% add the above commands to a list named `CmdList'
\listcsgadd{CmdList}{One}
\listcsgadd{CmdList}{Two}
\listcsgadd{CmdList}{Three}

% Now loop-over them to print their values
\begin{description}
	\renewcommand*{\do}[1]{\item[#1:] \csuse{#1}}
	\dolistcsloop{CmdList}
\end{description}    

% creating a table needs little bit more trickery; You cannot insert table entries directly.
% You first have to accumulate all table data into some temporary variable and then use that.
\begingroup
\newcommand\tablecontent{}
\def\do#1{\appto\tablecontent{\hline \textbf{#1} & \csuse{#1}\\}}%
\dolistcsloop{CmdList} % collect the data in a table format
\begin{tabulary}{\textwidth}{|L|L|} % now print the collected data 
	\tablecontent \hline
\end{tabulary}              
\endgroup

\section{Funcionalidades}


\setcounter{CounterFuncionalidade}{0}
\begin{itemize}
	\renewcommand*{\do}[1]{\item #1}
	\dolistcsloop{ListaFuncionalidade}
\end{itemize}   


%\hyperlink{TargetFuncionalidade\arabic{CounterFuncionalidade}}{#1}

separa

\setcounter{CounterFuncionalidade}{0}
\begin{itemize}
	\forlistloop{\stepcounter{CounterFuncionalidade} \item \hyperlink{TargetFuncionalidade\arabic{CounterFuncionalidade}}}{\ListaFuncionalidade}
\end{itemize}

%Posso fazer um novo counter aqui e vai funcionar


\hypertarget{TA}{Aqui}

\begin{envteste}[1]{Titulo}
Deu verdade = Foi feito + Nao entra na lista
\end{envteste}
 


\begin{envteste}{Titulo}
Deu falso = Nao foi feito = Entra na lista
\end{envteste}



\section{Comando Novo}

\begin{importante}{Importante 1}
	Teste
\end{importante}


\begin{importante}{Importante 2}
	Teste
\end{importante}


\begin{importante}{Importante 3}
	Teste
\end{importante}


\begin{importante}{Importante 4}
	Teste
\end{importante}


\begin{importante}{Importante 5}
	Teste
\end{importante}


%\setcounter{CounterImportante}{0}
%\begin{itemize}
%	\forlistloop{\stepcounter{CounterImportante} \item %\hyperlink{TargetImportante\arabic{CounterImportante}}}{\ListaImportante}
%\end{itemize}



\setcounter{CounterAmbientes}{0}
\begin{itemize}
	\forlistloop{\stepcounter{CounterAmbientes} \item \hyperlink{TargetFuncionalidade\arabic{CounterAmbientes}}}{\ListaFuncionalidade}
\end{itemize}


\setcounter{CounterFuncionalidade}{0}
\begin{itemize}
	\forlistloop{\stepcounter{CounterFuncionalidade} \item \hyperlink{TargetFuncionalidade\arabic{CounterFuncionalidade}}}{\ListaFuncionalidade}
\end{itemize}

Separa

%\setcounter{CounterAmbientes}{0}
%\begin{itemize}
%	\forlistloop{\stepcounter{CounterAmbientes} \item %\hyperlink{TargetImportante\arabic{CounterAmbientes}}}{\ListaImportante}
%\end{itemize}