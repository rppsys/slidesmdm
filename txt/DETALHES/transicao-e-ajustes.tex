\chapter{Backlog de Ajustes no Sistema}
\label{detalhes:transicao-e-ajustes}

\section*{Ajustes}

Na reunião de hoje vou apresentar essas coisas que precisam ser resolvidas.

\textbf{Evolutivo:}
\begin{itemize}
	\item Solicitar OS - MPAR - Listagem de Ementa;
	\item Alteração dos Nomes das Telas Atuais
	\item Alteração Cenário - OS Assinada - Enviar OS para ASSEL e Gerar Primeiros Ramos da Árvore de Histórico (E aqui entra modo de modelar o BD)
	\item Cenário adicional para Solicitar OS (Quando a Unidade Solicitante é a própria ASSEL)
	\item Log de Auditoria
\end{itemize}


\textbf{Manutenção:}
\begin{itemize}
	\item Tela Gerenciar Perfil - Não está funcionando;
	\item Solicitar Ordem de Serviço - SEI - Tramitação não está funcionando;
\end{itemize}

\section{Evolutivo}


\subsection{Solicitar OS - Campo Solicitante Já vem Preenchido}

Isso é importante.

\begin{importante}{Destacar em qual Unidade está}
	Na tela deve aparecer em algum lugar o nome da Unidade.
	Em letras maiusculas e bem grande.
	
	Para que o usuário ignorante veja.
	
	Na verdade o ideal é que raros usuários possam solicitar para mais de uma Unidade.
	
	A imensa maioria só poderá solicitar para uma única unidade. Isso vai resolver problemas de BIOS.
	
\end{importante}


\subsection{Solicitar OS - MPAR - Listagem de Ementa}

Tem que arrumar isso.

\subsection{Alteração dos Nomes das Telas Atuais}

Ver mapeamento de telas e perfis.



\subsection{Alteração Cenário - OS Assinada}


Idéia do João.

João deu idéia valiosa de como controlar a tramitação da OS. Tirei foto. Está desenhando na prancheta.

Dessa forma conseguimos saber onde está e ainda guardamos o histórico.


\subsection{Comunicação com o SEI}

Exibir mensagem de erro quando não conseguir acessar o SEI;

Gravar log de erro quando não conseguir isso;



\subsection{Cenário adicional para Solicitar OS}

Se necessário, cenário para quando a OS partir da ASSEL.

Na prática isso não deve existir.

ASSEL não deve poder fazer solicitações.


\subsection{Log de Auditoria}

Temos que definir e criar uma tabela de log para gravar ações de interesse do usuário e do sistema. 

\textbf{Gravar:}

\begin{itemize}
	\item Data e Hora;
	\item Tipo de Ator: Usuário, Sistema, Sistema-SEI
	\item Usuário que realizou a ação;
	\item Nome da Máquina de onde a ação foi realizada;
	\item Tela aonde ocorreu a ação;
	\item Tipo de Ação (Erro e outros tipos);
	\item Detalhes;
\end{itemize}

Pedir à THS propor uma modelagem de tabelas para organizar esse log.


\subsection{Correções da Tela de Gerenciar Unidades}

Em tese o ideal é que as Unidades adicionadas sejam somente Unidades Solicitantes.

Eliminar Unidade Interna e Externa. Isso foi sugestão do Pedro lá atrás quando não estava claro ainda como realizariamos o controle de exibição de telas.

Na prática o controle de exibição de telas será realizado de acordo com os Perfis.

\subsection{Gerenciamento de Cor de Fundo para Linhas Baseado em Andamento}

Tabela que contem os tipos de andamento deve ter um atributo cor. Essa cor será usada para pintar o fundo da linha da tabela nas caixas de entrada.





\section{Manutenções}

\subsection{Tela Gerenciar Perfil}

Tela Gerenciar Perfil - Não está funcionando;

Precisamos remover aqueles perfis e criar novos.
Um para cada tela que será criada.

E ao atribuir um Perfil a uma funcionalidade, usuários com aquele Perfil só terão acesso àquelas funcionalidades. Isso não está funcionando.

Criar logo todos os demais Perfis que serão necessários no sistema.

Perfis para acesso de supervisor ou consultor para cada Unidade Interna. Acesso de supervisor ou apoio para ASSEL. Acesso de Solicitante para Solicitante.



Alterar os Perfis para que exista um Perfil para cada possibilidade de Perfil de Uso do Sistema. Ou seja, usar principio MSMS (Simples e Estúpido).

Os perfis criados e cadastrados por padrão (hard-coded) que vão existir são:

\begin{itemize}
	\item ADMINISTRADOR SUPERVISOR
	
	\item  SOLICITANTE - SUPERVISOR
	\item  SOLICITANTE - NORMAL
	\item  SOLICITANTE - SOMENTE-LEITURA
	
	\item  ASSEL - ADMINISTRADOR
	\item  ASSEL - SUPERVISOR
	\item  ASSEL - APOIO
	
	\item  UNIDADE - SUPERVISOR
	\item  UNIDADE - ELABORADOR
	\item  UNIDADE - REVISOR
	\item  UNIDADE - SOMENTE-LEITURA
\end{itemize}

\begin{table}[!h]
	\begin{center}
		\begin{tabular}{|p{0.4\textwidth}|c|c|}
			\hline
			\rowcolor{lightgray!50} \multicolumn{3}{|c|}{\Large PERFIS \normalsize} \\ \hline \hline
			% CABEÇALHO        
			\rowcolor{lightgray}\textbf{Perfil} & \textbf{Permissões} & \textbf{Acesso} \\ \hline
			% CONTEÚDO
			% Código escrito manualmente
			\rowcolor{corCOULD!10} SOLICITANTE - SUPERVISOR  & COMPLETA & LEITURA E ESCRITA  \\ \hline
			\rowcolor{corCOULD!10} SOLICITANTE - APOIO  & LIMITADA & LEITURA E ESCRITA  \\ \hline
			\rowcolor{corCOULD!10} SOLICITANTE - ESPECTADOR  & LIMITADA & SOMENTE LEITURA  \\ \hline 
			\hline
			
			
			\rowcolor{corCOULD!20} ASSEL - SUPERVISOR  & COMPLETA & LEITURA E ESCRITA  \\ \hline
			\rowcolor{corCOULD!20} ASSEL - APOIO  & LIMITADA & LEITURA E ESCRITA  \\ \hline
			\rowcolor{corCOULD!20} ASSEL - ESPECTADOR  & LIMITADA & SOMENTE LEITURA  \\ \hline 
			\hline
			
			\rowcolor{corWOULD!20} UNIDADE - SUPERVISOR  & COMPLETA & LEITURA E ESCRITA  \\ \hline
			\rowcolor{corWOULD!20} UNIDADE - ELABORADOR  & LIMITADA & LEITURA E ESCRITA  \\ \hline
			\rowcolor{corWOULD!20} UNIDADE - ESPECTADOR  & LIMITADA & SOMENTE LEITURA  \\ \hline 
			\hline
			
		\end{tabular}    
		\caption{\label{tab:perfis:mapeamento} Perfis do Sistema.}
	\end{center}
\end{table}



\subsection{Solicitar OS SEI - Tramitação está com Bug}

Solicitar Ordem de Serviço - SEI - Tramitação não está funcionando;

Ao tramitar, o sistema não está recebendo a OS direito.

Isso aqui é complicado e vamos precisar de ajuda do joão.

