\begin{landscape}
\section{Caso 3 - Na Unidade GHI o supervisor é quem atribui os elaboradores e revisores de uma dada solicitação}

\subsection*{Configurações da Unidade GHI}

Suponha a existência de uma unidade denominada \textbf{Unidade GHI} configurada da seguinte forma: 

\CONFIGURAUNID{GHI}{\msnao}{\msnao}{\msnao}{\msnao}

Conforme tabela, a Unidade GHI:
\begin{itemize}
	\item Não permite auto-atribuição de elaboradores;
	\item Não permite auto-atribuição de revisores;
\end{itemize}

Ou seja, a Unidade GHI é a mais conservadora das três unidades e funciona de maneira oposta à Unidade ABC pois não permite auto-atribuição de consultores.

\subsection*{Descrição do Caso}

Chega uma solicitação \SOLU à Unidade GHI e assim ocorre a seguinte sequência de eventos:
\begin{itemize}
	\item O supervisor \SU recebe a solicitação e vai atribuir a solicitação para os consultores \EU e \ED atuarem como \textbf{elaboradores}. Contudo, \EU vai recusar a atribuição.
	\item Depois, outro supervisor, \SD, vai atribuir a solicitação para um outro consultor, \RU, atuar como revisor.
	\item Ao final do trabalho, \SD encaminha a solicitação finalizada para a ASSEL devolver ao solicitante.		
\end{itemize}

Nesse caso, não há auto-atribuição de nenhuma parte até porque as configurações da Unidade GHI não permite isso. As atribuições de elaboradores e revisores só são feitas pelo supervisor da unidade e, assim, precisam passar pela aceitação dos consultores atribuídos. 

\subsection*{Simulação}

\begin{enumerate}
	\item Solicitação \SOLU chega na unidade com estado \euni{Não Lido};

	\GERSOLUNID{CONS008-2023}{Gabinete Delmasso}{CONS}{}{}{}{}{Não}{Distribuição}
	{\euni{Não Lido}}{-}{-}{Indefinido}


	\item Curioso, \EC, que não possui o perfil de supervisor, clica em \bVisualizar para ver do que se trata, mas nenhum estado muda já que ele não é supervisor. Portanto, o estado mantém-se em \euni{Não Lido}.

	\item Mais tarde, \SU, que é supervisor, marca a solicitação na tabela e acessa \bVisualizar fazendo com que o estado da solicitação mude para \euni{Em Análise}:
	
	\GERSOLUNID{CONS008-2023}{Gabinete Delmasso}{CONS}{}{}{}{}{Não}{Distribuição}
	{\euni{Em Análise}}{-}{-}{Indefinido}
	
	
	\item Supervisor \SU clica em \bAnalisar para analisar a solicitação. Decide que a solicitação está correta em ter sido distribuída para a Unidade GHI.	Assim, fecha o modal de análise e clica em \bGerAtrib para escolher os elaboradores. 
	
	\item No modal de gerenciamento de atributos, ele escolhe os CLs \EU e \ED e salva. 
	
		\begin{enumerate}
			\item Neste momento o sistema coloca o estado para \euni{Na Fila} mostrando que a solicitação passou pela análise e entrou para esse estado. 

			\item Além disso, todos os usuários com perfil de \textbf{Supervisor da Unidade GHI} recebem notificações:
			
			\begin{itemize}
				\item ``\SU atribuiu a solicitação \SOLU para \EU \textbf{elaborar}.'' 
				\item ``\SU atribuiu a solicitação \SOLU para \ED \textbf{elaborar}.''
			\end{itemize}

			\item Contudo, a solicitação não entra ainda no estado ``Em Elaboração'' porque pelo menos um dos CLs precisa aceitar a atribuição.			
			
			\item Neste momento, tanto \EU e \ED recebem, nos seus respectivos pools de notificação, uma mensagem padrão do sistema ``Você foi atribuído para ser \textbf{elaborador} da solicitação \SOLU. Por favor, acesse sua área de trabalho para avaliar a atribuição.'' 
		\end{enumerate}	
	
	\GERSOLUNID{CONS008-2023}{Gabinete Delmasso}{CONS}{}{}{}{}{Não}{Distribuição}
	{\euni{Na Fila}}{-}{-}{Indefinido}
	
	\item Embora o estado seja \euni{Na Fila}, não há como nenhum outro CL se inscrever na solicitação por causa das configurações mais restritivas da unidade que não permite auto-inscrição.	
	
	\item Já dentro da Área de Trabalho dos Consultores \EU e \ED aparece uma linha na tabela de solicitações mostrando a solicitação com o estado de \ecl{Aceitar Elaboração} indicando que o CL precisa aceitar ou rejeitar a atribuição de elaboração.
	
	\AREATRABCL{CONS008-2023}{Gabinete Delmasso}{CONS}{}{}{}{}{Não}{Distribuição}
	{\ecl{Aceitar Elaboração}}{-}{-}{\EU}

	\AREATRABCL{CONS008-2023}{Gabinete Delmasso}{CONS}{}{}{}{}{Não}{Distribuição}
	{\ecl{Aceitar Elaboração}}{-}{-}{\ED}

	
	\item \EU está afastado e portanto não faz nada. Assim, na sua Área de Trabalho a solicitação se mantém naquele estado de \ecl{Aceitar Elaboração}.
	
	\item \ED, por sua vez, acessa sua Área de Trabalho, marca a solicitação \SOLU e clica em \bAvaliar fazendo com que o modal de ``Avaliação'' apareça mostrando as propriedades da solicitação bem como o andamento. O modal também indica que ele está sendo atribuído na qualidade de \textbf{Elaborador}. No final do modal há uma aba denominada ``Avaliação'' com duas opções: Aceitar ou Rejeitar. Ele opta por \textbf{Aceitar}.
	
	\item Nesse momento, o sistema verifica que pelo menos um CL aceitou ser o \textbf{elaborador} da solicitação. Então o estado no módulo de Gerenciamento de Unidades passa para \euni{Em Elaboração}. 

	\GERSOLUNID{CONS008-2023}{Gabinete Delmasso}{CONS}{}{}{}{}{Não}{Distribuição}
	{\euni{Em Elaboração}}{\ED}{-}{Indefinido}

	\item Ao mesmo tempo, na Área de Trabalho de \ED em particular, o estado muda para \ecl{Em Elaboração} indicando que aquela solicitação está na carga de trabalho dele. Ele sabe que seu papel ali é o de elaborador porque seu nome aparece na coluna de elaboradores.
	 
	 
	\item Veja que para \EU, o estado continua no estado \ecl{Aceitar Elaboração} visto que ele ainda não aceitou nem rejeitou a atribuição de elaborador daquela solicitação.

	\AREATRABCL{CONS008-2023}{Gabinete Delmasso}{CONS}{}{}{}{}{Não}{Distribuição}
	{\ecl{Aceitar Elaboração}}{-}{-}{\EU}

	\AREATRABCL{CONS008-2023}{Gabinete Delmasso}{CONS}{}{}{}{}{Não}{Distribuição}
	{\ecl{Em Elaboração}}{\ED}{-}{\ED}
	
	\item Pois então, após voltar ao trabalho, \EU verifica sua área de trabalho e encontra a solicitação no estado  ``Aceitar Elaboração''. Ele analisa a solicitação e decide por rejeitar a atribuição. Assim, ele marca a solicitação em sua tabela da área de trabalho, clica em \bAvaliar e no modal de avaliação escolhe a opção \textbf{``Rejeitar''}. Abre-se um modal para ele digitar um texto da justificativa da rejeição. Ele escreve: ``Eu não atuo nessa temática.''.
	
	\item \textbf{Todos os Supervisores da Unidade GHI} recebem no seu pool de notificações uma notificação dizendo ``\EU rejeitou a atribuição de \textbf{elaborador} da solicitação \SOLU com a seguinte justificativa: ``Eu não atuo nessa temática''.
	
	\item Assim, a solicitação \SOLU possui apenas um elaborador e já está com o estado \euni{Em Elaboração} setado indicando que há alguém trabalhando naquela solicitação.
	
	\item Durante o trabalho \ED finalmente entende que seu trabalho está finalizado. Assim, ele acessa a sua Área de Trabalho, marca a solicitação e clica em \bConcluir.
	
	\item Contudo ele é surpreendido com uma mensagem do sistema ``\textbf{Solicitação não pode ser concluída pois não passou por revisão}''. De fato, o sistema não permite que uma solicitação seja concluída se não houver no mínimo um usuário atribuído a ela na qualidade de revisor.
	
	\item Então \ED avisa \SU por whatsapp/email/pessoalmente que a solicitação precisa ser revisada já que até o momento nenhum revisor foi atribuído à solicitação. Contudo, \SU está de férias e pede para \SD, que é a chefe substituta, escolher um revisor.
	
	\item Assim \SD acessa o Módulo Gerenciar Solicitações da Unidade, encontra a solicitação \SOLU e clica em \bGerAtrib. Dentro do modal ele escolhe \RU para ser o revisor e clica em salvar.
	
	\item Contudo, \RU não se torna o revisor da solicitação automaticamente. Da mesma forma que na atribuição do elaborador, \RU deve aceitar ou rejeitar essa atribuição de revisor.
	
	\item Dessa forma, de modo análogo, dentro da Área de Trabalho  do CL \RU aparece uma linha na tabela de solicitações mostrando a solicitação com o estado de \ecl{Aceitar Revisão} indicando que o CL precisa aceitar ou rejeitar a atribuição de \textbf{revisão}.
	
	\AREATRABCL{CONS008-2023}{Gabinete Delmasso}{CONS}{}{}{}{}{Não}{Distribuição}
	{\ecl{Aceitar Revisão}}{\ED}{-}{\RU}
	
	
	\item Dessa feita, \RU marca a solicitação e clica em \bAvaliar para avaliar a atribuição. Opta por \textbf{``Aceitar''}. Esse evento gera uma notificação para todos os supervisores indicando que \RU aceitou ser revisor da solicitação \SOLU.
	
	\AREATRABCL{CONS008-2023}{Gabinete Delmasso}{CONS}{}{}{}{}{Não}{Distribuição}
	{\ecl{Em Elaboração}}{\ED}{\RU}{\RU}
	
	\item Contudo, o estado da solicitação no módulo de gerenciamento de solicitações da unidade não se altera. Continua em \euni{Em Elaboração}. Isso acontece porque a revisão faz parte da elaboração.
	
	\GERSOLUNID{CONS008-2023}{Gabinete Delmasso}{CONS}{}{}{}{}{Não}{Distribuição}
	{\euni{Em Elaboração}}{\ED}{\RU}{Indefinido}	
	
	\item \RU e \ED trabalham juntos até que ambos decidem que o trabalho está finalizado.
	
	\item Cabe a \ED, como elaborador, acessar sua Área de Trabalho, e clicar em \bConcluir. Ele faz isso.
	
	\item Nesse caso, como o sistema identifica que a solicitação \SOLU possui pelo menos um revisor, no caso, \RU. O sistema abre o modal de upload de artefatos permitindo que \ED faça upload do trabalho finalizado. 
	
	\item Feito o upload, o sistema coloca a solicitação no estado \euni{Conclusão da Solicitação}. Todos os supervisores são notificados disso. Esse estado indica que aquele serviço está pronto para ser devolvido para a ASSEL.

	\GERSOLUNID{CONS008-2023}{Gabinete Delmasso}{CONS}{}{}{}{}{Não}{Distribuição}
	{\euni{Conclusão da Solicitação}}{\ED}{\RU}{Conclusão da Solicitação}
	
	\item Nas respectivas áreas de trabalho de todos os envolvidos, as solicitações continuam lá com esse mesmo estado. Alí não há mais nada a ser feito. 

	\AREATRABCL{CONS008-2023}{Gabinete Delmasso}{CONS}{}{}{}{}{Não}{Distribuição}
	{\ecl{Conclusão da Solicitação}}{\ED}{\RU}{\ED}

	\AREATRABCL{CONS008-2023}{Gabinete Delmasso}{CONS}{}{}{}{}{Não}{Distribuição}
	{\ecl{Conclusão da Solicitação}}{\ED}{\RU}{\RU}

	
	\item A solicitação se mantém na tabela do módulo de gerenciamento de solicitações da unidade aguardando ser selecionado e encaminhado para a ASSEL. Essa ação só pode ser realizada por algum supervisor.
	
	\item Após aguns dias, \SD acessa o módulo, marca a solicitação e clica em \bEncaminhar. A solicitação é encaminhada para a ASSEL com andamento de conclusão e desaparece da caixa de entrada da unidade e, portanto, também vai desaparecer das ``Áreas de Trabalho'' de \ED e \RU.
\end{enumerate}

% -------------------------------------------------------------------------------------------------------------------------------------------------------
\end{landscape}

\pagebreak

\section{Conclusão}

Foram apresentados três simulações para estudo da lógica dos fluxos de trabalho no sistema e da interdependência do módulo de gerenciamento de solicitações das unidades, do módulo de área de trabalho do consultor e do sistema de notificações.

