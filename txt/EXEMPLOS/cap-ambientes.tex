\chapter{Ambientes}

\thispagestyle{empty}

Este modelo disponibiliza alguns ``ambientes'', ou seja, caixas de texto com formatação especial para certos tipos de elementos que são automaticamente numerados (e.g. teorema 1.1, teorema 1.2 etc.). Ambientes podem ser criados e configurados por edição do arquivo envs.tex da pasta config.

\section{Exemplos de ambientes disponíveis}

\begin{axiom}
    \LaTeX produz equações mais bonitas que qualquer editor WYSIWYG.
\end{axiom}

\begin{theorem}\textsc{teorema LaTeX-WYSIWYG}
    Todo físico prefere usar código \LaTeX puro que qualquer editor WYSIWYG.
\end{theorem}

\index{\LaTeX} % entrada para o índice remissivo
\index{WYSIWYG}

\begin{demonstration}
    Físicos gostam de equações bonitas. Editores What-You-See-Is-What-You-Get não são apropriados para fazer equações bonitas.\footnote{É certo que há editores WYSIWYG baseados em \LaTeX, mas eles não nos dão o mesmo nível de controle.} Logo, se algum físico preferisse usar um editor WYSIWYG no lugar de \LaTeX, não seria muito inteligente. Como todo físico é inteligente, o teorema está demonstrado \textit{ad absurdum}.
\end{demonstration}

\begin{example}
    Einstein usaria um editor WYSIWYG ou \LaTeX? \\
    Einstein era físico. Portanto, usando o teorema LaTeX-WYSIWYG, concluímos que ele usaria \LaTeX.
\end{example}

\begin{question}
    Einstein usaria um editor WYSIWYG ou \LaTeX?
\end{question}

\begin{solution}
    Einstein era físico. Portanto, usando o teorema LaTeX-WYSIWYG, concluímos que ele usaria \LaTeX.
\end{solution}

\begin{question}
    Marie Curie usaria um editor WYSIWYG ou \LaTeX?
\end{question}

\begin{solution}
    Deixamos esta sem resposta para o estudante se esforçar mais.
\end{solution}

\clearpage

\section{Soluções deste capítulo}

\printsolutions