\chapter{Introdução}

Sistema Eletrônico de Protocolo

O Sistema de Protocolo Eletrônico ou  Protocolo Online, possui diversos nomes, sendo ele conhecido como sistema de protocolo, sistema de gestão de documentos, GED (Gestor Eletrônico de Documentos), Sistema de gestão de processos, entre outros, possui a função de controlar a movimentação e trâmite de documentos, o Sistema de Protocolo Eletrônico estabelece um padrão para catalogar, arquivar e encaminhar todos os arquivos sob responsabilidade para seus determinados setores e demais encaminhamentos, facilitando assim, todo o trâmite de cada documento, desde a entrada até a sua saída, além de armazenar eletronicamente os documentos, garantindo a segurança dos mesmos.

O Sistema de Protocolo Eletrônico permite aos responsáveis acompanhar o andamento e trâmite de cada setor, acompanhar seus prazos e vencimentos. Dessa forma, poder tomar ações para que todos os prazos sejam cumpridos.

Visando a questão de transparência, o sistema permite que o cidadão ou solicitante de um determinado processo, faça consultas externas através da Internet e o número de protocolo de um determinado processo.

Atualmente, diversas prefeituras, câmaras de vereadores e órgãos públicos em geral, utilizam sistemas de protocolos eletrônicos para controlar a entrada e saída de documentos, tanto internos como externo, em vista que possuem uma demanda alta de documentos tramitados entre a prefeitura e a câmara. Muito utilizado também, para controlar o empréstimo de documentos ou para atender solicitações de documentos realizadas por cidadãos.



https://www.plenussistemas.com.br/blog/sistema-de-protocolo-eletronico-o-que-e-e-para-que-ele-serve/