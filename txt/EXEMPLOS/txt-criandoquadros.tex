%\fancyfoot[CO,CE]{}
\fancyfoot[LO]{\small}
\fancyfoot[RO]{\small}
%\fancyfoot[CE]{\Author}
\fancyfoot[LE]{\small}
\fancyfoot[RE]{\small}

\begin{discussion}

\hfill

\begin{center}
  \textsc{\large Criando quadros}\\
\end{center}

\vspace{1ex}

\begin{flushright}
  \textbf{Leonardo Luiz e Castro} \\
  \vspace{1ex}
  \textit{\small Instituto de Física} \\
  \textit{\small Universidade de Brasília} \\
\end{flushright}

\addcontentsline{toc}{section}{Texto complementar -   Criando quadros (Leonardo Luiz e Castro)}

\vspace{1ex}

\section*{O que são quadros?}

Quadros são parecidos com tabelas, com a diferença de que não contêm dados numéricos, mas sim informação textual. Aparentemente, a ABNT é pioneira no mundo em se preocupar com tal distinção.

\section*{Exemplo de quadro em \LaTeX?}

Preferi listar os quadros como figuras. Vi algum livro que fazia assim... Veja como a figura \ref{fig:estados-da-materia} ficou interessante! Lembre-se que só é possível inserir uma figura dentro de um ambiente como este se a opção de não flutuação ([H]) for utilizada.

\begin{figure}[H]
	\centering
	\begin{minipage}{\hsize}
		\centering
		\taburulecolor{white}\arrayrulecolor{white}
		\caption{Quadro de caracterização dos estados da matéria.}
		\label{fig:estados-da-materia}
		\begin{tabular}{| >{\centering\cellcolor{verde_UnB}\color{white}}p{0.11\hsize} | >{\centering\cellcolor{verde_UnB}\color{white}}p{0.12\hsize} | >{\centering\cellcolor{verde_UnB}\color{white}}p{0.12\hsize} | >{\centering\cellcolor{verde_UnB}\color{white}\footnotesize}p{0.11\hsize} | >{\centering\cellcolor{verde_UnB}\color{white}\footnotesize}p{0.16\hsize} | >{\centering\cellcolor{verde_UnB}\color{white}\footnotesize}p{0.17\hsize} |}
			\hline
			{\bf Estado} & {\bf Volume} & {\bf Forma} & {\bf É fluido?} & {\bf É matéria condensada?} & {\bf É compressível?} \tabularnewline
			\hline
		\end{tabular}
		\begin{tabular}{| >{\centering\cellcolor{verde_UnB!70}\color{white}\small}p{0.11\hsize} | >{\centering\cellcolor{verde_UnB!50}\color{black}\small}p{0.12\hsize} | >{\centering\cellcolor{verde_UnB!50}\color{black}\small}p{0.12\hsize} | >{\centering\cellcolor{verde_UnB!50}\color{black}\small}p{0.11\hsize} | >{\centering\cellcolor{verde_UnB!50}\color{black}\small}p{0.16\hsize} | >{\centering\cellcolor{verde_UnB!50}\color{black}\small}p{0.17\hsize} |}
			\hline
			Gasoso & ajustável & ajustável & sim & não & muito \tabularnewline
			\hline
			Líquido & fixo & ajustável & sim & sim & pouco \tabularnewline
			\hline
			Sólido & fixo & fixa & não & sim & não \tabularnewline
			\hline
		\end{tabular}
		\source{elaboração do autores de Física para Ciências Agrárias e Ambientais (Leonardo Castro e Olavo Filho).}
	\end{minipage}
\end{figure}

\end{discussion}
