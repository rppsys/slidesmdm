\chapter{Introdução}

\thispagestyle{empty} 

Este modelo tem como base as instruções da Editora UnB para a editoração dos livros da série \textit{Ensino de Graduação}. Autores de ciências naturais ou exatas gostam de usar \LaTeX, sistema de editoração que não é muito por editoras dedicadas às humanidades. Dá um certo trabalho adequar um livro às regras da ABNT em \LaTeX. Espero que este modelo encoraje outras pessoas a submeterem seus livros à Editora UnB.

\begin{mdframed}[style=noteSty]

{\center \textsc{Texto motivador} \par}

   Este é um breve texto motivador. Espero que você esteja motivado(a)!
   
\end{mdframed}

Muitos cientistas gostam de usar \LaTeX porque essa ferramenta possibilita escrever facilmente equações como a seguinte:
\begin{equation}
 \mathscr{p}+\frac{1}{2}{\rho}v^2+{\rho}gh = \text{constante}
 \label{eq:Bernoulli}
\end{equation}
onde $\mathscr{p}$ é a pressão, $v$ é a velocidade e $h$ é a elevação, ou seja, a “altura do tubo”. Essa equação pode ser deduzida a partir do \textit{Teorema Trabalho-Energia}. \index{Teorema Trabalho-Energia}  % entrada para o índice remissivo

\newpage

Note como a numeração das páginas começa aqui.