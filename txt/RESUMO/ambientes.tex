\chapter{Ambientes}
\newcounter{CounterAmbientes}

\TODO{Último}

\toGil{Último}

\toPedro{Último}

\begin{pergunta}{Última}
	Última
\end{pergunta}

\begin{resposta}{Última}
	Última
\end{resposta}

\begin{importante}{Última}
	Última
\end{importante}

\begin{requisito}{Última}
	Última
\end{requisito}

\begin{funcionalidade}{Última}
	Última
\end{funcionalidade}

\begin{exemplo}{Última}
	Última
\end{exemplo}

\begin{nota}{Última}
	Última
\end{nota}

\begin{imagine}{Última}
	Última
\end{imagine}

\begin{ajuste}{Último}
	Última
\end{ajuste}

\begin{falha}[1]{Último}
	Antes
	\tcblower
	Depois
\end{falha}


\begin{falha}[0]{Último}
	Antes
	\tcblower
	Depois
\end{falha}

\begin{evolutivo}[0]{Último}
	Evolutivo
\end{evolutivo}


Resumo dos principais \hypertarget{TargetResumoAmbientes}{ambientes}:

\section{Afazeres}

Afazeres é qualquer manifestação de algo que precisa ser feito em algum momento. Não sei ainda aonde se encaixa mas precisos deixar escrito para não esquecer e depois eu reavalio esse afazer.

\subsection{Gerais}

\listarAmbiente[TargetTodo]{\ListaTodo}

\subsection{Para Gilberto}

\listarAmbiente[TargetTogil]{\ListaTogil}

\subsection{Para Pedro}

\listarAmbiente[TargetTopedro]{\ListaTopedro}

\section{Perguntas}

São dúvidas e perguntas que preciso fazer para alguém me responder. Geralmente esse alguém é o Líder de Negócios ou integrantes da área de negócios.

\listarAmbiente[TargetPergunta]{\ListaPergunta}

\section{Respostas}

São respostas às perguntas. Deixo registrado.

\listarAmbiente[TargetResposta]{\ListaResposta}

\section{Importante}

Qualquer coisa que queira destacar como importante.

\listarAmbiente[TargetImportante]{\ListaImportante}

\section{Requisitos}

Nascem dentro das entrevistas.

\listarAmbiente[TargetRequisito]{\ListaRequisito}

\section{Funcionalidades}

Já são declarações mais consolidadas de coisas que precisam acontecer.

\listarAmbiente[TargetFuncionalidade]{\ListaFuncionalidade}

\section{Exemplos}

Exemplo de algo.

\listarAmbiente[TargetExemplo]{\ListaExemplo}

\section{Nota}

Notas sobre qualquer coisa.

\listarAmbiente[TargetNota]{\ListaNota}

\section{Imagine}

Descrição de uma possibilidade. Um cenário para servir de inspiração.

\listarAmbiente[TargetImagine]{\ListaImagine}


\section{Ajustes}

Preocupações relacionadas a coisas já desenvolvidas e que podem levar a erros. 

\listarAmbiente[TargetAjuste]{\ListaAjuste}


\section{Falhas}

Falhas encontradas nos testes.

\listarAmbiente[TargetFalha]{\ListaFalha}

\section{Evolutivos}

São funcionalidades evolutivas identificadas.

\listarAmbiente[TargetEvolutivo]{\ListaEvolutivo}





