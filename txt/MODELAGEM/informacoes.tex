\chapter{Informações Coletadas}

%Depois de consolidado cada seção aqui tem que virar uma nota técnica no SEI.

\section{Unidades Internas}

\begin{itemize}
	\item ASSEL - Assessoria Legislativa;
	\item UCJ - Unidade de Constituição e Justiça;
	\item URP - Unidade de Redação Parlamentar e Consolidação dos Textos Legislativos;
	\item UEF - Unidade de Economia e Finanças;
	\item USE - Unidade de Saúde, Educação, Cultura e Desenvolvimento Científico e Tecnológico; 
	\item UDA - Unidade de Desenvolvimento Urbano e Rural e Meio Ambiente;
\end{itemize}


\section{Unidades Externas}

São quaisquer unidades do SEI que forem cadastradas no sistema e atuam somente como Solicitantes.


\section{Tipos de Solicitações}

%\TODO{Consolidar e subir para SEI como Nota Técnica}

As demandas recebidas pela ASSEL classificam-se no \textbf{Tipos de Solicitação} a seguir:

%Isto foi me enviado pelo Gilberto no dia 27/04/21.

\begin{itemize}
	\item Consulta (CONS)
	\item Estudo (EST)
	\item Minuta de Pronunciamento / Discurso (PRO)
	\item Minuta de Parecer
	\begin{itemize}
		\item Denúncia (DEN)
		\item Indicação (IND)
		\item Mensagem (MENS) 
		\item Projeto de Decreto Legislativo (PDL)
		\item Proposta de Emenda à Lei Orgânica (PELO)
		\item Projeto de Lei (PL)
		\item Projeto de Lei Complementar (PLC) 
		\item Projeto de Resolução (PR)
		\item Moção (MOC)
		\item Recurso (REC)
		\item Requerimento (RQ)		
    \end{itemize}


	\item Minuta de Proposição (MP)
	\begin{itemize}
		\item Denúncia (DEN)
		\item Indicação (IND)
		\item Mensagem (MENS) 
		\item Projeto de Decreto Legislativo (PDL)
		\item Proposta de Emenda à Lei Orgânica (PELO)
		\item Projeto de Lei (PL)
		\item Projeto de Lei Complementar (PLC) 
		\item Projeto de Resolução (PR)
		\item Moção (MOC)
		\item Recurso (REC)
		\item Requerimento (RQ)		
    \end{itemize}

	\item Nota Técnica (NT)
	\item Relatório de Veto (RV)
	\item Outros (OUT)
\end{itemize}

\begin{importante}{Mudar aqui muda lá}
	Mudar esses tipos tem que definir novos códigos em \ref{ref:codigos}.
\end{importante}

\section{Módulos}

	Verificamos dois principais necessidades da \ASSEL:
	
	\begin{itemize}
		\item  \textbf{Sistema de Protocolo} - 	A necessidade da \ASSEL \xspace é de um sistema de protocolo integrado ao SEI que controle as atividades da unidade.  
		
		\item \textbf{Controle e Distribuição dos Artefato(s)} - Necessidade de controlar a forma como os artefato(s) desenvolvidos são distribuidos para os seus solicitantes.
	\end{itemize}

	Assim, acredita-se que para fins de desenvolvimento o sistema pode ser desenvolvidos em dois módulos.

\subsection{Módulo I - Sistema de Protocolo}

	Num primeiro momento seria desenvolvido um \textbf{sistema de protocolo integrado ao SEI} para controlar o fluxo de atividades da \ASSEL.
	
	Neste primeiro momento, a \textbf{forma de distribuição} dos artefato(s) criados pela \ASSEL não seria objeto de desenvolvimento. Isto é, a \ASSEL continuaria fazendo o controle e distribuição de Artefato(s) da maneira antiga, isto é, manualmente, utilizando a estrutura de rede da casa.   


\subsection{Módulo II - Controle e Distribuição de Artefato(s) produzidos}

	Em um segundo momento, após ter uma versão do Módulo I desenvolvida, em produção, funcionando com recursos mínimos, passaríamos a focar em fazer o \textbf{Controle e Distribuição de Artefato(s) produzidos}.
	
	Assim, neste momento, a distribuição dos artefato(s) produzidos deixariam de ser uma tarefa manual executada pela equipe de Apoio da \ASSEL e passaria a ser executada pelo próprio sistema.
	
	Em um experimento de imaginação, quando o \CL finalizasse uma demanda, ele faria \emph{upload} do(s) arquivos produzidos para o sistema e na outra ponta, o Solicitante faria o \emph{download} desses arquivos.
	
	Neste momento, as preocupações com os requisitos não funcionais de ``Sigilo'' passariam a ser considerados. 	


\section{Requisitos Funcionais}


\subsection{Requisito: Capacidades de Pesquisa}

Trata-se da capacidade de realizar pesquisas como:

\begin{itemize}
	\item Pesquisa textual dentro de todos os documentos já produzidos pela \ASSEL.
	
	\item Pesquisar os metadados de processos já tramitados pela Assel e que originaram artefato(s).	
\end{itemize}

\subsection{Requisito: Monitorar Estado das Solicitações}

Capacidade de permitir que o solicitante saiba onde o processo está e receba informações sobre o estado da solicitação.

\section{Requisitos Não Funcionais}

\subsection{ReqNFunc: Integração com o SEI}

Uma das restrições do projeto previsto no TAP é que o Sistema não pode substituir o SEI. Então um requisito não funcional muito importante é que o Sistema funcione juntamente com o SEI.


\subsection{ReqNFunc: Sigilo}

Da elaboração até a entrega dos artefato(s) aos Solicitantes, esses artefato(s) são sigilosos. Ou seja, só ganhará acesso ao artefato quem demandou.

\subsection{ReqNFunc: Acesso via Internet}

	O Sistema deve eliminar a necessidade de realizar o Acesso Remoto nos computadores. O sistema deve funcionar on-line na Internet mediante autenticação do usuário (login).

