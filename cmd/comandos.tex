\newcommand{\toPrev}{}
\newcommand{\toProx}{}

\newcommand{\red}[1]{\textcolor{red}{\textbf{#1}}}

\newcommand{\DV}[1]{\colorbox{orange}{\textbf{Dúvida}: #1}}

\newcommand{\WR}[1]{
	\begin{center}
		\colorbox{red}{#1}
	\end{center}
}

\newcommand{\CB}[1]{
    \begin{center}
	    \textbf{#1}
    \end{center}
}

\newcommand{\RPP}[1]{
    \begin{center}
	    \colorbox{green!10}{#1}
    \end{center}
}

\newcommand{\sigla}[1]{\textbf{\textsc{#1}}}


% ----------------------------------    
\newcounter{CounterTodo}
\setcounter{CounterTodo}{0}
\newcommand{\TODO}[2][0]
{
	\ifthenelse{#1 = 0}{\renewcommand{\auxCheck}{\msrln}}{\renewcommand{\auxCheck}{\msrls}}
	
	%\renewcommand{\toPrev}{\hyperlink{TargetTodo\arabic{CounterTodo}}{<<}}
		
	\stepcounter{CounterTodo}		
	
	\renewcommand{\auxTitulo}{\textbf{\auxCheck \xspace TODO \xspace \arabic{CounterTodo}}: \hypertarget{TargetTodo\arabic{CounterTodo}}{\nohyphens{#2}} \toPrev}
	
    \begin{center}
		\colorbox{blue!10}{\auxTitulo}
		\listcsgadd{ListaTodo}{\ifthenelse{#1 = 0}{\msrln}{\msrls} #2}	
	\end{center}
}
% -----------------------------------------------------------
\newcounter{CounterTogil}
\setcounter{CounterTogil}{0}
\newcommand{\toGil}[2][0]
{
	\ifthenelse{#1 = 0}{\renewcommand{\auxCheck}{\msrln}}{\renewcommand{\auxCheck}{\msrls}}
	
	%\renewcommand{\toPrev}{\hyperlink{TargetTogil\arabic{CounterTogil}}{<<}}
	
	\stepcounter{CounterTogil}		
	
	\renewcommand{\auxTitulo}{\textbf{\auxCheck \xspace toGil \xspace \arabic{CounterTogil}}: \hypertarget{TargetTogil\arabic{CounterTogil}}{\nohyphens{#2}} \toPrev}
	
	\begin{center}
		\colorbox{blue!10}{\auxTitulo}
		\listcsgadd{ListaTogil}{\ifthenelse{#1 = 0}{\msrln}{\msrls} #2}	
	\end{center}
}
% -----------------------------------------------------------
\newcounter{CounterTopedro}
\setcounter{CounterTopedro}{0}
\newcommand{\toPedro}[2][0]
{
	\ifthenelse{#1 = 0}{\renewcommand{\auxCheck}{\msrln}}{\renewcommand{\auxCheck}{\msrls}}
	
	%\renewcommand{\toPrev}{\hyperlink{TargetTopedro\arabic{CounterTopedro}}{<<}}
	
	\stepcounter{CounterTopedro}		
	
	\renewcommand{\auxTitulo}{\textbf{\auxCheck \xspace toPedro \xspace \arabic{CounterTopedro}}: \hypertarget{TargetTopedro\arabic{CounterTopedro}}{\nohyphens{#2}} \toPrev}
	
	\begin{center}
		\colorbox{blue!10}{\auxTitulo}
		\listcsgadd{ListaTopedro}{\ifthenelse{#1 = 0}{\msrln}{\msrls} #2}	
	\end{center}
}
% -----------------------------------------------------------

% ----------------------------------
% Comando Genérico para Imprimir Listas
% ----------------------------------

\newcommand{\printlista}[2][\\]{{
		\def\listsep{\def\listsep{#1}}
		\renewcommand{\do}[1]{\listsep ##1}
		\noindent \dolistloop#2		
}}


% -----------------------------------------------
% Colorir texto com quebra de linhas automáticas
% ------------------------------------------------

\newcommand{\ctextrgb}[3][RGB]{%
	\begingroup
	\definecolor{hlcolor}{#1}{#2}\sethlcolor{hlcolor}%
	\hl{#3}%
	\endgroup
}
%Uso: \ctextrgb[RGB]{153,221,231}{Colorido}

\newcommand{\ctextcolor}[2]{%
	\begingroup
	\colorlet{usercolor}{#1}
	\sethlcolor{usercolor}\hl{#2}%
	\endgroup
}
%Uso: \ctextcolor{orange}{Colorido}

% ----------------------------------
%           Ger Sol Unid
% ----------------------------------        

\newcommand{\GERSOLUNID}[9]{
	\def\argI{#1} 
	\def\argII{#2} 
	\def\argIII{#3} 
	\def\argIV{#4} 
	\def\argV{#5} 
	\def\argVI{#6} 
	\def\argVII{#7} 
	\def\argVIII{#8} 
	\def\argIX{#9} 
	\GERSOLUNIDCMD
}

\newcommand{\GERSOLUNIDCMD}[4]{
	% ArgI: Solicitação
	% ArgII: Unidade Solicitante
	% ArgIII: Tipo
	% ArgIV: Subtipo
	% ArgV: Prop. Numero
	% ArgVI: Prop. Ano
	% ArgVII: Comissão
	% ArgVIII: Urgente
	% ArgIX: Andamento	 
	% Arg 10 - #1: Estado
	% Arg 11 - #2: Elaborador(es) 
	% Arg 12 - #3: Revisor(es) 
	% Arg 13 - #4: Ação
\begin{center}
	\scalebox{0.8}{
	\begin{tabularx}{22cm}{|c|c|X|X|}
		\hline
		\rowcolor{yellow!70} \multicolumn{4}{|c|}{ \textbf{Gerenciar Solicitações Unidade}} \\ \hline
		
		% CABEÇALHO
		\rowcolor{yellow!20} \textbf{Solicitação} & \textbf{Estado} & \textbf{Elaborador(es)} & \textbf{Revisor(es)} \\ \hline

		% CONTEUDO
		\rowcolor{cldfG!10} \mssim \argI & #1 & #2 & #3 \\ \hline
		% -----			
	\end{tabularx}   
	} 
\end{center}
}

% ----------------------------------
%           Área de Trabalho do CL
% ----------------------------------        

\newcommand{\AREATRABCL}[9]{
	\def\argI{#1} 
	\def\argII{#2} 
	\def\argIII{#3} 
	\def\argIV{#4} 
	\def\argV{#5} 
	\def\argVI{#6} 
	\def\argVII{#7} 
	\def\argVIII{#8} 
	\def\argIX{#9} 
	\AREATRABCLCMD
}

\newcommand{\AREATRABCLCMD}[4]{
	% ArgI: Solicitação
	% ArgII: Unidade Solicitante
	% ArgIII: Tipo
	% ArgIV: Subtipo
	% ArgV: Prop. Numero
	% ArgVI: Prop. Ano
	% ArgVII: Comissão
	% ArgVIII: Urgente
	% ArgIX: Andamento	 
	% Arg 10 - #1: Estado
	% Arg 11 - #2: Elaborador(es) 
	% Arg 12 - #3: Revisor(es) 
	% Arg 13 - #4: CL
	\begin{center}
		\scalebox{0.8}{
		\begin{tabularx}{22cm}{|c|c|X|X|}
			% -----	
			\hline
			\rowcolor{orange!70} \multicolumn{4}{|c|}{\textbf{Área de Trabalho de #4}} \\ \hline
			
			% CABEÇALHO
			\rowcolor{orange!20} \textbf{Solicitação} & \textbf{Estado} & \textbf{Elaborador(es)} & \textbf{Revisor(es)} \\ \hline
			
			% CONTEUDO
			\rowcolor{cldfG!10} \mssim \argI & #1 & #2 & #3 \\ \hline
			% -----	
		\end{tabularx}
		}    
	\end{center}
}

% ----------------------------------
%           Configuração das Unidades
% ----------------------------------        

\newcommand{\CONFIGURAUNID}[5]{
	\begin{center}
	\scalebox{0.8}{
		\begin{tabularx}{\textwidth}{|c|X|}
			% -----	
			\hline
			\rowcolor{cldfA!70} \multicolumn{2}{|c|}{\textbf{Configurações da Unidade #1}} \\ \hline
			
			% CABEÇALHO
			\rowcolor{cldfA!20} & \textbf{Configuração} \\ \hline
			
			% CONTEUDO
			\rowcolor{cldfD!30} #2 & Permitir auto-atribuição de \textbf{elaboradores} \\ \hline
			\rowcolor{cldfD!30} #3 & Ao permitir auto-atribuição de \textbf{elaborador(es)}, dispensar consentimento do supervisor \\ \hline
			\rowcolor{cldfG!30} #4 & Permitir auto-atribuição de \textbf{revisor(es)} \\ \hline
			\rowcolor{cldfG!30} #5 & Ao permitir auto-atribuição de \textbf{revisor(es)}, dispensar consentimento do supervisor \\ \hline
			% -----	
		\end{tabularx}
	}    
	\end{center}
}

