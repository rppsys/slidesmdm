% ####
% APRESENTAÇÃO
% ####

\newpage
~\vfill
\thispagestyle{empty}

\noindent \textsc{Curso Técnico em Eletrotécnica, Instituto Federal Fluminense - Campus Itaperuna}\\

\noindent 
Caro estudante,

Bem-vindos à disciplina de Meio Ambiente e Energias Renováveis que tratará [...] para aprofundar seus conhecimentos sobre [...] no curso Técnico em Eletrotécnica do Instituto Federal Fluminense – Campus Itaperuna.

Para que seu estudo se torne proveitoso e prazeroso, esta disciplina foi organizada em [...] capítulos, com temas e subtemas que, por sua vez, são subdivididos em seções (tópicos), atendendo aos objetivos do processo de ensino-aprendizagem.
O capítulo 1, que trata [...], procuraremos compreender [...].  No capítulo 2, descreveremos
[...]. No capítulo 3, detalharemos [...]. Finalmente, no capítulo 4 refletiremos um pouco sobre [...]. Esperamos que, até o final da disciplina vocês possam:
- Ampliar a compreensão sobre [...];
- Conhecer [...];
- Identificar os aspectos [...];
- Compreender a importância [...];
Para tanto, a metodologia das aulas [...].

Porém, antes de iniciar a leitura, gostaríamos que vocês parassem um instante para refletir sobre algumas questões [...].
Não se preocupe.  Não queremos que vocês respondam de imediato todas essas questões.  Mas esperamos que, até o final, vocês tenham respostas e também formulem outras perguntas.

Vamos, então, iniciar nossas aulas? Bons estudos!

\vspace{3cm}
\DTMlangsetup{showdayofmonth=false}
\noindent \textit{1ª Ed., \today }
\DTMlangsetup{showdayofmonth=true}
